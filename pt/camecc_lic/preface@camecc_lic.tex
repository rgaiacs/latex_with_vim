% Filename: preface@cammec_lic.tex
% This code is part of 'Cursos CAMECC: Introducao ao LaTeX para o Curso 29 - Licenciatura em Matematica'
% 
% Description: This file correspond to the preface of the textbook using in the course.
% 
% Created: 07.06.12 11:30:49 AM
% Last Change: 07.06.12 11:30:53 AM
% 
% Authors:
% - Raniere Silva, r.gaia.cs@gmail.com
% 
% Organization: CAMECC - Centro Academico dos Estudantes do IMECC
% 
% Copyright (c) 2012, Raniere Silva. All rights reserved.
% 
% This work is licensed under the Creative Commons Attribution-ShareAlike 3.0 Unported License. To view a copy of this license, visit http://creativecommons.org/licenses/by-sa/3.0/ or send a letter to Creative Commons, 444 Castro Street, Suite 900, Mountain View, California, 94041, USA.
%
% This work is distributed in the hope that it will be useful, but WITHOUT ANY WARRANTY; without even the implied warranty of MERCHANTABILITY or FITNESS FOR A PARTICULAR PURPOSE.
%
\chapter{Pref\'{a}cio}
Esse mat\'{e}ria foi desenvolvido para um mini-curso voltados aos aulos do curso de Licenciatura em Matem\'{a}tica da Universidade Estadual de Campinas (UNICAMP).

O objetivo do mini-curso \'{e} apresentar o LaTeX aos futuros licenciados em matem\'{a}tica e ajud\'{a}-los a dominar essa poderosa ferramenta de trabalho.

\nocite{LaTeX_Project:EN:Home}Quando me perguntam por que utilizar o LaTeX eu respondo:
\begin{quote}
    \begin{enumerate}
        \item \'{E} uma ferramenta livrer.
        \item \'{E} bastante est\'{a}vel (lan\c{c}ado em 1985 por Laslie Lamport e baseado no TeX que foi lan\c{c}ado por Donald Knuth em 1978).
        \item E possue uma \'{o}tima qualidade tipogr\'{a}fica, i.e., muito bonito.
    \end{enumerate}
\end{quote}
Al\'{e}m dos tr\^{e}s motivos mencionados acima ainda posso dizer que
\begin{itemize}
    \item \'{e}, usualmente, utilizado na produ\c{c}\~{a}o dos mais variados documentos t\'{e}cnico e cient\'{i}ficos,
    \item e encoraja o autor a preocupar-se apenas com o conte\'{u}do.
\end{itemize}

O mini-curso foi preparado para ser ministrado em quatro horas sendo que cada hora deve cobrir um dos cap\'{i}tulos. Al\'{e}m dos cap\'{i}tulos encontra-se no ap\^{e}ndice uma preve hist\'{o}ria de fatos importantes na \'{a}rea de computa\c{c}\~{a}o que ajudam a entender o surgimento do LaTeX, uma explica\c{c}\~{a}o t\'{e}cnica do LaTeX e dicas de locais para procurar ajuda, e alguns exerc\'{i}cios.
