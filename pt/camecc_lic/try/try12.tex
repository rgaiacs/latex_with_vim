% Filename: try12.tex
% This code is part of 'Cursos CAMECC: Introducao ao LaTeX para o Curso 29 - Licenciatura em Matematica'
% 
% Description: This file correspond exercise 12 of the textbook using in the course.
% 
% Created: 07.06.12 11:30:10 AM
% Last Change: 07.06.12 11:30:30 AM
% 
% Authors:
% - Raniere Silva, r.gaia.cs@gmail.com
% 
% Organization: CAMECC - Centro Academico dos Estudantes do IMECC
% 
% Copyright (c) 2012, Raniere Silva. All rights reserved.
% 
% This work is licensed under the Creative Commons Attribution-ShareAlike 3.0 Unported License. To view a copy of this license, visit http://creativecommons.org/licenses/by-sa/3.0/ or send a letter to Creative Commons, 444 Castro Street, Suite 900, Mountain View, California, 94041, USA.
%
% This work is distributed in the hope that it will be useful, but WITHOUT ANY WARRANTY; without even the implied warranty of MERCHANTABILITY or FITNESS FOR A PARTICULAR PURPOSE.
%
\documentclass[12pt, a4paper]{article}
\usepackage[utf8]{inputenc}
\usepackage[T1]{fontenc} 
\usepackage[top=3cm,left=2cm,right=2cm,bottom=3cm]{geometry}
\usepackage[brazil]{babel}
\usepackage{url}

\title{Taxa de juros para financiamento de imóveis cai hoje}
\author{MARIANA SALLOWICZ}
\date{04/05/2012 às 07h30}

\begin{document}
\maketitle

A Caixa Econômica Federal, maior agente financeiro no setor de habitação, reduz os juros dos financiamentos imobiliários a partir de hoje. Começa também a rodada de feirões da casa própria.

Mais de 430 mil imóveis novos, usados e na planta serão vendidos em 13 cidades até 10 de junho. Os primeiros municípios a receber o evento --de hoje a domingo-- são Belo Horizonte, Brasília, Rio, Salvador e Recife.

Em São Paulo, o feirão vai de 18 a 20 deste mês.

``A vantagem do evento é reunir em um local uma grande oferta de construtoras, que permitem comparar as ofertas'', diz José Urbano Duarte, vice-presidente de governo e habitação da Caixa.

Além das empresas, há outros agentes da cadeia, como corretores, cartórios e técnicos do banco responsáveis por liberar financiamentos.

Existe a possibilidade de fechar o negócio na hora, mas especialistas recomendam cautela. O pagamento vai comprometer a renda do mutuário por até 30 anos.

\subsubsection*{FAÇA SUAS CONTAS}

``É preciso fazer as contas antes, saber o quanto do orçamento há disponível para fazer o financiamento sem se endividar'', diz o professor do Insper Ricardo Almeida.

As taxas da Caixa vão de 4,5\% a 10\% ao ano mais Taxa Referencial, de acordo com o valor de imóvel e a renda. Antes, chegava a 11\%.

No caso dos mutuários que adquirem um imóvel avaliado em até R\$ 500 mil, as taxas serão reduzidas de 10\% para pelo menos 9\% ao ano.

Para quem tem conta-corrente, cheque especial e cartão de crédito do banco, os juros podem chegar a 8,4\%. Já os clientes que optarem por transferir o salário para Caixa podem ter até 7,9\%.

Imóveis com valores superiores a R\$ 500 mil terão taxas de financiamento reduzidas de 11\% ao ano para 10\% ao ano, podendo chegar a 9\% ao ano de acordo com os produtos e os serviços da Caixa que os clientes usarem.

Na linha que usa os recursos do FGTS, a taxa máxima foi reduzida de 8,4\% para 7,9\%. Se o cliente tiver conta no fundo de garantia (caso de assalariados), fica em 7,4\%. A modalidade é válida para compra de imóveis de no máximo R\$ 170 mil e famílias com renda de até R\$ 5.400.
	

\begin{flushright}
Retirado de \url{http://www1.folha.uol.com.br/mercado/1085326-taxa-de-juros-para-financiamento-de-imoveis-cai-hoje.shtml}
\end{flushright}
\end{document}
