% Filename: try11.tex
% This code is part of 'Cursos CAMECC: Introducao ao LaTeX para o Curso 29 - Licenciatura em Matematica'
% 
% Description: This file correspond exercise 11 of the textbook using in the course.
% 
% Created: 07.06.12 11:30:10 AM
% Last Change: 07.06.12 11:30:30 AM
% 
% Authors:
% - Raniere Silva, r.gaia.cs@gmail.com
% 
% Organization: CAMECC - Centro Academico dos Estudantes do IMECC
% 
% Copyright (c) 2012, Raniere Silva. All rights reserved.
% 
% This work is licensed under the Creative Commons Attribution-ShareAlike 3.0 Unported License. To view a copy of this license, visit http://creativecommons.org/licenses/by-sa/3.0/ or send a letter to Creative Commons, 444 Castro Street, Suite 900, Mountain View, California, 94041, USA.
%
% This work is distributed in the hope that it will be useful, but WITHOUT ANY WARRANTY; without even the implied warranty of MERCHANTABILITY or FITNESS FOR A PARTICULAR PURPOSE.
%
\documentclass[12pt, a4paper]{article}
\usepackage[utf8]{inputenc}
\usepackage[T1]{fontenc} 
\usepackage[top=3cm,left=2cm,right=2cm,bottom=3cm]{geometry}
\usepackage[brazil]{babel}
\usepackage{url}

\title{Bradesco e Itaú vão reduzir taxas de juros}
\author{REDAÇÃO ÉPOCA COM AGÊNCIA BRASIL}
\date{18/04/2012 às 14h42}

\begin{document}
\maketitle

Os bancos Bradesco e Itaú anunciaram nesta quarta-feira (18) que vão reduzir as taxas de juros a seus clientes. Nas últimas semanas, Banco do Brasil, Caixa Econômica Federal, HSBC e Santander também anunciaram reduções nas taxas de juros. As decisões foram feitas depois da presidente Dilma Rousseff defender a redução do ``spread'', que é a diferença entre o que os bancos brasileiros pagam para captar o dinheiro do investidor e o que eles cobram dos tomadores de empréstimos.

Na quinta-feira (12), o ministro da Fazenda, Guido Mantega também fez críticas aos bancos privados por não reduzir as taxas e cobrar altos spreads. Hoje, a taxa básica de juro está em 9,75\% ao ano, um dos níveis mais baixos da história. Mas esse valor não tem efeito na vida do cidadão comum. Os juros cobrados no cheque especial chegam a 300\% ao ano, as taxas de empréstimos para negócios de menor porte estão na faixa de 60\% ao ano e o spread brasileiro não tem precedente em nenhum outro lugar do mundo.

No Bradesco, a taxa mínima do crédito pessoal cairá de 2,66\% para a partir de 1,97\% ao mês. Na linha CDC Bens, a taxa será reduzida de 3,54\% para a partir de 2,97\% ao mês. No caso do financiamento de veículos, a taxa, que era 1,35\%, passará a ser a partir de 0,97\% ao mês. Nas operações de crédito consignado para os aposentados, o Bradesco reduziu a taxa de 1,32\% para a partir de 0,9\% ao mês. Os cartões de crédito emitidos em parceria com redes varejistas terão taxas para parcelamento com juros a partir de 2,49\% ao mês, com prazo de até 24 meses. Além da redução das taxas, o Bradesco informou que ampliou o limite de crédito em mais R\$ 15 bilhões, sendo R\$ 9 bilhões para pessoas físicas e R\$ 5 bilhões para pessoas jurídicas.

No Itaú, no caso de financiamento de veículos, a taxa mínima sofrerá redução de 8\% e será de 0,99\% ao mês. A taxa será válida para clientes correntistas há mais de um ano, em operações com 50\% de entrada e parcelamento em até 24 meses. Nos empréstimos consignados para beneficiários do INSS, a taxa mínima foi reduzida para 0,89\%, e a máxima, para 2,2\% ao mês.

As novas taxas de juros e limites, sujeitos a aprovação de crédito, entram em vigor na segunda-feira (23).

\begin{flushright}
Retirado de \url{http://revistaepoca.globo.com/Negocios-e-carreira/noticia/2012/04/bradesco-e-itau-vao-reduzir-taxas-de-juros.html}
\end{flushright}
\end{document}
