% Filename: chap03@camecc_lic.tex
% This code is part of 'Cursos CAMECC: Introducao ao LaTeX para o Curso 29 - Licenciatura em Matematica'
% 
% Description: This file correspond to the chapter 01 of the textbook using in the course.
% 
% Created: 07.06.12 11:30:49 AM
% Last Change: 07.06.12 11:30:53 AM
% 
% Authors:
% - Raniere Silva, r.gaia.cs@gmail.com
% 
% Organization: CAMECC - Centro Academico dos Estudantes do IMECC
% 
% Copyright (c) 2012, Raniere Silva. All rights reserved.
% 
% This work is licensed under the Creative Commons Attribution-ShareAlike 3.0 Unported License. To view a copy of this license, visit http://creativecommons.org/licenses/by-sa/3.0/ or send a letter to Creative Commons, 444 Castro Street, Suite 900, Mountain View, California, 94041, USA.
%
% This work is distributed in the hope that it will be useful, but WITHOUT ANY WARRANTY; without even the implied warranty of MERCHANTABILITY or FITNESS FOR A PARTICULAR PURPOSE.
%
\chapter{Matem\'{a}tica no \LaTeX, amsmath} \label{sch:math}
Neste capítulo abordaremos o modo matemático do LaTeX, com uma \^{e}nfase nos pacotes amsmath\index{pacote!amsmath@\pkgname{amsmath}}, amsfonts, amssymb e amsthm.

\section{Modo matemático}
Para que expressões matemáticas seja processadas corretamentes, deve-se mudar do modo texto para o modo matemático, o que pode ser feito de várias maneiras.

A apresentação de expressões matemáticas pode ocorrer de duas maneiras: \flang{inline}\index{modo matematico@modo matem\'{a}tico!inline@\flang{inline}}, quando aparecem na mesma linha do texto, e \flang{displayed}\index{modo matematico@modo matem\'{a}tico!displayed@\flang{displayed}}, quando aparecem em uma linha própria e centralizada (podendo ou n\~{a}o ser numerada\footnote{Deve-se numerar apenas equa\c{c}\~{o}es as quais ser\~{a}o feita refer\^{e}ncias posteriormente.}).

A seguir, informaremos como proceder para produzir expressões matemáticas \flang{inline} ou \flang{displayed}. Ao final, apresentaremos algumas dicas sobre o uso de express\~{o}es \flang{inline} e \flang{displayed}.

\subsection{\flang{Inline}}
Expressões matemáticas \flang{inline}\index{modo matematico@modo matem\'{a}tico!inline@\flang{inline}} devem ser iniciadas por \lstinline!$! e fechadas por \lstinline!$! ou iniciadas por \lstinline!\)! e fechadas por \lstinline!\)!. \\
\example{codes/math_inline@latex_with_vim.tex}

\subsection{\flang{Displayed}}
Expressões matemáticas \flang{displayed}\index{modo matematico@modo matem\'{a}tico!displayed@\flang{displayed}} devem ser iniciadas por \lstinline!$$! e fechadas por \lstinline!$$! ou iniciadas por \lstinline!\[! e fechadas por \lstinline!\]!. \\
\example{codes/math_display@latex_with_vim.tex}

Alguns ambientes, como \envname{equation}, \envname{eqnarray} e \envname{align}, também produzem expressões matemáticas \flang{displayed}.

\subsection{Uso de \flang{inline} e \flang{displayed}}
Um \'{o}timo resumo sobre quando usar express\~{o}es \flang{inline} e \flang{displayed} encontra-se em \url{http://www.math.uiuc.edu/~hildebr/tex/displays.html}\nocite{Hildebrand:TeX_Resoures} e a seguir apresentaremos tradu\c{c}\~{a}o de alguns trechos. Para maiores detalhes recomenda-se uma leitura na obra ``Mathematics Into Type''\nocite{Swanson:1999:Mathematics}.

Express\~{o}es \flang{inline} s\~{a}o ``feias'' quando apresentam fra\c{c}\~{o}es, somat\'{o}rios, integrais, \ldots e algumas vezes precisam de um cuidado especial para respeitarem as margens. Entretanto, deve-se preferir utilizar express\~{o}es \flang{displayed} apenas nas seguintes ocasi\~{o}es:
\begin{itemize}
    \item a express\~{a}o \'{e} longa (ocupa mais da metade de uma linha);
    \item a express\~{a}o requer bastante espa\c{c}o vertical, i.e., possue v\'{a}rias fra\c{c}\~{o}es, somat\'{o}rios, integrais, \ldots;
    \item a equa\c{c}\~{a}o ser\'{a} numerada;
    \item a express\~{a}o que voc\^{e} deseja destacar/enfatizar.
\end{itemize}

\section{Primeiros comandos no modo matemático}
A seguir enunciaremos como proceder para produzir as primeiras equações, mas antes é importante saber que o modo matemático ignora qualquer espaço (para inserir um espa\c{c}o em branco no modo matem\'{a}tico veja a se\c{c}\~{a}o \ref{sss:math:textos_e_espacamentos}).

\subsection{Operações aritméticas básicas}
As operações aritméticas básicas\index{modo matematico@modo matem\'{a}tico!operacoes aritmeticas basicas@opera\c{c}\~{o}es aritm\'{e}ticas b\'{a}sicas} são escritas normalmente, exceto pela multiplicação que utiliza-se dos comandos \lstinline!\times! ou \lstinline!\cdot!\footnote{O uso do comando mais adequado depende muito do campo de estudo.} e das frações representada pelo comando \lstinline!\frac!\footnote{Deve-se ponderar o uso deste comando por quest\~{a}o de legibilidade.}. \\
\example{codes/math_powers_indices@latex_with_vim.tex}

\subsection{Índices e expoentes}
\'{I}ndices\index{modo matematico@modo matem\'{a}tico!indice@\'{i}ndice} e expoentes\index{modo matematico@modo matem\'{a}tico!expoente} são indicados pelos respectivos comandos: \flang{underscore}, \lcode{_}, e \flang{caret}, \lcode{^}. Por padrão apenas o primeiro símbolo depois do comando é alterado, quando for necessário mais de um símbolo deve-se utilizar chaves.

O símbolo \flang{prime}, muito utilizado para derivadas, já vem posicionado corretamente.\footnote{Algumas vezes deve-se preferir utilizar o comando \lcode{prime} em conjunto com \flang{underscore} e/ou \flang{caret}.} \\
\example{codes/math_powers_indices@latex_with_vim.tex}

\subsection{Acentos}
Os acentos\index{modo matematico@modo matem\'{a}tico!acento} disponíveis no modo matemático são apresentados na Tabela~\ref{tab:math_accents}.
\begin{table}[h!tb]
    \centering
    \caption{Acentos disponíveis no modo matemático.}
    \label{tab:math_accents}
    % Filename: math_accents@latex_with_vim.tex
% This code is part of LaTeX with Vim.
% 
% Description: LaTeX with Vim is free book about Vim, LaTeX and Git.
% 
% Created: 30.03.12 12:14:55 AM
% Last Change: 30.03.12 12:15:01 AM
% 
% Author: Raniere Gaia Costa da Silva, r.gaia.cs@gmail.com
% Organization:  
% 
% Copyright (c) 2010, 2011, 2012, Raniere Gaia Costa da Silva. All rights 
% reserved.
% 
% This file is license under the terms of a Creative Commons Attribution 
% 3.0 Unported License, or (at your option) any later version. More details
% at <http://creativecommons.org/licenses/by/3.0/>.
\begin{tabular}{cc|cc|cc}
    \hline
    Comando & Resultado & Comando & Resultado & Comando & Resultado \\ \hline
    \textbackslash\textsf{acute\{a\}} & $\acute{a}$ & \textbackslash\textsf{bar\{a\}} & $\bar{a}$ & \textbackslash\textsf{breve\{a\}} & $\breve{a}$ \\
    \textbackslash\textsf{check\{a\}} & $\check{a}$ & \textbackslash\textsf{dot\{a\}} & $\dot{a}$ & \textbackslash\textsf{ddot\{a\}} & $\ddot{a}$ \\
    \textbackslash\textsf{dddot\{a\}} & $\dddot{a}$ & \textbackslash\textsf{ddddot\{a\}} & $\ddddot{a}$ & \textbackslash\textsf{grave\{a\}} & $\grave{a}$ \\
    \textbackslash\textsf{hat\{a\}} & $\hat{a}$ & \textbackslash\textsf{widehat\{a\}} & $\widehat{a}$ & \textbackslash\textsf{mathring\{a\}} & $\mathring{a}$ \\
    \textbackslash\textsf{tilde\{a\}} & $\tilde{a}$ & \textbackslash\textsf{widetilde\{a\}} & $\widetilde{a}$ & \textbackslash\textsf{vec\{a\}} & $\vec{a}$ \hfill
\end{tabular}

\end{table}

\subsection{Delimitadores}
Parênteses\index{modo matematico@modo matem\'{a}tico!parenteses@par\^{e}nteses|see{delimitadores}}, colchetes\index{modo matematico@modo matem\'{a}tico!colchetes|see{delimitadores}} e chaves\index{modo matematico@modo matem\'{a}tico!chaves|see{delimitadores}} são exemplos de delimitadores\index{modo matematico@modo matem\'{a}tico!delimitadores}. Uma lista completa dos delimitadores dispon\'{i}veis no LaTeX encontra-se na Tabela~\ref{tab:math_delimiter}.
\begin{table}[h!tb]
    \centering
    \caption{Delimitadores disponíveis no LaTeX.}
    \label{tab:math_delimiter}
    % Filename: math_delimiter@latex_with_vim.tex
% This code is part of LaTeX with Vim.
% 
% Description: LaTeX with Vim is free book about Vim, LaTeX and Git.
% 
% Created: 30.03.12 12:16:24 AM
% Last Change: 30.03.12 12:16:29 AM
% 
% Author: Raniere Gaia Costa da Silva, r.gaia.cs@gmail.com
% Organization:  
% 
% Copyright (c) 2010, 2011, 2012, Raniere Gaia Costa da Silva. All rights 
% reserved.
% 
% This file is license under the terms of a Creative Commons Attribution 
% 3.0 Unported License, or (at your option) any later version. More details
% at <http://creativecommons.org/licenses/by/3.0/>.
\begin{tabular}{>{\centering}p{0.13\linewidth}<{\centering}>{\centering}p{0.07\linewidth}<{\centering}|>{\centering}p{0.13\linewidth}<{\centering}>{\centering}p{0.07\linewidth}<{\centering}|>{\centering}p{0.13\linewidth}<{\centering}>{\centering}p{0.07\linewidth}<{\centering}|>{\centering}p{0.13\linewidth}<{\centering}>{\centering}p{0.07\linewidth}<{\centering}}
    \hline
    Com. & Res. & Com. & Res. & Com. & Res. & Com. & Res. \tabularnewline \hline
    \lstinline!(! & $($ & \lstinline!)! & $)$ & \lstinline![! & $[$ & \lstinline!]! & $]$ \tabularnewline
    \lstinline!\{! & $\{$ & \lstinline!\}! & $\}$ & \lstinline!\backslash! & $\backslash$ & \lstinline!/! & $/$ \tabularnewline
    \lstinline!\langle! & $\langle$ & \lstinline!\rangle! & $\rangle$ & \lstinline!|! & $|$ & \lstinline!\|! & $\|$ \tabularnewline
    \lstinline!\lfloor! & $\lfloor$ & \lstinline!\rfloor! & $\rfloor$ & \lstinline!\lceil! & $\lceil$ & \lstinline!\rceil! & $\rceil$ \tabularnewline
    \lstinline!\ulcorner! & $\ulcorner$ & \lstinline!\urcorner! & $\urcorner$ &\lstinline!\llcorner! & $\llcorner$ & \lstinline!\lrcorner! & $\lrcorner$ \tabularnewline \hline
\end{tabular}
\begin{flushleft}
    \textbf{Nota:} Enquanto que \lstinline!|! \'{e} um limitador \lstinline!\mid! \'{e} um operador l\'{o}gico.
\end{flushleft}

\end{table}

Para expressões matemáticas no modo \flang{displayed} ou longas é aconselável utilizar os comandos \lstinline!\left! e \lstinline!\right! anteriormente ao limitador para ajustá-lo verticalmente. \\
\example{codes/math_delimiters_sizes@latex_with_vim.tex}

\subsection{Textos e espa\c{c}amentos} \label{sss:math:textos_e_espacamentos}
Existem tr\^{e}s ocasi\~{o}es em que \'{e} preciso inserir um texto\index{modo matematico@modo matem\'{a}tico!texto} dentro de uma express\~{a}o matem\'{a}tica:
\begin{itemize}
    \item um operador matem\'{a}tico \'{e} representado pelas primeiras letras de seu nome, e.g., $\max$, $\min$, $\lim$, \ldots;
    \item uma vari\'{a}vel \'{e} representada por mais de uma letra;
    \item incluir uma explica\c{c}\~{a}o/justificativa.
\end{itemize}

O LaTeX j\'{a} possue v\'{a}rios operadores matem\'{a}ticos definidos (s\~{a}o apresentados mais a frente) e quando o operador desejado n\~{a}o estiver definido deve-se utilizar o comando \lstinline!\operatorname!\index{modo matematico@modo matem\'{a}tico!novos operadores} ou \lstinline!\DeclareMathOperator!, este \'{u}ltimo quando o operador for ser utilizado v\'{a}rias vezes no documento. 

Em rela\c{c}\~{a}o ao nome de vari\'{a}veis\index{modo matematico@modo matem\'{a}tico!nomes longos para variaveis@nomes longos para vari\'{a}veis}, deve-se evitar ao m\'{a}ximo nome\'{a}-las com mais de uma letra (utilizar o alfabeto grego para isso). Quando n\~{a}o for poss\'{i}vel evitar, deve-se utilizar o comando \lstinline!\mathrm! para evitar confus\~{o}es. \\
\example{codes/math_var_names@latex_with_vim.tex}

J\'{a} para a inclusão de textos explicativos deve-se utilizar o comando \lstinline!\text!\index{comando!text@\lstinline+\text+} e \lstinline!\intertext!, este \'{u}ltimo reservado apenas para express\~{o}es \flang{displayed}. \\
\example{codes/math_text@latex_with_vim.tex}

Quanto ao espa\c{c}amento\index{modo matematico@modo matem\'{a}tico!espacamento@espa\c{c}amento}, normalmente n\~{a}o \'{e} preciso se preocupar com este pois o LaTeX inclui o espa\c{c}amento adequado. Em raras ocasi\~{o}es deve-se incluir algum espa\c{c}o apresentado na Tabela~\ref{tab:math_spacing}.
\begin{table}[!htb]
    \centering
    \caption{Espa\c{c}amento no modo matem\'{a}tico.}
    % Filename: algorithmic@latex_with_vim.tex
% This code is part of LaTeX with Vim.
% 
% Description: LaTeX with Vim is free book about Vim, LaTeX and Git.
% 
% Created: 30.03.12 12:11:31 AM
% Last Change: 30.03.12 12:11:38 AM
% 
% Author: Raniere Gaia Costa da Silva, r.gaia.cs@gmail.com
% Organization:  
% 
% Copyright (c) 2010, 2011, 2012, Raniere Gaia Costa da Silva. All rights 
% reserved.
% 
% This file is license under the terms of a Creative Commons Attribution 
% 3.0 Unported License, or (at your option) any later version. More details
% at <http://creativecommons.org/licenses/by/3.0/>.
\begin{tabular}{>{\centering}p{0.1\linewidth}<{\centering}>{\centering}p{0.15\linewidth}<{\centering}>{\centering}p{0.15\linewidth}<{\centering}|>{\centering}p{0.1\linewidth}<{\centering}>{\centering}p{0.15\linewidth}<{\centering}>{\centering}p{0.15\linewidth}<{\centering}}
    \hline
    Abrev. & Comando & Exemplo & Abrev. & Comando & Exemplo \tabularnewline \hline
    & sem espa\c{c}o & $\Rightarrow \Leftarrow$ & \lstinline!\,! & \lstinline!\thinspace! & $\Rightarrow \, \Leftarrow$ \tabularnewline
    \lstinline!\:! & \lstinline!\medspace! & $\Rightarrow \; \Leftarrow$ & \lstinline!\;! & \lstinline!\thickspace! & $\Rightarrow \; \Leftarrow$ \tabularnewline
    & \lstinline!\quad! & $\Rightarrow \quad \Leftarrow$ & & \lstinline!\qquad! & $\Rightarrow \qquad \Leftarrow$ \tabularnewline \hline
\end{tabular}

    \label{tab:math_spacing}
\end{table}

\subsection{Matrizes}
Para a construção de matrizes\index{modo matematico@modo matem\'{a}tico!matrizes} (e vetores\index{modo matematico@modo matem\'{a}tico!vetores|see{matrizes}}) utiliza-se o ambiente \envname{matrix} onde as colunas são separadas por \lstinline!&! e as linhas por \lstinline!\\!. \\
\example{codes/math_matrix@latex_with_vim.tex}

Destaca-se que o ambiente \envname{matrix} só pode ser utilizado dentro do ambiente matemático e que na última linha não utiliza-se o comando \lstinline!\\!\index{comando! @\lstinline+\\+}.

Pode-se utilizar limitadores envolvendo o ambiente \envname{matrix} ou utilizar uma variante: \envname{pmatrix}, \envname{bmatrix}, \envname{Bmatrix}, \envname{vmatrix} ou \envname{Vmatrix} que corresponde, respectivamente, aos delimitadores $( )$, $[ ]$, $\{ \}$, $| |$ e $\| \|$. \\

\section{Comandos avan\c{c}ados no modo matem\'{a}tico}
\subsection{Equações, numera\c{c}\~{a}o e referenciação} \label{sse:latex:equation}
Para o uso de expressões matemáticas a serem referenciadas\index{modo matematico@modo matem\'{a}tico!numeracao@numera\c{c}\~{a}o} posteriormente, recomenda-se o ambiente \envname{equation}\index{ambiente!equation@\envname{equation}} em conjunto com o comando \lstinline!\label!\index{comando!label@\lstinline+\label+}. \\
\example{codes/math_equation@latex_with_vim.tex}

No exemplo acima, \lcode{E:TeoPit} correspondente ao parâmetro do comando \lstinline!\label!, como apresentado na Seção~\ref{sse:cross_reference}. A referência a equação ocorre pelo comando \lstinline!\eqref!. \\
\example{codes/math_equation_eqref@latex_with_vim.tex}

\subsection{\flang{Tags}}
O comando \lstinline!\tag!\index{comando!tag@\lstinline+\tag+} do LaTeX nomeia uma equação\index{modo matematico@modo matem\'{a}tico!tag@\flang{tag}} e a referência passa a ser feito por este. \\
\example{codes/math_equation_tag@latex_with_vim.tex}

Vale destacar que podemos utilizar o comando \lstinline!\label! como parâmetro do comando \lstinline!\tag!.

\subsection{Teorema}
O comando \lstinline{\newtheorem}\index{modo matematico@modo matem\'{a}tico!teorema} deve ser inserido no \textit{preâmbulo} e é responsável por criar um ambiente numerado para informações. Sua síntaxe é
\begin{code}
\newtheorem{nome}{texto}
\end{code}
onde \lcode{nome} é o nome do ambiente a ser criado e \lcode{texto} é a sequência de caracteres que precede a numeração. Caso deseje-se não numerar deve-se utilizar a síntaxe
\begin{code}
\newtheorem*{nome}{texto}
\end{code}

Para fazer uso do novo ambiente deve-se utilizar a síntaxe padrão para um ambiente
\begin{code}
\begin{nome}
    ...
\end{nome}
\end{code}
ou ainda
\begin{code}
\begin{nome}[XXX]
    ...
\end{nome}
\end{code}
onde \lcode{XXX} é uma sequ\^{e}ncia de caracteres que aparece entre parênteses logo após a numeração.

\subsection{Demonstra\c{c}\~{a}o}
O ambiente \envname{proof} é destinada a demonstrações\index{modo matematico@modo matem\'{a}tico!demonstracao@demonstra\c{c}\~{a}o} e caracterizado por terminar com o comando \lstinline!\qed!. \\
\example{codes/math_proof@latex_with_vim.tex}

O ambiente \envname{proof}, como podemos observar no exemplo abaixo, não trabalha adequadamente quando é finalizado com uma expressão matemática \flang{displayed} e para corrigir isso devemos informar onde onde ser\'{a} inserido o s\'{i}mbolo \flang{qed}. \\
\example{codes/math_proof_set_qed@latex_with_vim.tex}

\subsection{Alinhamento}
O ambiente \envname{equation} foi projetado para trabalhar apenas com equações de uma única linha, nesta seção vamos apresentar algumas formas de trabalhar com equações com várias linhas\index{modo matematico@modo matem\'{a}tico!multiplas equacoes@m\'{u}ltiplas equa\c{c}\~{o}es}.

Para multiplas equações alinhadas utilizamos o ambiente \envname{align}\index{ambiente!align@\envname{align}}, sendo cada linha separada pelo comando \lstinline!\\!\index{comando! @\lstinline+\\+} e o alinhamento por \lstinline!&!\index{comando! @\lstinline+&+}. \\
\example{codes/math_align@latex_with_vim.tex}

Quando o alinhamento ocorrer adjacente a um sinal de \lstinline!=!, \lstinline!+!, \dots devemos utilizar o comando \lstinline!&! antes do sinal.

O ambiente \envname{align} numera todas as equações. Caso não queira numerar uma ou mais equações deve-se utilizar o comando \lstinline!\notag! em cada linha correspondente.

O comando \lstinline!\label! deve estar presente em cada linha.

Quando desejar adicionar a alguma linha alguma anotação utiliza-se o comando \lstinline!&&! entre a equação e a anotação. \\
\example{codes/math_align_annotation@latex_with_vim.tex}

\subsection{Fórmulas longas}
Para fórmulas muito longas\index{modo matematico@modo matem\'{a}tico!formulas longas@f\'{o}rmulas longas} que extrapolam a largura da caixa de texto deve-se utilizar o ambiente \envname{multline}, para uma \'{u}nica equa\c{c}\~{a}o, ou \envname{split}, este \'{u}ltimo deve ser utilizado dentro de um outro ambiente matem\'{a}tico.

\subsection{Ocultando termos}
Ao trabalhar com fórmulas muito longas tenta-se diminuir o tamanho utilizando sequências e muitas vezes é aconcelhável indicar o número de termos. Para isso podemos utilizar os comandos \lstinline!\overbrace! ou \lstinline!\underbrace!. \\
\example{codes/math_underbrace@latex_with_vim.tex}

\subsection{Funções definidas por partes}
É relativamente comum definirmos uma equações por partes e o ambiente adequado para representar esta construção é o \envname{cases}\index{modo matematico@modo matem\'{a}tico!funcoes definidas por partes@fun\c{c}\~{o}es definidas por partes}. \\
\example{codes/math_cases@latex_with_vim.tex}

O ambiente \envname{cases}\index{modo matematico@modo matem\'{a}tico!sistemas de equacoes@sistemas de equa\c{c}\~{o}es} tamb\'{e}m pode ser utilizado para sistemas de equa\c{c}\~{o}es.

\subsection{Fonte e S\'{i}mbolos}
No modo matemático, o LaTeX classifica os caracteres em alfabeto matemático e símbolos matemáticos. Baseado nessa classificação escolhe uma fonte a ser usada.

Para alterar a fonte de caracteres do alfabeto matemático utiliza-se o comando \lstinline!\mathXX! sendo que \lcode{XX} corresponde ao código da fonte a ser utilizada. A Tabela~\ref{tab:op_amfonte} apresenta alguns das opções disponíveis.
\begin{table}[h!tb]
    \centering
    \caption{Opções disponíveis para \lcode{XX} da fonte para o alfabeto matemático.}
    \label{tab:op_amfonte}
    % Filename: math_op_amfonte@latex_with_vim.tex
% This code is part of LaTeX with Vim.
% 
% Description: LaTeX with Vim is free book about Vim, LaTeX and Git.
% 
% Created: 30.03.12 12:11:31 AM
% Last Change: 30.03.12 12:11:38 AM
% 
% Author: Raniere Gaia Costa da Silva, r.gaia.cs@gmail.com
% Organization:  
% 
% Copyright (c) 2010, 2011, 2012, Raniere Gaia Costa da Silva. All rights 
% reserved.
% 
% This file is license under the terms of a Creative Commons Attribution 
% 3.0 Unported License, or (at your option) any later version. More details
% at <http://creativecommons.org/licenses/by/3.0/>.
\begin{tabular}{lp{0.8\textwidth}}
    \hline
    Código & Descrição \\ \hline
    \lcode{it} & Texto em itálico. \\
    \lcode{bf} & Texto em negrito. \\
    \lcode{rm} & Texto em romano. \\
    \lcode{sf} & Texto em sans serif. \\
    \lcode{tt} & Texto na tipografia de uma máquina de escrever.
\end{tabular}

\end{table}

A seguir é ilustrado as opções apresentadas na Tabela \ref{tab:op_amfonte}. \\
\example{codes/math_op_amfonte@latex_with_vim.tex}

Para símbolos matemáticos apenas é possível apresentá-los em negrito e, para isso, utiliza-se o comando \lstinline!\boldsymbol!. \\
\example{codes/math_boldsymbol@latex_with_vim.tex}

No LaTeX também existe quatro alfabetos que são interpretados como símbolos. Um deles é o alfabeto grego, apresentado no capítulo anterior e os outros três são acessados com o comando \lstinline!\mathXX!, sendo que \lcode{XX} corresponde ao código da fonte a ser utilizada. A Tabela~\ref{tab:op_asfonte} apresenta as opções disponíveis.
\begin{table}[h!tb]
    \centering
    \caption{Opções disponíveis para \lcode{XX} da fonte para o alfabeto matemático interpretado como símbolo.}
    \label{tab:op_asfonte}
    % Filename: math_op_asfonte@latex_with_vim.tex
% This code is part of 'LaTeX with Vim'.
% 
% Description: This file correspond to a example of LaTeX producing newlines.
% 
% Created: 27.06.12 08:29:06 AM
% Last Change: 27.06.12 08:29:06 AM
% 
% Author: Raniere Gaia Costa da Silva, r.gaia.cs@gmail.com
% 
% Copyright (c) 2012, Raniere Gaia Costa da Silva. All rights 
% reserved.
% 
% This work is licensed under the Creative Commons Attribution-ShareAlike 3.0 Unported License. To view a copy of this license, visit http://creativecommons.org/licenses/by-sa/3.0/ or send a letter to Creative Commons, 444 Castro Street, Suite 900, Mountain View, California, 94041, USA.
%
\begin{tabular}{lp{0.8\textwidth}}
    \hline
    Código & Descrição \\ \hline
    \lcode{cal} & Texto em caligráfico, apenas para caixa alta. \\
    \lcode{frak} & Texto em Euler Fraktur. \\
    \lcode{bb} & Texto em blackboard bold, apenas para caixa alta.
\end{tabular}

\end{table}

A seguir é ilustrado as opções apresentadas na Tabela~\ref{tab:op_asfonte}. \\
\example{codes/math_op_asfonte@latex_with_vim.tex}

Destaca-se que a fonte blackboard bold é normalmente utilizada para representar os conjuntos dos números naturais ($\mathbb{N}$), inteiros ($\mathbb{Z}$), reais ($\mathbb{R}$) e complexos ($\mathbb{C}$).

\section{S\'{i}mbolos e operadores}
A seguir apresentaremos v\'{a}rios dos s\'{i}mbolos e operadores dispon\'{i}veis no LaTeX. Para uma lista completa recomenda-se ``The Comprehensive LaTeX Symbol List''\nocite{Pakin:2009:Symbol}. Ao final, abordamos os comandos para ra\'{i}z quadrada, binomial e congru\^{e}ncias.
\begin{table}[h!tb]
    \centering
    \caption{Setas}
    \label{tab:math_arrows}
    % Filename: math_arrows@latex_with_vim.tex
% This code is part of LaTeX with Vim.
% 
% Description: LaTeX with Vim is free book about Vim, LaTeX and Git.
% 
% Created: 30.03.12 12:15:12 AM
% Last Change: 30.03.12 12:15:22 AM
% 
% Author: Raniere Gaia Costa da Silva, r.gaia.cs@gmail.com
% Organization:  
% 
% Copyright (c) 2010, 2011, 2012, Raniere Gaia Costa da Silva. All rights 
% reserved.
% 
% This file is license under the terms of a Creative Commons Attribution 
% 3.0 Unported License, or (at your option) any later version. More details
% at <http://creativecommons.org/licenses/by/3.0/>.
\begin{tabular}{cc|cc|cc|cc}
    \hline
    Comando & Res. & Comando & Res. & Comando & Res. & Comando & Res. \\ \hline
    \textbackslash\textsf{leftarrow} & $\leftarrow$ & \textbackslash\textsf{rightarrow} & $\rightarrow$ & \textbackslash\textsf{longleftarrow} & $\longleftarrow$ & \textbackslash\textsf{longrightarrow} & $\longrightarrow$ \\
    \textbackslash\textsf{Leftarrow} & $\Leftarrow$ & \textbackslash\textsf{Rightarrow} & $\Rightarrow$ & \textbackslash\textsf{Longleftarrow} & $\Longleftarrow$ & \textbackslash\textsf{Longrightarrow} & $\Longrightarrow$ \\
    \textbackslash\textsf{nleftarrow} & $\nleftarrow$ & \textbackslash\textsf{nrightarrow} & $\nrightarrow$ & \textbackslash\textsf{nLeftarrow} & $\nLeftarrow$ & \textbackslash\textsf{nRightarrow} & $\nRightarrow$ \\
    \textbackslash\textsf{leftrightarrow} & $\leftrightarrow$ & \textbackslash\textsf{longleftrightarrow} & $\longleftrightarrow$ & \textbackslash\textsf{Leftrightarrow} & $\Leftrightarrow$ & \textbackslash\textsf{Longleftrightarrow} & $\Longleftrightarrow$ \\
    \textbackslash\textsf{nleftrightarrow} & $\nleftrightarrow$ & \textbackslash\textsf{nLeftrightarrow} & $\nLeftrightarrow$ & \textbackslash\textsf{dashleftarrow} & $\dashleftarrow$ & \textbackslash\textsf{dashrightarrow} & $\dashrightarrow$ \\
    \textbackslash\textsf{leftrightharpoons} & $\leftrightharpoons$ & \textbackslash\textsf{rightleftharpoons} & $\rightleftharpoons$ & \textbackslash\textsf{leftrightarrows} & $\leftrightarrows$ & \textbackslash\textsf{rightleftarrows} & $\rightleftarrows$ \\
    \textbackslash\textsf{mapsto} & $\mapsto$ & \textbackslash\textsf{longmapsto} & $\longmapsto$ & \textbackslash\textsf{iff} & $\iff$ & &  \\
    \textbackslash\textsf{uparrow} & $\uparrow$ & \textbackslash\textsf{downarrow} & $\downarrow$ & \textbackslash\textsf{Uparrow} & $\Uparrow$ & \textbackslash\textsf{Downarrow} & $\Downarrow$ \\
    \textbackslash\textsf{updownarrow} & $\updownarrow$ & \textbackslash\textsf{Updownarrow} & $\Updownarrow$ & \textbackslash\textsf{Lsh} & $\Lsh$ & \textbackslash\textsf{Rsh} & $\Rsh$ \\
    \textbackslash\textsf{curvearrowleft} & $\curvearrowleft$ & \textbackslash\textsf{curvearrowright} & $\curvearrowright$ & \textbackslash\textsf{circlearrowleft} & $\circlearrowleft$ & \textbackslash\textsf{circlearrowright} & $\circlearrowright$ \\ \hline
\end{tabular}

\end{table}
\begin{table}[!htbp]
    \caption{Relações binárias} \centering
    \label{tab:math_binary_relations}
    % Filename: math_binary_relations@latex_with_vim.tex
% This code is part of LaTeX with Vim.
% 
% Description: LaTeX with Vim is free book about Vim, LaTeX and Git.
% 
% Created: 30.03.12 12:15:58 AM
% Last Change: 30.03.12 12:16:09 AM
% 
% Author: Raniere Gaia Costa da Silva, r.gaia.cs@gmail.com
% Organization:  
% 
% Copyright (c) 2010, 2011, 2012, Raniere Gaia Costa da Silva. All rights 
% reserved.
% 
% This file is license under the terms of a Creative Commons Attribution 
% 3.0 Unported License, or (at your option) any later version. More details
% at <http://creativecommons.org/licenses/by/3.0/>.
\begin{tabular}{cc|cc|cc|cc}
    \hline
    Comando & Res. & Comando & Res. & Comando & Res. & Comando & Res. \\ \hline
    \textless & $<$ & \textbackslash\textsf{nless} & $\nless$ & \textgreater & $>$ & \textbackslash\textsf{ngtr} & $\ngtr$ \\ 
    \textbackslash\textsf{ll} & $\ll$ & \textbackslash\textsf{lll} & $\lll$ & \textbackslash\textsf{gg} & $\gg$ & \textbackslash\textsf{ggg} & $\ggg$ \\ 
    = & $=$ & \textbackslash\textsf{neq} & $\neq$ & : & $:$ &  \textbackslash\textsf{doteq} & $\doteq$  \\ 
    \textbackslash\textsf{sim} & $\sim$ & \textbackslash\textsf{nsim} & $\nsim$ & \textbackslash\textsf{cong} & $\cong$ &  \textbackslash\textsf{ncong} & $\ncong$  \\ 
    \textbackslash\textsf{simeq} & $\simeq$ & \textbackslash\textsf{approx} & $\approx$ &  \textbackslash\textsf{equiv} & $\equiv$ & & \\
    \textbackslash\textsf{leq} ou \textbackslash\textsf{le} & $\leq$ & \textbackslash\textsf{nleq} & $\nleq$ & \textbackslash\textsf{geq} ou \textbackslash\textsf{ge} & $\geq$ & \textbackslash\textsf{ngeq} & $\ngeq$ \\ 
    \textbackslash\textsf{leqslant} & $\leqslant$ &  \textbackslash\textsf{nleqslant} & $\nleqslant$  & \textbackslash\textsf{geqslant} & $\geqslant$ & \textbackslash\textsf{ngeqslant} & $\ngeqslant$ \\ 
    \textbackslash\textsf{eqslantless}  & $\eqslantless$  &  &  & \textbackslash\textsf{eqslantgtr} & $\eqslantgtr$ &  &  \\ 
    \textbackslash\textsf{leqq} & $\leqq$ & \textbackslash\textsf{nleqq} & $\nleqq$ & \textbackslash\textsf{geqq} & $\geqq$ & \textbackslash\textsf{ngeqq} & $\ngeqq$ \\ 
    \textbackslash\textsf{lesssim}  & $\lesssim$  & \textbackslash\textsf{lessapprox}  & $\lessapprox$ & \textbackslash\textsf{gtrsim} & $\gtrsim$ & \textbackslash\textsf{gtrapprox} & $\gtrapprox$ \\ 
    \textbackslash\textsf{prec} & $\prec$ & \textbackslash\textsf{nprec} & $\nprec$ & \textbackslash\textsf{succ} & $\succ$ & \textbackslash\textsf{nsucc} & $\nsucc$ \\ 
    \textbackslash\textsf{preceq} & $\preceq$ & \textbackslash\textsf{npreceq} & $\npreceq$  & \textbackslash\textsf{succeq} & $\succeq$ & \textbackslash\textsf{nsucceq} & $\nsucceq$ \\
    \textbackslash\textsf{in} & $\in$ & \textbackslash\textsf{notin} & $\notin$ & \textbackslash\textsf{owns} & $\owns$ &  &  \\ 
    \textbackslash\textsf{subset} & $\subset$ &  &  & \textbackslash\textsf{supset} & $\supset$ &  &  \\ 
    \textbackslash\textsf{subseteq} & $\subseteq$ & \textbackslash\textsf{nsubseteq}  & $\nsubseteq$ & \textbackslash\textsf{supseteq} & $\supseteq$ & \textbackslash\textsf{nsupseteq} & $\nsupseteq$ \\ 
    \textbackslash\textsf{subseteqq}  & $\subseteqq$ & \textbackslash\textsf{nsubseteqq}  & $\nsubseteqq$ & \textbackslash\textsf{supseteqq}  & $\supseteqq$ & \textbackslash\textsf{nsupseteqq} & $\nsupseteqq$ \\ 
    \textbackslash\textsf{sqsubset} & $\sqsubset$ & \textbackslash\textsf{sqsubseteq} & $\sqsubseteq$ & \textbackslash\textsf{sqsupset} & $\sqsupset$ & \textbackslash\textsf{sqsupseteq} & $\sqsupseteq$  \\ 
    \textbackslash\textsf{smile} & $\smile$ & \textbackslash\textsf{smallsmile} & $\smallsmile$ & \textbackslash\textsf{frown} & $\frown$ & \textbackslash\textsf{smallfrown} & $\smallfrown$ \\ 
    \textbackslash\textsf{perp} & $\perp$ &  &  & \textbackslash\textsf{models} & $\models$ &  &  \\ 
    \textbackslash\textsf{mid} & $\mid$ & \textbackslash\textsf{nmid} & $\nmid$ & \textbackslash\textsf{parallel} & $\parallel$ & \textbackslash\textsf{nparallel} & $\nparallel$ \\ 
    \textbackslash\textsf{shortmid} & $\shortmid$ & \textbackslash\textsf{nshortmid} & $\nshortmid$ & \textbackslash\textsf{shortparallel} & $\shortparallel$ & \textbackslash\textsf{nshortparallel} & $\nshortparallel$ \\ 
    \textbackslash\textsf{vdash} & $\vdash$ & \textbackslash\textsf{nvdash} & $\nvdash$ & \textbackslash\textsf{dashv} & $\dashv$ &  &  \\ 
    \textbackslash\textsf{vDash}  & $\vDash$ & \textbackslash\textsf{nvDash}  & $\nvDash$ & \textbackslash\textsf{Vdash}  & $\Vdash $ & \textbackslash\textsf{nVdash} & $\nVdash$ \\ 
    \textbackslash\textsf{propto} & $\propto$ & \textbackslash\textsf{asymp} & $\asymp$ & \textbackslash\textsf{bowtie} & $\bowtie$ & \textbackslash\textsf{Join} & $\Join$ \\ 
    \textbackslash\textsf{vartriangleleft}  & $\vartriangleleft$ & \textbackslash\textsf{ntriangleleft}  & $\ntriangleleft$ & \textbackslash\textsf{vartriangleright}  & $\vartriangleright$ & \textbackslash\textsf{ntriangleright} & $\ntriangleright$ \\ 
    \textbackslash\textsf{trianglelefteq}  & $\trianglelefteq$ & \textbackslash\textsf{ntrianglelefteq}  & $\ntrianglelefteq$ & \textbackslash\textsf{trianglerighteq}  & $\trianglerighteq$ & \textbackslash\textsf{ntrianglerighteq} & $\ntrianglerighteq$ \\ 
    \textbackslash\textsf{blacktriangleleft} & $\blacktriangleleft$ &  &  & \textbackslash\textsf{blacktriangleright} & $\blacktriangleright$ &  &  \\ 
    \textbackslash\textsf{between}  & $\between$ & \textbackslash\textsf{pitchfork} & $\pitchfork$ & \textbackslash\textsf{therefore} & $\therefore$ & \textbackslash\textsf{because} & $\because$ \\ \hline
\end{tabular}

\end{table}
\begin{table}[!htb]
    \centering
    \caption{Operadores binários}
    \label{tab:math_binary_operations}
    % Filename: math_binary_operations@latex_with_vim.tex
% This code is part of LaTeX with Vim.
% 
% Description: LaTeX with Vim is free book about Vim, LaTeX and Git.
% 
% Created: 30.03.12 12:15:37 AM
% Last Change: 30.03.12 12:15:42 AM
% 
% Author: Raniere Gaia Costa da Silva, r.gaia.cs@gmail.com
% Organization:  
% 
% Copyright (c) 2010, 2011, 2012, Raniere Gaia Costa da Silva. All rights 
% reserved.
% 
% This file is license under the terms of a Creative Commons Attribution 
% 3.0 Unported License, or (at your option) any later version. More details
% at <http://creativecommons.org/licenses/by/3.0/>.
\begin{tabular}{cc|cc|cc|cc}
    \hline
    Comando & Res. & Comando & Res. & Comando & Res. & Comando & Res. \\ \hline
    + & $+$ & - & $-$ & \textbackslash\textsf{pm} & $\pm$ & \textbackslash\textsf{mp} & $\mp$ \\
    \textbackslash\textsf{times} & $\times$ & \textbackslash\textsf{cdot} & $\cdot$ & \textbackslash\textsf{div} & $\div$ & \textbackslash\textsf{And} & $\And$ \\
    \textbackslash\textsf{setminus} & $\setminus$ & \textbackslash\textsf{smallsetminus} & $\smallsetminus$ & \textbackslash\textsf{dagger} & $\dagger$ & \textbackslash\textsf{ddagger} & $\ddagger$ \\
    \textbackslash\textsf{ast} & $\ast$ & \textbackslash\textsf{star} & $\star$ & \textbackslash\textsf{wedge} &    $\wedge$ &  \textbackslash\textsf{vee} & $\vee$ \\
    \textbackslash\textsf{cap} & $\cap$  & \textbackslash\textsf{cup} & $\cup$  & \textbackslash\textsf{sqcap} & $\sqcap$  & \textbackslash\textsf{sqcup} & $\sqcup$ \\
    \textbackslash\textsf{oplus} & $\oplus$ & \textbackslash\textsf{ominus} & $\ominus$ & \textbackslash\textsf{otimes} & $\otimes$ & \textbackslash\textsf{oslash} & $\oslash$ \\
    \textbackslash\textsf{odot} & $\odot$  & \textbackslash\textsf{bigcirc} & $\bigcirc$  & \textbackslash\textsf{circ} & $\circ$ & \textbackslash\textsf{bullet} & $\bullet$ \\
    \textbackslash\textsf{bigtriangleup} & $\bigtriangleup$ & \textbackslash\textsf{bigtriangledown} & $\bigtriangledown$  & \textbackslash\textsf{triangleleft} & $\triangleleft$ & \textbackslash\textsf{triangleright} & $\triangleright$  \\
    \textbackslash\textsf{diamond} & $\diamond$ & \textbackslash\textsf{wr} & $\wr$ & \textbackslash\textsf{amalg} & $\amalg$ \\ \hline
\end{tabular}

\end{table}
\begin{table}[!htb]
    \centering
    \caption{Operadores puros.}
    \label{tab:math_functions1}
    % Filename: math_functions1@latex_with_vim.tex
% This code is part of LaTeX with Vim.
% 
% Description: LaTeX with Vim is free book about Vim, LaTeX and Git.
% 
% Created: 30.03.12 12:16:46 AM
% Last Change: 30.03.12 12:16:51 AM
% 
% Author: Raniere Gaia Costa da Silva, r.gaia.cs@gmail.com
% Organization:  
% 
% Copyright (c) 2010, 2011, 2012, Raniere Gaia Costa da Silva. All rights 
% reserved.
% 
% This file is license under the terms of a Creative Commons Attribution 
% 3.0 Unported License, or (at your option) any later version. More details
% at <http://creativecommons.org/licenses/by/3.0/>.
\begin{tabular}{cc|cc|cc}
    \hline
    Com. & Res. & Com. & Res. & Com. & Res. \\ \hline
    \lstinline!\log! & $\log$ & \lstinline!\ln! & $\ln$ & \lstinline!\exp! & $\exp$ \\
    \lstinline!\arccos! & $\arccos$ & \lstinline!\arcsin! & $\arcsin$ & \lstinline!\arctan! & $\arctan$ \\
    \lstinline!\cos! & $\cos$ & \lstinline!\sin! & $\sin$ & \lstinline!\tan! & $\tan$ \\
    \lstinline!\csc! & $\csc$ & \lstinline!\sec! & $\sec$ & \lstinline!\cot! & $\cot$ \\
    \lstinline!\cosh! & $\cosh$ & \lstinline!\sinh! & $\sinh$ & \lstinline!\tanh! & $\tanh$ \\
    \lstinline!\lg! & $\lg$ & \lstinline!\arg! & $\arg$ & \lstinline!\hom! & $\hom$ \\
    \lstinline!\dim! & $\dim$ & \lstinline!\ker! & $\ker$ & \lstinline!\det! & $\det$ \\
    \lstinline!\gcd! & $\gcd$ & & & & \\ \hline
\end{tabular}

\end{table}
\begin{table}[!htb]
    \centering
    \caption{Operadores com intervalos.}
    \label{tab:math_functions2}
    % Filename: math_functions2@latex_with_vim.tex
% This code is part of LaTeX with Vim.
% 
% Description: LaTeX with Vim is free book about Vim, LaTeX and Git.
% 
% Created: 30.03.12 12:17:02 AM
% Last Change: 30.03.12 12:17:06 AM
% 
% Author: Raniere Gaia Costa da Silva, r.gaia.cs@gmail.com
% Organization:  
% 
% Copyright (c) 2010, 2011, 2012, Raniere Gaia Costa da Silva. All rights 
% reserved.
% 
% This file is license under the terms of a Creative Commons Attribution 
% 3.0 Unported License, or (at your option) any later version. More details
% at <http://creativecommons.org/licenses/by/3.0/>.
\begin{tabular}{cc|cc|cc|cc}
    \hline
    Comando & Resultado & Comando & Resultado & Comando & Resultado & Comando & Resultado \\ \hline
    \textbackslash\textsf{int} & $\int$ & \textbackslash\textsf{iint} & $\iint$ & \textbackslash\textsf{iiint} & $\iiint$ & \textbackslash\textsf{iiiint} & $\iiiint$ \\
    \textbackslash\textsf{idotsint} & $\idotsint$ & \textbackslash\textsf{oint} & $\oint$ & \textbackslash\textsf{prod} & $\prod$ & \textbackslash\textsf{coprod} & $\coprod$ \\ 
    \textbackslash\textsf{bigcap} & $\bigcap$ & \textbackslash\textsf{bigcup} & $\bigcup$ & \textbackslash\textsf{bigwedge} & $\bigwedge$ & \textbackslash\textsf{bigvee} & $\bigvee$ \\ 
    \textbackslash\textsf{bigsqcup} & $\bigsqcup$ & \textbackslash\textsf{biguplus} & $\biguplus$ & \textbackslash\textsf{bigotimes} & $\bigotimes$ & \textbackslash\textsf{bigoplus} & $\bigoplus$ \\ 
    \textbackslash\textsf{bigodot} & $\bigodot$ & \textbackslash\textsf{sum} & $\sum$ & & & & \\ \hline
\end{tabular}

\end{table}
\begin{table}[!htb]
    \centering
    \caption{Operadores similares ao limites.}
    \label{tab:math_functions3}
    % Filename: math_functions3@latex_with_vim.tex
% This code is part of LaTeX with Vim.
% 
% Description: LaTeX with Vim is free book about Vim, LaTeX and Git.
% 
% Created: 30.03.12 12:17:16 AM
% Last Change: 30.03.12 12:17:21 AM
% 
% Author: Raniere Gaia Costa da Silva, r.gaia.cs@gmail.com
% Organization:  
% 
% Copyright (c) 2010, 2011, 2012, Raniere Gaia Costa da Silva. All rights 
% reserved.
% 
% This file is license under the terms of a Creative Commons Attribution 
% 3.0 Unported License, or (at your option) any later version. More details
% at <http://creativecommons.org/licenses/by/3.0/>.
\begin{tabular}{cc|cc|cc}
    \hline
    Com. & Res. & Com. & Res. & Com. & Res.  \\ \hline
    \lstinline!\lim! & $\lim$ & \lstinline!\inf! & $\inf$ & \lstinline!\sup! & $\sup$ \\
     \lstinline!\max! & $\max$ & \lstinline!\injlim! & $\injlim$ & \lstinline!\liminf! & $\liminf$ \\
     \lstinline!\limsup! & $limsup$ & \lstinline!\min! & $\min$ & \lstinline!\varinjlim! & $\varinjlim$ \\
     \lstinline!\varliminf! & $\varliminf$ & \lstinline!\varlimsup! & $\varlimsup$ & \lstinline!\Pr! & $\Pr$ \\
    \lstinline!\projlim! & $\projlim$ & \lstinline!\varprojlim! & $\varprojlim$ & \\ \hline
\end{tabular}

\end{table}
\begin{table}[!htb]
    \centering
    \caption{Outros símbolos matemáticos}
    \label{tab:math_others}
    % Filename: math_others@latex_with_vim.tex
% This code is part of LaTeX with Vim.
% 
% Description: LaTeX with Vim is free book about Vim, LaTeX and Git.
% 
% Created: 30.03.12 12:18:26 AM
% Last Change: 30.03.12 12:18:30 AM
% 
% Author: Raniere Gaia Costa da Silva, r.gaia.cs@gmail.com
% Organization:  
% 
% Copyright (c) 2010, 2011, 2012, Raniere Gaia Costa da Silva. All rights 
% reserved.
% 
% This file is license under the terms of a Creative Commons Attribution 
% 3.0 Unported License, or (at your option) any later version. More details
% at <http://creativecommons.org/licenses/by/3.0/>.
\begin{tabular}{cc|cc|cc|cc}
    \hline
    Comando & Res. & Comando & Res. & Comando & Res. & Comando & Res. \\ \hline
    \textbackslash\textsf{Re} & $\Re$ & \textbackslash\textsf{Im} & $\Im$ & \textbackslash\textsf{nabla} & $\nabla$ & \textbackslash\textsf{partial} & $\partial$  \\
    \textbackslash\textsf{infty} & $\infty$ & \textbackslash\textsf{emptyset} & $\emptyset$ & \textbackslash\textsf{varnothing} & $\varnothing$ \\
    \textbackslash\textsf{forall} & $\forall$ & \textbackslash\textsf{exists} & $\exists$ & \textbackslash\textsf{nexists} & $\nexists$ \\
    \textbackslash\textsf{angle} & $\angle$ & \textbackslash\textsf{measuredangle} & $\measuredangle$ & \textbackslash\textsf{sphericalangle} & $\sphericalangle$ \\
    \textbackslash\textsf{top} & $\top$ & \textbackslash\textsf{bot} & $\bot$ & \textbackslash\textsf{diagup} & $\diagup$ & \textbackslash\textsf{diagdown} & $\diagdown$ \\
    \textbackslash\textsf{triangle} & $\triangle$ & \textbackslash\textsf{triangledown} & $\triangledown$ & \textbackslash\textsf{blacktriangle} & $\blacktriangle$ & \textbackslash\textsf{blacktriangledown} & $\blacktriangledown$ \\
    \textbackslash\textsf{Diamond} & $\Diamond$ & \textbackslash\textsf{lozenge} & $\lozenge$ & \textbackslash\textsf{blacklozenge} & $\blacklozenge$ & \textbackslash\textsf{bigstar} & $\bigstar$ \\
    \textbackslash\textsf{Box} & $\Box$ & \textbackslash\textsf{square} & $\square$ & \textbackslash\textsf{blacksquare} & $\blacksquare$ \\
    \textbackslash\textsf{clubsuit} & $\clubsuit$ & \textbackslash\textsf{diamondsuit} & $\diamondsuit$ & \textbackslash\textsf{heartsuit} & $\heartsuit$ & \textbackslash\textsf{spadesuit} & $\spadesuit$ \\ \hline
\end{tabular}

\end{table}
\begin{table}[h!tb]
    \centering
    \caption{Alfabeto Grego, letras minúsculas}
    \label{tab:math_greek}
    % Filename: math_greek@latex_with_vim.tex
% This code is part of LaTeX with Vim.
% 
% Description: LaTeX with Vim is free book about Vim, LaTeX and Git.
% 
% Created: 30.03.12 12:17:35 AM
% Last Change: 30.03.12 12:17:39 AM
% 
% Author: Raniere Gaia Costa da Silva, r.gaia.cs@gmail.com
% Organization:  
% 
% Copyright (c) 2010, 2011, 2012, Raniere Gaia Costa da Silva. All rights 
% reserved.
% 
% This file is license under the terms of a Creative Commons Attribution 
% 3.0 Unported License, or (at your option) any later version. More details
% at <http://creativecommons.org/licenses/by/3.0/>.
\begin{tabular}{cc|cc|cc|cc}
    \hline
    Com. & Res. & Com. & Res. & Com. & Res. & Com. & Res. \\ \hline
    \lstinline!\alpha! & $\alpha$ & \lstinline!\beta! & $\beta$ & \lstinline!\gamma! & $\gamma$ & \lstinline!\delta! & $\delta$ \\
    \lstinline!\epsilon! & $\epsilon$ & \lstinline!\zeta! & $\zeta$ & \lstinline!\eta! & $\eta$ & \lstinline!\theta! & $\theta$ \\
    \lstinline!\iota! & $\iota$ & \lstinline!\kappa! & $\kappa$ & \lstinline!\lambda! & $\lambda$ & \lstinline!\mu! & $\mu$ \\
    \lstinline!\nu! & $\nu$ & \lstinline!\xi! & $\xi$ & \lstinline!\pi! & $\pi$ & \lstinline!\rho! & $\rho$ \\
    \lstinline!\sigma! & $\sigma$ & \lstinline!\tau! & $\tau$ & \lstinline!\upsilon! & $\upsilon$ & \lstinline!\phi! & $\phi$ \\
    \lstinline!\chi! & $\chi$ & \lstinline!\psi! & $\psi$ & \lstinline!\omega! & $\omega$ & \lstinline!\digamma! & $\digamma$ \\
    \lstinline!\varepsilon! & $\varepsilon$ & \lstinline!\vartheta! & $\vartheta$ & \lstinline!\varkappa! & $\varkappa$ & \lstinline!\varpi! & $\varpi$ \\
    \lstinline!\varrho! & $\varrho$ & \lstinline!\varsigma! & $\varsigma$ & \lstinline!\varphi! & $\varphi$ & & \\ \hline
\end{tabular}

\end{table}
\begin{table}[!tb]
    \centering
    \caption{Alfabeto Grego, letras maiúsculo}
    \label{tab:math_greek_capital}
    % Filename: math_greek_capital@latex_with_vim.tex
% This code is part of LaTeX with Vim.
% 
% Description: LaTeX with Vim is free book about Vim, LaTeX and Git.
% 
% Created: 30.03.12 12:17:50 AM
% Last Change: 30.03.12 12:17:56 AM
% 
% Author: Raniere Gaia Costa da Silva, r.gaia.cs@gmail.com
% Organization:  
% 
% Copyright (c) 2010, 2011, 2012, Raniere Gaia Costa da Silva. All rights 
% reserved.
% 
% This file is license under the terms of a Creative Commons Attribution 
% 3.0 Unported License, or (at your option) any later version. More details
% at <http://creativecommons.org/licenses/by/3.0/>.
\begin{tabular}{cc|cc|cc|cc}
    \hline
    Comando & Resultado & Comando & Resultado & Comando & Resultado & Comando & Resultado \\ \hline
    \textbackslash\textsf{Gamma} & $\Gamma$ & \textbackslash\textsf{Delta} & $\Delta$ & \textbackslash\textsf{Theta} & $\Theta$ & \textbackslash\textsf{Lambda} & $\Lambda$ \\
    \textbackslash\textsf{Xi} & $\Xi$ & \textbackslash\textsf{Pi} & $\Pi$ & \textbackslash\textsf{Sigma} & $\Sigma$ & \textbackslash\textsf{Upsilon} & $\Upsilon$ \\
    \textbackslash\textsf{Phi} & $\Phi$ & \textbackslash\textsf{Psi} & $\Psi$ & \textbackslash\textsf{Omega} & $\Omega$ \\
    \textbackslash\textsf{varGamma} & $\varGamma$ & \textbackslash\textsf{varDelta} & $\varDelta$ & \textbackslash\textsf{varTheta} & $\varTheta$ & \textbackslash\textsf{varLambda} & $\varLambda$ \\
    \textbackslash\textsf{varXi} & $\varXi$ & \textbackslash\textsf{varPi} & $\varPi$ & \textbackslash\textsf{varSigma} & $\varSigma$ & \textbackslash\textsf{varUpsilon} & $\varUpsilon$ \\
    \textbackslash\textsf{varPhi} & $\varPhi$ & \textbackslash\textsf{varPsi} & $\varPsi$ &
    \textbackslash\textsf{varOmega} & $\varOmega$ & & \\ \hline
\end{tabular}

\end{table}

\subsection{Raiz quadrada}
Utiliza-se o comando \lstinline!\sqrt! para raiz quadrada\index{modo matematico@modo matem\'{a}tico!raiz quadrada}. \\
\example{codes/math_sqrt@latex_with_vim.tex}

\subsection{Binomial}
Utiliza-se o comando \lstinline!\binom! para os bin\^{o}mios\index{modo matematico@modo matem\'{a}tico!binomio@bin\^{o}mio}. \\
\example{codes/math_binom@latex_with_vim.tex}

\subsection{Congruências}
A forma mais comum para congruências\index{modo matematico@modo matem\'{a}tico!congruencia@congru\^{e}ncia} corresponde ao uso dos comandos \lstinline!\equiv! e \lstinline!\pmod!. \\
\example{codes/math_mod@latex_with_vim.tex}
