% Filename: p1_introduction@tikz_for_teachers.tex
% This code is part of LaTeX with Vim.
% 
% Description: TikZ for teachers is free book about TikZ and Sage.
% 
% Created: 30.03.12 08:24:41 PM
% Last Change: 30.03.12 08:25:01 PM
% 
% Author: Raniere Gaia Costa da Silva, r.gaia.cs@gmail.com
% Organization:  
% 
% Copyright (c) 2010, 2011, 2012, Raniere Gaia Costa da Silva. All rights 
% reserved.
% 
% This file is license under the terms of a Creative Commons Attribution 
% 3.0 Unported License, or (at your option) any later version. More details
% at <http://creativecommons.org/licenses/by/3.0/>.
\chapter{Introdu\c{c}\~{a}o}
O pacote \TikZ and \PGF foi escrito e \'{e} mantido por Till Tantau estando atualmente na vers\~{a}o 2.10. A melhor explica\c{c}\~{a}o deste pacote encontra-se em ``The \TikZ and \PGF Packages - Manual for version 2.10''\nocite{Tantau:2010:Tikz-and-PGF} e transcrito a baixo:
\begin{quote}
    O pacote \PGF, onde ``PGF'' sup\~{o}em-se significar ``formato gr\'{a}fico port\'{a}til'', \'{e} um pacote para cria\c{c}\~{a}o gr\'{a}fica de uma maneira ``\textit{inline}''. Ele define um n\'{u}mero de comandos \TeX \  que desenham gr\'{a}ficos. (\dots)

    Quando voc\^{e} utiliza \PGF voc\^{e} ``programa'' seu gr\'{a}fico, assim como voc\^{e} ``programa'' o seu documento quando usa o \TeX \ . Voc\^{e} ganha todas as vantagens da composi\c{c}\~{a}o tipogr\'{a}fica do \TeX \ para os seus gr\'{a}ficos: r\'{a}pida cria\c{c}\~{a}o de gr\'{a}ficos simples, posicionamento preciso, uso de macros, tipografia superior. Consequentemente voc\^{e} tamb\'{e}m ganha todas as desvantagens: curva de apredizado lenta, \textit{no WYSIWYG}\footnote{\textit{WISWYG}: \'{e} o acr\^{o}nimo da express\~{a}o em ingl\^{e}s ``\textit{What You See Is What You Get}'', cuja tradu\c{c}\~{a}o remete a algo como ``O que voc\^{e} v\^{e} \'{e} o que voc\^{e} obtem''. (Wikip\'{e}dia)}, pequenas mudan\c{c}as requerem um longo tempo para recompilar, e o c\'{o}digo n\~{a}o d\'{a} pistas de como o gr\'{a}fico ir\'{a} parecer. (Tradu\c{c}\~{a}o livre do autor.)
\end{quote}
