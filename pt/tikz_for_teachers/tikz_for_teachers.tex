% Filename: tikz_for_teachers.tex
% This code is part of LaTeX with Vim.
% 
% Description: TikZ for teachers is free book about TikZ and Sage.
% 
% Created: 30.03.12 08:26:57 PM
% Last Change: 30.03.12 08:28:02 PM
% 
% Author: Raniere Gaia Costa da Silva, r.gaia.cs@gmail.com
% Organization:  
% 
% Copyright (c) 2010, 2011, 2012, Raniere Gaia Costa da Silva. All rights 
% reserved.
% 
% This file is license under the terms of a Creative Commons Attribution 
% 3.0 Unported License, or (at your option) any later version. More details
% at <http://creativecommons.org/licenses/by/3.0/>.
\documentclass[12pt,a4paper,openright,twoside]{book}
\usepackage[brazil]{babel}
\usepackage[utf8]{inputenc}
\usepackage{pb-diagram}
\usepackage{graphicx, color}
\usepackage[top=3cm,left=2cm,right=2cm,bottom=3cm]{geometry}
\usepackage{subfig}
\usepackage{enumerate}
\usepackage{algorithmic}
\usepackage{algorithm}
\usepackage{listings}
\usepackage{listingsutf8}
\usepackage{indentfirst}
\usepackage{multirow}
\usepackage{amsthm}
\usepackage{amsmath}
\usepackage{amsfonts}
\usepackage{amssymb}
\usepackage{makeidx}  % For index.
\makeindex  % For index.
\usepackage{tikz}
\usetikzlibrary{patterns}
\usepackage{epsfig}
\usepackage{latexsym}
\usepackage{makeidx}
\usepackage{url}
\usepackage{hyperref}
\usepackage{breakurl}
\usepackage{multicol}
\usepackage{parcolumns}
\usepackage[official]{eurosym}
\lstset{
breaklines=true,
literate={é}{{\'e}}1 {á}{{\'a}}1 {ã}{{\~a}}1,
}

\renewcommand{\lstlistingname}{C\'{o}digo}
\lstset{
language=TeX,                   % the language of the code
basicstyle=\ttfamily\footnotesize,     % the size of the fonts that are used for the code
%numbers=left,                   % where to put the line-numbers
%numberstyle=\footnotesize,      % the size of the fonts that are used for the line-numbers
%stepnumber=5,                   % the step between two line-numbers. If it's 1, each line 
%numbersep=5pt,                  % how far the line-numbers are from the code
showspaces=false,               % show spaces adding particular underscores
showstringspaces=false,         % underline spaces within strings
showtabs=false,                 % show tabs within strings adding particular underscores
tabsize=2,                      % sets default tabsize to 2 spaces
%captionpos=t,                   % sets the caption-position to bottom
breaklines=true,                % sets automatic line breaking
breakatwhitespace=false,        % sets if automatic breaks should only happen at whitespace
%caption={\texttt{\lstname}},    % show the filename of files included with \lstinputlisting;
}

% New commands and enviromments
\newcommand{\TikZ}{Ti\emph{k}Z }
\newcommand{\PGF}{\textsc{PGF} }
\newcommand{\lcode}[1]{\texttt{#1}}  % Code in the same line.
\lstnewenvironment{code}{}{}  % Code in new line.
\newcommand{\fcode}[1]{\lstinputlisting[firstline=18]{#1}}  % Code in a file.
\newcommand{\example}[1]{  % For examples.
\begin{minipage}[htb]{0.55\textwidth}
    \fcode{#1}
\end{minipage}
\begin{minipage}[htb]{0.35\textwidth}
    \include{#1}
\end{minipage}
}

\begin{document}
% Informations
\title{Ti\emph{k}Z para professores \\ \vspace*{4pt} \small{Versão 1.0}}
\author{\url{http://code.google.com/p/latex-with-vim/}}
\maketitle

% Frontmatter
\frontmatter 

% Filename: licence@tikz_for_teachers.tex
% This code is part of LaTeX with Vim.
% 
% Description: TikZ for teachers is free book about TikZ and Sage.
% 
% Created: 30.03.12 08:15:11 PM
% Last Change: 30.03.12 08:33:22 PM
% 
% Author: Raniere Gaia Costa da Silva, r.gaia.cs@gmail.com
% Organization:  
% 
% Copyright (c) 2010, 2011, 2012, Raniere Gaia Costa da Silva. All rights 
% reserved.
% 
% This file is license under the terms of a Creative Commons Attribution 
% 3.0 Unported License, or (at your option) any later version. More details
% at <http://creativecommons.org/licenses/by/3.0/>.
\section*{Licen\c{c}a}

``Ti\emph{k}Z para professores'' de \url{https://github.com/r-gaia-cs/latex_with_vim/} foi licenciada com uma Licen\c{c}a Creative Commons - Atribui\c{c}\~{a}o - Uso N\~{a}o Comercial 3.0 N\~{a}o Adaptada (\url{http://creativecommons.org/licenses/by-nc/3.0/}). 

Com base na obra dispon\'{i}vel em \url{https://github.com/r-gaia-cs/latex_with_vim/}. 

Podem estar dispon\'{i}veis permiss\~{o}es adicionais ao \^{a}mbito desta licen\c{c}a em \url{https://github.com/r-gaia-cs/latex_with_vim/}.

\begin{center}
    \includegraphics[keepaspectratio=true]{../../figures/by_nc.png}
\end{center}

\section*{Licence}

``Ti\emph{k}Z para professores'' by \url{https://github.com/r-gaia-cs/latex_with_vim} is licensed under a Creative Commons Attribution 3.0 Unported License (\url{http://creativecommons.org/licenses/by/3.0/}).

Based on a work at \url{https://github.com/r-gaia-cs/latex_with_vim/}.

Permissions beyond the scope of this license may be available at \url{https://github.com/r-gaia-cs/latex_with_vim/}.

\begin{center}
    \includegraphics[keepaspectratio=true]{../../figures/by_nc.png}
\end{center}


% Preface
\input{preface@tikz_for_teachers} 

\tableofcontents 
\listoftables

% Mainmatter
\mainmatter 

\part{Ti\emph{k}Z e PGF}
% Introdu\c{c}\~{a}o
% Filename: p1_introduction@tikz_for_teachers.tex
% This code is part of LaTeX with Vim.
% 
% Description: TikZ for teachers is free book about TikZ and Sage.
% 
% Created: 30.03.12 08:24:41 PM
% Last Change: 30.03.12 08:25:01 PM
% 
% Author: Raniere Gaia Costa da Silva, r.gaia.cs@gmail.com
% Organization:  
% 
% Copyright (c) 2010, 2011, 2012, Raniere Gaia Costa da Silva. All rights 
% reserved.
% 
% This file is license under the terms of a Creative Commons Attribution 
% 3.0 Unported License, or (at your option) any later version. More details
% at <http://creativecommons.org/licenses/by/3.0/>.
\chapter{Introdu\c{c}\~{a}o}
O pacote \TikZ and \PGF foi escrito e \'{e} mantido por Till Tantau estando atualmente na vers\~{a}o 2.10. A melhor explica\c{c}\~{a}o deste pacote encontra-se em ``The \TikZ and \PGF Packages - Manual for version 2.10''\nocite{Tantau:2010:Tikz-and-PGF} e transcrito a baixo:
\begin{quote}
    O pacote \PGF, onde ``PGF'' sup\~{o}em-se significar ``formato gr\'{a}fico port\'{a}til'', \'{e} um pacote para cria\c{c}\~{a}o gr\'{a}fica de uma maneira ``\textit{inline}''. Ele define um n\'{u}mero de comandos \TeX \  que desenham gr\'{a}ficos. (\dots)

    Quando voc\^{e} utiliza \PGF voc\^{e} ``programa'' seu gr\'{a}fico, assim como voc\^{e} ``programa'' o seu documento quando usa o \TeX \ . Voc\^{e} ganha todas as vantagens da composi\c{c}\~{a}o tipogr\'{a}fica do \TeX \ para os seus gr\'{a}ficos: r\'{a}pida cria\c{c}\~{a}o de gr\'{a}ficos simples, posicionamento preciso, uso de macros, tipografia superior. Consequentemente voc\^{e} tamb\'{e}m ganha todas as desvantagens: curva de apredizado lenta, \textit{no WYSIWYG}\footnote{\textit{WISWYG}: \'{e} o acr\^{o}nimo da express\~{a}o em ingl\^{e}s ``\textit{What You See Is What You Get}'', cuja tradu\c{c}\~{a}o remete a algo como ``O que voc\^{e} v\^{e} \'{e} o que voc\^{e} obtem''. (Wikip\'{e}dia)}, pequenas mudan\c{c}as requerem um longo tempo para recompilar, e o c\'{o}digo n\~{a}o d\'{a} pistas de como o gr\'{a}fico ir\'{a} parecer. (Tradu\c{c}\~{a}o livre do autor.)
\end{quote}

% Instala\c{c}\~{a}o
\input{p1_instalation@tikz_for_teachers}
% Tutorial B\'{a}sico
% Filename: p1_basic_tutorial@tikz_for_teachers.tex
% This code is part of LaTeX with Vim.
% 
% Description: TikZ for teachers is free book about TikZ and Sage.
% 
% Created: 30.03.12 08:21:08 PM
% Last Change: 30.03.12 08:24:03 PM
% 
% Author: Raniere Gaia Costa da Silva, r.gaia.cs@gmail.com
% Organization:  
% 
% Copyright (c) 2010, 2011, 2012, Raniere Gaia Costa da Silva. All rights 
% reserved.
% 
% This file is license under the terms of a Creative Commons Attribution 
% 3.0 Unported License, or (at your option) any later version. More details
% at <http://creativecommons.org/licenses/by/3.0/>.
\chapter{Tutorial B\'{a}sico}
Neste cap\'{i}tulo apresentamos os commandos mais b\'{a}sicos do \TikZ.

\section{Ambiente \lcode{tikzpicture}}\index{ambiente|see{\textit{environment}}}
Ao utilizar \TikZ para desenhar uma figura voc\^{e} precisa informar ao \LaTeX \ que deseja-se iniciar uma figura. Para isso utiliza-se o ambiente \lcode{tikzpicture}\index{environment@\textit{environment}!tikzpicture@\lcode{tikzpicture}}. A seguir encontra-se um pequeno exemplo do ambiente \lcode{tikzpicture}. 

\example{codes/line01@tikz_for_teachers}

No exemplo acima podemos notar que, dentro do ambiente \lcode{tikzpicture}, os comandos devem terminar com um ponto e v\'{i}rgula.

Tamb\'{e}m no exemplo acima, observamos que o ambiente \lcode{tikzpicture} n\~{a}o \'{e} flutuante. Uma maneira de torn\'{a}-lo flutuante \'{e} envolvendo-o pelo ambiente \lcode{figure}\index{environment@\textit{environment}!figure@\lcode{figure}} como apresentado a seguir.

\fcode{codes/line02@tikz_for_teachers}

Uma outra caracter\'{i}stica do ambiente \lcode{tikzpicture} \'{e} que comandos recentes s\~{a}o sobrepostos aos comandos antigos. No exemplo a seguir observamos essa caracter\'{i}stica.

\example{codes/overwrite@tikz_for_teachers}

\section{Sistema de coordenadas}\index{sistema de coordenadas}
A constru\c{c}\~{a}o de qualquer figura usando o \TikZ requer que seja informado coordenadas de acordo com algum sistema. O \TikZ aceita o sistema de coordenadas cartesianas, que corresponde a forma \lcode{(x, y)}, onde \lcode{x} corresponde a coordenada horizontal e \lcode{y} a vertical, e o sistema de coordenadas polares, que corresponde a forma \lcode{(a, r)}, onde \lcode{a} a dire\c{c}\~{a}o em graus e \lcode{r} corresponde ao comprimento do raio.

\example{codes/coordinate_system@tikz_for_teachers}

Al\'{e}m de coordenadas absolutas, o \TikZ tamb\'{e}m aceita coordenadas relativas. Coordenadas relativas devem ser precedidas por \lcode{+}, que significa ``adicionar as seguintes coordenadas \`{a} coordenada absoluta previamente informada'', ou \lcode{++}, que significa ``adicionar as seguintes coordenadas \`{a} coordenada absoluta previamente informada e tornar esta a nova coordenada absoluta previamente informada''.

\example{codes/relative_coordinates@tikz_for_teachers}

O \TikZ aceita uma vasta variedade de unidades de medida para as coordendas, por exemplo: \lcode{pt}, \lcode{cm}, \lcode{mm} \ldots

\example{codes/measure_units@tikz_for_teachers}

Pelo exemplo acima verifica-se que caso nenhuma unidade seja especificada \'{e} utilizada \lcode{cm}.

Outra caracter\'{i}stica do \TikZ \'{e} que ele ajusta a figura criada para ocupar o espa\c{c}o m\'{i}nimo necess\'{a}rio. Essa caracter\'{i}stica \'{e} observada no exemplo a seguir que corresponde ao primeiro exemplo com um deslocamento de $5$ unidades horizontais e o resultado produzido \'{e} id\^{e}ntico ao do primeiro exemplo.

\example{codes/line03@tikz_for_teachers}

\section{Linhas}\index{linhas|see{\lcode{draw}}}
Nesta se\c{c}\~{a}o iremos tratar da constru\c{c}\~{a}o de linhas com o \TikZ. Pelos exemplos anteriores o leitor j\'{a} deve ter inferido que o comando \lcode{draw} \'{e} respons\'{a}vel pela constru\c{c}\~{a}o de linhas.

No primeiro exemplo, o comando \lcode{draw}\index{draw@\lcode{draw}} \'{e} seguido por um conjunto de op\c{c}\~{o}es envolvidas em colchetes, pelas coordenadas do ponto inicial, um operador (no caso \lcode{-{}-{}}) e pelas coordenadas do ponto final.

\'{E} poss\'{i}vel utilizar o mesmo comando \lcode{draw} com pontos intermedi\'{a}rios, a seguir apresentamos um exemplo desste uso.

\example{codes/line04@tikz_for_teachers}

Al\'{e}m da op\c{c}\~{a}o \lcode{color} que corresponde a cor da linha e do operador \lcode{-{}-{}} que corresponde a uma linha entre dois pontos existem muitos outros. A seguir apresentamos algumas op\c{c}\~{o}es e depois alguns operadores.

\section{Op\c{c}\~{o}es}
A seguir abordamos algumas das op\c{c}\~{o}es dispon\'{i}veis no \TikZ.

\subsection{Escala}\index{escala|see{\lcode{scale}}}
Uma das grandes vantagens do \TikZ \'{e} a capacidade de reescalar uma figura sem perder qualidade no processo.

A op\c{c}\~{a}o \lcode{scale}\index{scale@\lcode{scale}} \'{e} respons\'{a}vel por escalar a linha a ser desenhada e deve receber o fator de escala a ser utilizado.

\example{codes/scale@tikz_for_teachers}

\subsection{Rota\c{c}\~{a}o}\index{rotacao@rota\c{c}\~{a}o|see{\lcode{rotate}}}
A op\c{c}\~{a}o \lcode{rotate}\index{rotate@\lcode{rotate}} \'{e} respons\'{a}vel por rotacionar a linha a ser desenhada e deve receber a medida em grau a ser utilizada.

\example{codes/rotate@tikz_for_teachers}

Como podemos observar pelo exemplo acima, o ponto fixo da rota\c{c}\~{a}o corresponde ao primeiro ponto do comando.

\subsection{Cores}\index{cores|see{\lcode{color}}}
A op\c{c}\~{a}o \lcode{color}\index{color@\lcode{color}} \'{e} respons\'{a}vel pela cor da linha a ser desenhada e deve receber o nome de uma cor previamente definida. No \LaTeX \, o nome das cores previamente definidas encontram-se dispon\'{i}veis no pacote \lcode{color} e a cria\c{c}\~{a}o de novas cores pode ser feita utilizando o pacote \lcode{xcolor} (um resumo deste pacote \'{e} encontrado em \url{http://en.wikibooks.org/wiki/LaTeX/Colors}).

\example{codes/color01@tikz_for_teachers}

\subsection{Padr\~{a}o}\index{padrao@padr\~{a}o|see{\textit{pattern}}}
Para modificar o padr\~{a}o da linha utiliza-se as op\c{c}\~{o}es \lcode{dash pattern}\index{pattern@\textit{pattern}} e \lcode{dash phase}. A primeira delas corresponde ao padr\~{a}o a ser utilizado e a segunda a um deslocamento no pad\~{a}o.

\example{codes/line_pattern01@tikz_for_teachers}

Encontram-se predefinidos alguns estilos que fornecem uma maneira mais natural de informar o padr\~{a}o da linha, alguns deles s\~{a}o: \lcode{solid}, \lcode{dotted}\index{pontilhado|see{\lcode{dotted}}}\index{dotted@\lcode{dotted}}, \lcode{dashed}\index{tracejado|see{\lcode{dashed}}}\index{dashed@\lcode{dashed}}, \ldots

\example{codes/line_pattern02@tikz_for_teachers}

\subsection{Setas}\index{seta|see{\textit{arrow}}}
Para a constru\c{c}\~{a}o de setas\index{arrow@\textit{arrow}} pode-se utilizar uma dentre as seguintes op\c{c}\~{o}es: \lcode{-}\textgreater, \textless\lcode{-} e \textless\lcode{-}\textgreater.

\example{codes/arrow01@tikz_for_teachers}

Tamb\'{e}m \'{e} poss\'{i}vel duplicar o indicador da seta utilizando uma dentre as seguintes op\c{c}\~{o}es: \lcode{-}\textgreater\textgreater, \textless\textless\lcode{-} e \textless\textless\lcode{-}\textgreater\textgreater.

\example{codes/arrow02@tikz_for_teachers}

\subsection{Espessura}\index{espessura|see{\lcode{line width}}}
A op\c{c}\~{a}o \lcode{line width}\index{line width@\lcode{line width}} \'{e} respons\'{a}vel pela espessura da linha a ser desenhada e deve receber uma medida para a espessura da linha.

Encontram-se predefinidos alguns estilos que fornecem uma maneira mais ``natural'' de informar a espessura da linha, alguns deles s\~{a}o: \lcode{ultra thin}, \lcode{thin}, \lcode{thick} \lcode{ultra thick}, \ldots

\example{codes/line_width01@tikz_for_teachers}

\subsection{\lcode{line cap}}
A op\c{c}\~{a}o \lcode{line cap}\index{line cap@\lcode{line cap}} \'{e} respons\'{a}vel por como a linha termina. Existem apenas tr\^{e}s tipos dispon\'{i}veis: \lcode{round}, \lcode{rect} e \lcode{butt}.

\example{codes/line_cap01@tikz_for_teachers}

\subsection{\lcode{line join}}
A op\c{c}\~{a}o \lcode{line join}\index{line join@\lcode{line join}} \'{e} respons\'{a}vel por como duas linhas s\~{a}o unidas. Existem apenas tr\^{e}s tipos dispon\'{i}veis: \lcode{round}, \lcode{bevel} e \lcode{miter}.

\example{codes/line_join01@tikz_for_teachers}

\section{Operadores}
\subsection{Ret\^{a}ngulos}\index{retangulo@ret\^{a}ngulo|see{\lcode{rectangle}}}
Para a constru\c{c}\~{a}o de ret\^{a}ngulos pode-se utilizar o operador \lcode{retangle}\index{rectangle@\lcode{rectangle}} sendo que as coordenadas correspondem dois v\'{e}rtices n\~{a}o adjacentes do ret\^{a}ngulo.

\example{codes/rectangle01@tikz_for_teachers}

No exemplo acima observamos a ocorr\^{e}ncia de um ret\^{a}ngulo degenerado em uma linha.

\subsection{Malha retangular}
Algumas vezes deseja-se incluir na figura uma malha retangular. Para isso pode-se utilizar o operador \lcode{grid} sendo que, de maneira an\'{a}loga ao operador \lcode{rectangle}, as coordenads correspondem a dois v\'{e}rtices n\~{a}o adjacentes do ret\^{a}ngulo maior.

\example{codes/grid01@tikz_for_teachers}

Para o operador \lcode{grid} est\~{a}o dispon\'{i}veis as tr\^{e}s op\c{c}\~{o}es a seguir:
\begin{enumerate}
    \item \lcode{step}: especifica a dist\^{a}ncia horizontal e vertical dos elementos da malha ret\^{a}ngular;
    \item \lcode{xstep}: especifica a dist\^{a}ncia horizontal dos elementos da malha ret\^{a}ngular;
    \item \lcode{ystep}: especifica a dist\^{a}ncia vertical dos elementos da malha ret\^{a}ngular.
\end{enumerate}

\example{codes/grid02@tikz_for_teachers}

\subsection{Circunfer\^{e}ncias}\index{circunferencia@circunfer\^{e}ncia|see{\lcode{circle}}}
Para a constru\c{c}\~{a}o de circunfer\^{e}ncias pode-se utilizar o operador \lcode{circle}\index{circle@\lcode{circle}} sendo que o operador \'{e} seguido pela medida do raio.

\example{codes/circle@tikz_for_teachers}

\subsection{Elipse}\index{elipse|see{\lcode{ellipse}}}
Para a constru\c{c}\~{a}o de uma elipse pode-se utilizar o operador \lcode{ellipse}\index{ellipse@\lcode{ellipse}} sendo que o operador \'{e} seguido pela medida dos raios horizontais e verticais.

\example{codes/ellipse@tikz_for_teachers}

\subsection{Arcos}\index{arcos|see{\lcode{arc}}}
Para a constru\c{c}\~{a}o de parte de circunfer\^{e}ncia ou de elipse, i.e., um arco pode-se utilizar o operador \lcode{arc}\index{arc@\lcode{arc}} que sendo que o operador \'{e} seguido por uma tripla separada por dois pontos referentes ao grau inicial, grau final e o raio.

\example{codes/arc01@tikz_for_teachers}

Para o caso de elipses deve-se especificar o raio horizontal e vertical.

\example{codes/arc02@tikz_for_teachers}

\section{Texto}\index{texto|see{n\'{o}}}\index{no@n\'{o}|see{\lcode{node}}}
Na se\c{c}\~{a}o anterior apresentamos como construir linhas e algumas figuras geom\'{e}tricas como ret\^{a}ngulos e circunfer\^{e}ncias. Nesta se\c{c}\~{a}o iremos apresentar como adicionar um pequeno texto pr\'{o}ximo a uma linha.

No \TikZ o comando \lcode{node}\index{node@\lcode{node}} \'{e} respons\'{a}vel por inserir um pequeno texto em uma posi\c{c}\~{a}o espec\'{i}fica. A seguir encontra-se um exemplo bastante simples.

\example{codes/node01@tikz_for_teachers}

Al\'{e}m do uso apresentado no exemplo acima, o comando \lcode{node} tamb\'{e}m pode ser utilizado em conjunto com o comando \lcode{draw} como apresentado a seguir.

\example{codes/node02@tikz_for_teachers}

Assim como o comando \lcode{draw}, o comando \lcode{node} permite algumas op\c{c}\~{o}es que possibilitam aprimorar o exemplo acima. Tais op\c{c}\~{o}es ser\~{a}o descritas a seguir.

\subsection{Cores}\index{node@\lcode{node}!cores|see{\lcode{node text}}}
A cor do texto de um n\'{o} \'{e} definido pela op\c{c}\~{a}o \lcode{text}\index{node@\lcode{node}!text@\lcode{text}} que recebe o nome de uma cor.

\example{codes/node_color@tikz_for_teachers}

Pelo exemplo acima verificamos que a op\c{c}\~{a}o \lcode{text} pode ser utilizada tanto como op\c{c}\~{a}o do comando \lcode{node} como do comando \lcode{draw}.

\subsection{Ancoras}\index{ancoras|see{\lcode{anchor}}}
Muitas vezes n\~{a}o deseja-se colocar o n\'{o} nas coordenadas indicada mas pr\'{o}ximo dela. Nestes casos deve-se utilizar a op\c{c}\~{a}o \lcode{node@\lcode{node}!anchor@\lcode{anchor}} que recebe uma das seguintes orienta\c{c}\~{o}es:
\begin{enumerate}
    \item \lcode{north},
    \item \lcode{south},
    \item \lcode{east},
    \item \lcode{west}.
\end{enumerate}

\'{E} poss\'{i}vel combinar as orienta\c{c}\~{o}es tomando o cuidado da primeira orienta\c{c}\~{a}o sempre corresponder ao eixo vertical, e.g., \lcode{north east}.

\example{codes/node_anchor01@tikz_for_teachers}

Como o uso de \^{a}ncoras costuma ser pouco intuitivo existem algumas op\c{c}\~{o}es que s\~{a}o equivalente:
\begin{enumerate}
    \item \lcode{below} \'{e} equivalente a \lcode{anchor=north},
    \item \lcode{above} \'{e} equivalente a \lcode{anchor=south},
    \item \lcode{right} \'{e} equivalente a \lcode{anchor=east},
    \item \lcode{left} \'{e} equivalente a \lcode{anchor=west}.
\end{enumerate}

Tamb\'{e}m \'{e} poss\'{i}vel combinar as op\c{c}\~{o}es enumeradas acima seguindo o mesmo cuidado do uso de \^{a}ncoras, i.e., a primeira orienta\c{c}\~{a}o sempre corresponde ao eixo vertical. Al\'{e}m disso, essas op\c{c}\~{o}es permitem atribuir uma medida para o deslocamento em cada uma das dire\c{c}\~{o}es.

\example{codes/node_anchor02@tikz_for_teachers}

\subsection{Nomea\c{c}\~{a}o}\index{node@\lcode{node}!name}
Os n\'{o}s possuem uma caracter\'{i}stica muito \'{u}til que \'{e} a possibilidade de nome\'{a}-los. Para atribuir um nome a um n\'{o} utiliza-se par\^{e}nteses logo em seguida do comando \lcode{node}.

\example{codes/node_name01@tikz_for_teachers}

Ap\'{o}s nomear um n\'{o} podemos utilizar sua posi\c{c}\~{a}o a partir de seu nome.

\example{codes/node_name02@tikz_for_teachers}

No exemplo acima nota-se que a linha desenhada n\~{a}o inicia exatamente nas coordenadas correspondentes aos n\'{o}s mas na fronteira do n\'{o}, i.e., a linha inicia-se no contorno do n\'{o}.

\example{codes/node_name03@tikz_for_teachers}

% Tutorial Avan\c{c}ado
% Filename: p1_advanced_tutorial@tikz_for_teachers.tex
% This code is part of LaTeX with Vim.
% 
% Description: TikZ for teachers is free book about TikZ and Sage.
% 
% Created: 30.03.12 08:18:37 PM
% Last Change: 30.03.12 08:34:14 PM
% 
% Author: Raniere Gaia Costa da Silva, r.gaia.cs@gmail.com
% Organization:  
% 
% Copyright (c) 2010, 2011, 2012, Raniere Gaia Costa da Silva. All rights 
% reserved.
% 
% This file is license under the terms of a Creative Commons Attribution 
% 3.0 Unported License, or (at your option) any later version. More details
% at <http://creativecommons.org/licenses/by/3.0/>.
\chapter{Tutorial Avan\c{c}ado}
\section{Estilo}\index{estilo|see{\lcode{style}}}
Estilos s\~{a}o um conjunto de op\c{c}\~{o}es que s\~{a}o utilizadas para organizar como uma figura \'{e} desenhada. Para a defini\c{c}\~{a}o de um estilo dentro de um ambiente utiliza-se o comando \lcode{style}\index{style@\lcode{style}}.

\example{codes/style01@tikz_for_teachers}

O uso de estilos torna o c\'{o}digo mais flex\'{i}vel de modo que \'{e} poss\'{i}vel alter\'{a}-lo de uma maneira mais consistente.

Para a defini\c{c}\~{a}o de estilos globais, i.e., que existem em todos os ambientes deve-se utilizar o comando \lcode{tikzset}\index{tikzset@\lcode{tikzset}} no in\'{i}cio do documento e, como par\^{a}metro, o estilo desejado utilizando o comando \lcode{style}.

\section{Caminho}\index{caminho|see{\lcode{path}}}
No cap\'{i}tulo anterior foi apresentado como construir linhas utilizando o comando \lcode{draw}. Na verdade o comando \lcode{draw} \'{e} apenas um caso especial do comando \lcode{path}\index{path@\lcode{path}}. A seguir apresentamos algumas das op\c{c}\~{o}es que podem ser utilizadas com o comando \lcode{path}.

\subsection{Linha}
Para construir uma linha utiliza-se a op\c{c}\~{a}o \lcode{draw}\index{path@\lcode{path}!draw@\lcode{draw}}.

\example{codes/path_draw@tikz_for_teachers}

Pelo exemplo acima observa-se que o resultado \'{e} o mesmo do comando \lcode{draw}.

\subsection{Preenchimento}\index{preenchimento|see{\lcode{path fill}}}
At\'{e} o momento apenas contruimos linhas e algumas figuras geom\'{e}tricas. Como devemos proceder para preencher uma figura? Para preencher uma figura utiliza-se a op\c{c}\~{a}o \lcode{fill}\index{path@\lcode{path}!fill@\lcode{fill}}.

\example{codes/path_fill@tikz_for_teachers}

Pelo exemplo acima verifica-se que a op\c{c}\~{a}o \lcode{fill} apenas preenche a figura sem tratar o contorno. Isso ocorre pois o contorno \'{e} determinado pela op\c{c}\~{a}o \lcode{draw} vista anteriormente. No exemplo a seguir utilizamos as op\c{c}\~{o}es \lcode{fill} e \lcode{draw} em conjunto. 

\example{codes/path_filldraw@tikz_for_teachers}

Ao inv\'{e}s de utilizar o comando \lcode{path} com a op\c{c}\~{a}o \lcode{fill} \'{e} poss\'{i}vel utilizar o comando \lcode{fill}\index{fill@\lcode{fill}} e o comando \lcode{filldraw}\index{filldraw@\lcode{filldraw}} no lugar do comando \lcode{path} com as op\c{c}\~{o}es \lcode{fill} e \lcode{draw}.

De maneira geral, \'{e} permitido utilizar qualquer op\c{c}\~{a}o do comando \lcode{path} como uma op\c{c}\~{a}o de um comando correspondente a uma op\c{c}\~{a}o do comando \lcode{path}, portanto as seguintes constru\c{c}\~{o}es s\~{a}o v\'{a}lidas:
\begin{code}
    \fill[draw=red] (0,-1) rectangle (1,-3);
\end{code}
e
\begin{code}
    \draw[fill=blue] (2,-1) rectangle (3,-3);
\end{code}
e equivalentes a constru\c{c}\~{a}o utilizada no exemplo anterior.

\subsection{Padr\~{a}o}
No cap\'{i}tulo anterior foi apresentado alguns padr\~{o}es para linhas como pontilhado e tracejado. Agora vamos paresentar alguns padr\~{o}es de preenchimento que s\~{a}o definidos pela op\c{c}\~{a}o \lcode{pattern}\index{path@\lcode{path}!pattern@\lcode{pattern}}.

Para utilizar os padr\~{o}es predefinidos \'{e} necess\'{a}rio carregar a biblioteca \lcode{patterns}\index{library@\textit{library}!\lcode{patterns}}, i.e, adicionar a seguinte linha.
\begin{code}
    \usetikzlibrary{patterns}
\end{code}
no pre\^{a}mbulo do documento.

\example{codes/path_pattern@tikz_for_teachers}

Para atribuir um cor ao padr\~{a}o a ser utilizado deve-se utilizar a op\c{c}\~{a}o \lcode{pattern color}\index{path@\lcode{path}!pattern color@\lcode{pattern color}}.

\example{codes/path_pattern_color@tikz_for_teachers}

\subsection{Sombra}\index{sombra|see{\lcode{shade}}}
Para preencher uma figura com a impress\~{a}o de sombra utiliza-se a op\c{c}\~{a}o \lcode{shade}\index{path@\lcode{path}!shade@\lcode{shade}}.

\example{codes/path_shade01@tikz_for_teachers}

Al\'{e}m da op\c{c}\~{a}o \lcode{shade} tamb\'{e}m \'{e} poss\'{i}vel utilizar o comando \lcode{shade}\index{shade@\lcode{shade}} que produz o mesmo resultado.

Existem alguns modelos de sombra predefinidos que podem ser acessados com a op\c{c}\~{a}o \lcode{shading}\index{shading@\lcode{shading}} que recebe um dos seguintes tipos: \lcode{axis}, \lcode{radial} e \lcode{ball}.

\example{codes/path_shade02@tikz_for_teachers}

Para o tipo \lcode{axis} \'{e} poss\'{i}vel rotacionar a sombra com a op\c{c}\~{a}o \lcode{shading angle}. Para o cado se sobras verticais \'{e} poss\'{i}vel definir as cores das sobras com as op\c{c}\~{o}es \lcode{left color} e \lcode{right color}.

\example{codes/path_shade03@tikz_for_teachers}

Outros tipos de sombras s\~{a}o definidos na biblioteca \lcode{shadings}\index{library@\textit{library}!shadings@\lcode{shadings}}.

\subsection{Recortes}\index{recorte|see{\lcode{path clip}}}
\'{E} poss\'{i}vel recortar uma figura guardando apenas a regi\c{c}\~{a}o interna ao recorte e para isso utiliza-se a op\c{c}\~{a}o \lcode{clip}\index{path@\lcode{path}!clip@\lcode{clip}}.

\example{codes/path_clip01@tikz_for_teachers}

Como \'{e} observado pelo exemploa acima a op\c{c}\~{a}o \lcode{clip} afeta apenas os comandos subsequentes.

No lugar da op\c{c}\~{a}o \lcode{clip} para o comando \lcode{path} \'{e} poss\'{i}vel utilizar o comando \lcode{clip}\index{clip@\lcode{clip}}.

% Dicas
% Filename: p1_tips@tikz_for_teachers.tex
% This code is part of LaTeX with Vim.
% 
% Description: TikZ for teachers is free book about TikZ and Sage.
% 
% Created: 30.03.12 08:25:17 PM
% Last Change: 30.03.12 08:25:36 PM
% 
% Author: Raniere Gaia Costa da Silva, r.gaia.cs@gmail.com
% Organization:  
% 
% Copyright (c) 2010, 2011, 2012, Raniere Gaia Costa da Silva. All rights 
% reserved.
% 
% This file is license under the terms of a Creative Commons Attribution 
% 3.0 Unported License, or (at your option) any later version. More details
% at <http://creativecommons.org/licenses/by/3.0/>.
\chapter{Dicas}
Neste cap\'{i}tulo ser\'{a} apresentados algumas dicas que complementam os dois cap\'{i}tulos anteriores.

\section{Linhas Duplicadas}\index{linha!duplicada|see{\lcode{double}}}
Para duplicar uma linha pode-se utilizar a op\c{c}\~{a}o \lcode{double}\index{double@\lcode{double}}.

\example{codes/double01@tikz_for_teachers}

Por padr\~{a}o a dist\^{a}ncia entre as linhas \'{e} de 0.6pt e para alter\'{a}-la deve-se utilizar a op\c{c}\~{a}o \lcode{double distance}\index{double distance@\lcode{double distance}}.

\example{codes/double02@tikz_for_teachers}


\part{Exemplos}
% Fundamentos de Matem\'{a}tica Elementar 1: Conjuntos e Fun\c{c}\~{o}es
% Filename: p2_set_and_functions@tikz_for_teachers.tex
% This code is part of LaTeX with Vim.
% 
% Description: TikZ for teachers is free book about TikZ and Sage.
% 
% Created: 30.03.12 08:25:52 PM
% Last Change: 30.03.12 08:26:12 PM
% 
% Author: Raniere Gaia Costa da Silva, r.gaia.cs@gmail.com
% Organization:  
% 
% Copyright (c) 2010, 2011, 2012, Raniere Gaia Costa da Silva. All rights 
% reserved.
% 
% This file is license under the terms of a Creative Commons Attribution 
% 3.0 Unported License, or (at your option) any later version. More details
% at <http://creativecommons.org/licenses/by/3.0/>.
\chapter{Conjuntos e Fun\c{c}\~{o}es}

\section{No\c{c}\~{o}es de L\'{o}gica}

\section{Conjuntos}

\example{codes/sets01@tikz_for_teachers}

\example{codes/sets02@tikz_for_teachers}

\example{codes/sets03@tikz_for_teachers}

\example{codes/sets04@tikz_for_teachers}

\section{Conjuntos Num\'{e}ricos}

\section{Rela\c{c}\~{o}es}

\section{Fun\c{c}\~{o}es}





% Backmatter
\backmatter 

% References
\bibliographystyle{alpha}
\bibliography{../../references}

% Show index.
\addcontentsline{toc}{chapter}{Index}
\printindex
\end{document}
