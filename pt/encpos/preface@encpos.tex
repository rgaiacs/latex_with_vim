\chapter{Prefácio}
Esse matéria foi desenvolvido para o minicurso do Encontro Científico dos
Pós-graduandos do IMECC 2013 da Universidade Estadual de Campinas (UNICAMP).

O minicurso foi preparado para ser ministrado em três aulas com duração de uma
hora e vinte minutos cada com a seguinte distribuição didática:
\begin{description}
  \item[Aula 0] \lstinline+find / -name '*tex*'+

    Na primeira aula fala-se sobre a história do TeX e LaTeX, o significado de
    alguns nomes, alguns programas úteis.

    São escritos os primeiros arquivos \lstinline+.tex+ que não utilizam nenhum
    pacote. Algumas classes são apresentadas e dependendo do tempo é apresentado
    o \pkgname{beamer}.

    Alguns ambientes são apresentados, dentre eles as listas e tabelas.

  \item[Aula 1] O preâmbulo, onde a mágica começa

    Na segunda aula é construído um preâmbulo. Esse preâmbulo deve conter dentre
    outros pacotes aqueles voltados para internacionalização, codificação,
    formatação de página, inclusão de figuras.

  \item[Aula 2] AMSMATH, TikZ e BibTeX

    A terceira e última aula destina-se aos pacotes \pkgname{amsmath} (e
    família), \pkgname{tikz} e \pkgname{biblatex}. Esses são três pacotes muito
    utilizados.
\end{description}
