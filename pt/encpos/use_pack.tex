% Filename: use_pack.tex
% This code is part of 'Cursos CAMECC: Introducao ao LaTeX para o Curso 29 - Licenciatura em Matematica'
% 
% Description: This file is only a list of the packages used in the course.
% 
% Created: 07.06.12 11:30:10 AM
% Last Change: 07.06.12 11:30:30 AM
% 
% Authors:
% - Raniere Silva, r.gaia.cs@gmail.com
% 
% Organization: CAMECC - Centro Academico dos Estudantes do IMECC
% 
% Copyright (c) 2012, Raniere Silva. All rights reserved.
% 
% This work is licensed under the Creative Commons Attribution-ShareAlike 3.0 Unported License. To view a copy of this license, visit http://creativecommons.org/licenses/by-sa/3.0/ or send a letter to Creative Commons, 444 Castro Street, Suite 900, Mountain View, California, 94041, USA.
%
% This work is distributed in the hope that it will be useful, but WITHOUT ANY WARRANTY; without even the implied warranty of MERCHANTABILITY or FITNESS FOR A PARTICULAR PURPOSE.
%
\usepackage[brazil]{babel}
\usepackage[utf8]{inputenc}
\usepackage[T1]{fontenc} 
\usepackage{pb-diagram}
\usepackage{graphicx, color}
% \usepackage[top=3cm,left=2cm,right=2cm,bottom=3cm]{geometry} % Must be repeat in the file.
\usepackage{subfig}
\usepackage{enumerate}
\usepackage{algorithmic}
\usepackage{algorithm}
\usepackage{listings}
\usepackage{listingsutf8}
\usepackage{indentfirst}
\usepackage{multirow}
\usepackage{amsthm}
\usepackage{amsmath}
\usepackage{amsfonts}
\usepackage{amssymb}
\usepackage{makeidx}  % For index.
\makeindex  % For index.
\usepackage{tikz}
\usetikzlibrary{patterns}
\usepackage{epsfig}
\usepackage{latexsym}
\usepackage{makeidx}
\usepackage{url}
\usepackage{hyperref}
\usepackage{breakurl}
\usepackage{multicol}
\usepackage{parcolumns}
\usepackage[official]{eurosym}
\lstset{
breaklines=true,
literate={é}{{\'e}}1 {á}{{\'a}}1 {ã}{{\~a}}1,
}

\renewcommand{\lstlistingname}{Código}
\lstset{
language=TeX,                   % the language of the code
basicstyle=\ttfamily\footnotesize,     % the size of the fonts that are used for the code
%numbers=left,                   % where to put the line-numbers
%numberstyle=\footnotesize,      % the size of the fonts that are used for the line-numbers
%stepnumber=5,                   % the step between two line-numbers. If it's 1, each line 
%numbersep=5pt,                  % how far the line-numbers are from the code
showspaces=false,               % show spaces adding particular underscores
showstringspaces=false,         % underline spaces within strings
showtabs=false,                 % show tabs within strings adding particular underscores
tabsize=2,                      % sets default tabsize to 2 spaces
%captionpos=t,                   % sets the caption-position to bottom
breaklines=true,                % sets automatic line breaking
breakatwhitespace=false,        % sets if automatic breaks should only happen at whitespace
%caption={\texttt{\lstname}},    % show the filename of files included with \lstinputlisting;
}

% New commands and enviromments
\newcommand{\TikZ}{Ti\emph{k}Z }
\newcommand{\PGF}{\textsc{PGF} }
\newcommand{\lcode}[1]{\texttt{#1}}  % Code in the same line.
\lstnewenvironment{code}{}{}  % Code in new line.
\newcommand{\fcode}[1]{\lstinputlisting[firstline=18]{#1}}  % Code in a file.
\newcommand{\example}[1]{  % For examples.
\begin{minipage}[htb]{0.55\textwidth}
    \fcode{#1}
\end{minipage} \hfill \vrule \hfill
\begin{minipage}[htb]{0.35\textwidth}
    \include{#1}
\end{minipage}
}
