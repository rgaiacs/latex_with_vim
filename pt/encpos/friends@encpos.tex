\chapter{Utilitários}
Devido ao LaTeX ser modular, é interessante conhecer alguns dos executáveis que
costumam compor uma distribuição. Neste capítulo apresentaremos alguns destes
executáveis.

\section{Compilação}
Relacionado com a compilação e manipulação do arquivo \lstinline+.tex+ temos:
\begin{description}
  \item[latex] gera um arquivo \lstinline+dvi+ a partir de um arquivo LaTeX.
  \item[latexmk] automação completa do processo de compilação de documentos
    LaTeX.
  \item[luatex] extensão do \binname{pdftex} utilizando Lua como linguagem de
    script.
  \item[pdftex] gera um \lstinline+pdf+ a partir de uma arquivo TeX.
  \item[pdflatex] versão do \binname{pdftex} para arquivo LaTeX.
  \item[tex] gera um \lstinline+div+ a partir de um arquivo TeX.
\end{description}

Algumas das opções para alguns dos comandos anteriores são:
\begin{description}
  \item[-interaction mode] Configura o modo de iteração com o usuário. O modo
    deve ser uma das opções:
    \begin{itemize}
      \item \lstinline+batchmode+,
      \item \lstinline+nonstopmode+,
      \item \lstinline+scrollmode+, e
      \item \lstinline+errorstopmode+.
    \end{itemize}
  \item[-shell-escape] Habilita o uso de \lstinline+\write18{comando}+.
    \lstinline+comando+ pode ser qualquer instrução válida para a linha de
    comando. Esse comando é normalmente desabilitado por razões de seguranças
    mas necessários ao utilizar alguns pacotes para criar gráficos.
\end{description}

\section{Bibliografia}
Para o processamento de referências bibliográficas temos:
\begin{description}
  \item[bibtex] utiliza uma arquivo auxiliar gerado durante a compilação do
    arquivo \lstinline+.tex+ para criar o arquivo de bibliografia
    (\lstinline+.bbl+) que será posteriormente incorporado.
  \item[biber] é um substituto para o \binname{bibtex} escrito para ser
    utilizado em conjunto com o pacote \pkgname{biblatex}.
\end{description}

\section{Conversores}
Muitas vezes é preciso converter imagens que são incluídas durante a compilação
para outro formato. Para essa tarefa temos:
\begin{description}
  \item[a2ping] utilitário que converte imagens rasterizadas e vetoriais para
    EPS e PDF.
  \item[e2pall] procura no arquivo .tex pelo comando
    \lstinline+\includegraphics+ para encontrar os arquivos EPS utilizados e
    convertê-los para PDF.
\end{description}

\section{Gerenciador de pacotes}
Para o gerenciamento da distribuição LaTeX instalada, incluindo pacotes e
configurações, temos o \binname{tmlgt}.

\section{Outras funcionalidades}
Para remover todos os comentários e instruções do TeX e LaTeX de um arquivo
pode-se utiliza o \binname{detex}.

O índice remissivo é construído pelo comando \binname{makeindex}.

Para localizar e visualizar a documentação da distribuição, de classes ou de
pacotes temos:
\begin{description}
  \item[texdoc] é um utilitário de linha de comando.
  \item[texdoctk] é uma interface gráfica.
\end{description}

Para verificar o arquivo \lstinline+.tex+ por erros temos o \lstinline{lacheck}
lê o documento LaTeX e mostra mensagens caso encontre erros no documento.

Para comparar dois arquivos \lstinline+.tex+ temos:
\begin{description}
  \item[latexdiff] compara dois arquivos ignorando características da sintaxe do
    LaTeX.
  \item[texdiff] compara dois arquivos para criar uma versão mostrando as
    diferenças.
\end{description}

Para navegar do código (La)TeX para o resultado após a compilação e fazer o
caminho contrário de maneira sincronizada temos o \binname{synctex}.

\section{Relacionados com PDF}
Atualmente, o formato de saída dos documentos escritos utilizando (La)TeX é o
PDF. Poppler (ou libpoppler) é uma biblioteca para acessar arquivos no formato
PDF que disponibiliza alguns binários enventualmente úteis:
\begin{description}
  \item[pdfimages] extrator de imagens.
  \item[pdfinfo] informações do documento.
  \item[pdfseparate] ferramenta de extração de página.
  \item[pdftoppm] conversor de PDF para imagens PPM/PNG/JPEG.
  \item[pdftotext] extrator de texto.
  \item[pdfunite] ferramenta de mesclagem de documentos.
\end{description}

Além da biblioteca Poppler, outra biblioteca bastante útil é a Ghostscript que
processa os arquivos PostScript. Para converter um arquivo \lstinline+ps+ para
\lstinline+pdf+ pode-se utilizar o \binname{ps2pdf} presente no Ghostscript e
para a compressão do PDF:
\begin{lstlisting}
$ gs -sDEVICE=pdfwrite -dCompatibilityLevel=1.4 -dPDFSETTINGS=/resolucao \
> -dNOPAUSE -dQUIET -dBATCH -sOutputFile=saida.pdf entrada.pdf
\end{lstlisting}
onde \lstinline+resolucao+ deve ser substituído por um dos valores da lista
abaixo:
\begin{itemize}
  \item \lstinline+screen+: para resolução baixa,
  \item \lstinline+ebook+: para resolução média,
  \item \lstinline+printer+: para qualidade de impressão (alta),
  \item \lstinline+prepress+: para qualidade de pré-impressão,
  \item \lstinline+default+: padrão.
\end{itemize}
