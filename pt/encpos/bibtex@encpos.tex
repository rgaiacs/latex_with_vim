\chapter{Referência bibliográfica}
O ambiente \envname{thebibliography} é utilizado para a inclusão da referência
bibliográfica. Como ele exige um grande trabalho para ser utilizado e é difícil
reutilizá-lo foi desenvolvido o BibTeX (um banco de dados plano para referências
bibliográfica e um executável para construção do ambiente
\envname{thebibliography}). Posteriormente foi criado o pacote
\pkgname{biblatex} que extende o BibTeX. A seguir será apresentado um pouco do
BibTeX e do \envname{biblatex}.

\section{BibTeX}
O ``banco de dados'' corresponde a um arquivo de texto com a extensão
\lstinline+.bib+. Cada referência no BibTeX segue a seguinte estrutura:
\begin{lstlisting}
@tipo{identificador,
campo1 = {valor do campo 1},
campo2 = {valor do campo 2},
campo3 = {valor do campo 3},
...
}
\end{lstlisting}

Uma lista com alguns dos \lstinline+tipo+'s permitido pelo BibTeX é
apresentada na Tabela~\ref{tab:bibtex_entry_type}.
\begin{table}[h!tb]
  \centering
  \caption{\textsf{tipo}'s disponíveis no BibTeX padrão.}
  \label{tab:bibtex_entry_type}
  % Filename: bibtex_entry_type@latex_with_vim.tex
% This code is part of LaTeX with Vim.
% 
% Description: LaTeX with Vim is free book about Vim, LaTeX and Git.
% 
% Created: 30.03.12 12:12:14 AM
% Last Change: 30.03.12 12:12:19 AM
% 
% Author: Raniere Gaia Costa da Silva, r.gaia.cs@gmail.com
% Organization:  
% 
% Copyright (c) 2010, 2011, 2012, Raniere Gaia Costa da Silva. All rights 
% reserved.
% 
% This file is license under the terms of a Creative Commons Attribution 
% 3.0 Unported License, or (at your option) any later version. More details
% at <http://creativecommons.org/licenses/by/3.0/>.
\begin{tabular}{lp{0.8\textwidth}}
    \hline
    Código & Descrição \\ \hline
    \textsf{article} & Um artigo presente em algum periódico, revista, jornal que forme uma unidade própria e possua título. \\
    \textsf{book} & Um livro com um ou mais autores que levam crédito pela obra. \\
    \textsf{inbook} & Uma parte de um livro que forme uma unidade própria e possua título. \\
    %\textsf{suppbook} & Um material suplementar de um livro. \\
    \textsf{booklet} & Material com as características de um livro, mas que não foi formalmente publicado. \\
    %\textsf{collection} & Um livro composto dos trabalhos de vários autores, normalmente possui um editor. \\
    \textsf{incollection} & Uma parte de um livro composto dos trabalhos de vários autores, normalmente possui um editor. \\
    %\textsf{suppcollection} & Um suplemento de um livro composto dos trabalhos de vários autores, normalmente possui um editor. \\
    \textsf{proceedings} & Uma palestra de uma conferência. \\
    \textsf{inproceedings} & Um artigo apresentado em uma conferência. \\
    %\textsf{periodical} & Um periódico em sua totalidade. \\
    %\textsf{suppperiodical} & Um suplemento de um periódico. \\
    \textsf{manual} & Um documento técnico, pode não estar disponível em versão impressa. \\
    \textsf{techreport} & Um documento técnico produzido por uma instituição de ensino, comércio \dots \\
    \textsf{mastersthesis} & Uma tese de mestrado escrita para uma instituição de ensino. \\
    \textsf{phdthesis} & Uma tese de doutorado escrita para uma instituição de ensino. \\
    %\textsf{thesis} & Uma tese escrita para uma instituição de ensino como requisito para uma formação. \\
    %\textsf{report} &  \\
    %\textsf{patent} & Uma patente ou requerimento de patente. \\
    %\textsf{reference} & Uma enciclopédia ou dicionário. \\
    \textsf{unpublished} & Um trabalho que não foi formalmente publicado, como um manuscrito. \\
    %\textsf{online} & Um documento disponível apenas on-line, como por exemplo um site. \\
    \textsf{misc} & Utilizado quando a obra não se encaixa nos \textsf{tipo}'s anteriores.
\end{tabular}

\end{table}

Uma lista com alguns dos \lcode{campo}'s permitido pelo BibTeX é apresentada na
Tabela~\ref{tab:bibtex_entry_field}.
\begin{table}[h!tb]
    \centering
    \caption{\textsf{campo}'s disponíveis no BibTeX padrão.}
    \label{tab:bibtex_entry_field}
    % Filename: bibtex_entry_field@latex_with_vim.tex
% This code is part of LaTeX with Vim.
% 
% Description: LaTeX with Vim is free book about Vim, LaTeX and Git.
% 
% Created: 30.03.12 12:11:50 AM
% Last Change: 30.03.12 12:11:56 AM
% 
% Author: Raniere Gaia Costa da Silva, r.gaia.cs@gmail.com
% Organization:  
% 
% Copyright (c) 2010, 2011, 2012, Raniere Gaia Costa da Silva. All rights 
% reserved.
% 
% This file is license under the terms of a Creative Commons Attribution 
% 3.0 Unported License, or (at your option) any later version. More details
% at <http://creativecommons.org/licenses/by/3.0/>.
\begin{tabular}{lp{0.8\textwidth}}
    \hline
    Código & Descrição \\ \hline
    \textsf{author} & Autor(es) da obra. \\
    \textsf{editor} & Editor da obra, caso exista. \\
    %\textsf{translator} & Tradutor(es) da obra. \\
    \textsf{publisher} & Editora da obra. \\
    \textsf{title} & Título da obra. \\
    %\textsf{subtitle} & Subtítulo da obra, caso exista. \\
    \textsf{booktitle} & Quando a obra encontra-se como parte de um livro utiliza-se este campo para o título do livro. \\
    %\textsf{booksubtitle} & Quando a obra encontra-se como parte de um livro utiliza-se este campo para o subtítulo do livro, caso exista. \\
    \textsf{journal} & Título do jornal ou periódico que contem a obra. \\
    %\textsf{journalsubtitle} & Subtítulo do jornal ou periódico, caso exista, que contem a obra. \\
    %\textsf{date} & Data de publicação da obra. \\
    \textsf{month} & Mês da publicação da obra. \\
    \textsf{year} & Ano da publicação da obra, deve ser um inteiro. \\
    \textsf{edition} & Edição da obra. Deve ser um número inteiro. \\
    %\textsf{eventdate} & Data de uma conferência. \\
    %\textsf{eventtitle} & Título de uma conferência. \\
    %\textsf{chapter} & Qualquer unidade da obra, como um capítulo ou seção. \\
    %\textsf{file} & Link para o \textsf{PDF} ou outra versão da obra. \\
    %\textsf{holder} & Detentor de uma patente. \\
    \textsf{howpublished} & Tipo de publicação não usual. \\
    \textsf{school} & Instituição detentora da obra. \\
    %\textsf{number} & Número de um jornal, revista ou livro, para o caso de uma série. \\
    \textsf{pages} & Uma página ou mais de um trabalho. \\
    %\textsf{url} & Endereço eletrônico de uma publicação, utilizado para o \textsf{tipo online}.  \\
    %\textsf{urldate} & Data de acesso do endereço eletrônico. \\
    %\textsf{doi} & Digital Object Identifier da obra. \\
    %\textsf{eid} & Eletronic Identifier para um artigo. \\
    %\textsf{isbn} & International Standard Book Number de um livro. \\
    %\textsf{issn} & International Standard Serial Number de um periódico.
    \textsf{note} & Alguma informação que não adequa-se aos \textsf{camp}'s anteriores.
\end{tabular}

\end{table}

Uma das grandes vantagens de se utilizar o BibTeX é que as chances de encontrar
o BibTeX de algum material na internet é extremamente alta. Tanto o Google
Scholar como o Google Books disponibilizam o BibTeX para todos os materiais
indexados em suas respectivas bases de dados.

\section{\lstinline+biblatex+}
O pacote \pkgname{biblatex} define o comando
\lstinline+\addbibresource{referencias.bib}+ que é inserido no \emph{preâmbulo}
e especifica o arquivo que armazena as referências bibliográficas, nesse caso
\lstinline+referencias.bib+ e o comando \lstinline+\printbibliography+ que é
inserido na posição onde deseja-se incluir as referências.

O estilo a ser utilizado nas referências bibliográficas é informado como uma
opção do pacote \pkgname{biblatex} como indicado a seguir:
\begin{code}
\usepackage[style=estilo]{biblatex}
\end{code}
Alguns dos estilos existentes são:
\begin{itemize}
  \item \lstinline+numeric+,
  \item \lstinline+alphabetic+,
  \item \lstinline+authoryear+, \dots
\end{itemize}

Para que uma entrada do bando de dados seja incluído na referência bibliográfica
ele precisa ser mencionada em algum dos arquivos \lstinline+.tex+ que compõe a
obra. Para mencionar uma referência utiliza-se uma das variantes do comando
\lstinline+\cite{id}+, onde \lstinline+id+ corresponde ao
\lstinline+identificador+ utilizado na entrada do BibTeX para a referência
desejada.

O comando \lstinline+\cite{id}+ insere o número da referência entre colchetes,
como mostrado abaixo:
\begin{table}[!h]
  \centering
  \begin{tabular}{lc}
    \hline
    Comando & Resultado \\ \hline
    \lstinline+\cite{Sauer:2004:Parcolumns}+ & \cite{Sauer:2004:Parcolumns} \\
    \lstinline+\cite{Neves:AprendendoLaTeX}+ & \cite{Neves:AprendendoLaTeX} \\
    \lstinline+\cite{Pakin:2009:Symbol}+ & \cite{Pakin:2009:Symbol} \\
    \lstinline+\cite{Moses:2007:Listings}+ & \cite{Moses:2007:Listings} \\ \hline
  \end{tabular}
\end{table}

Para inserir o nome dos autores e o número da referência entre colchetes,
utiliza-se o comando \lstinline+\textcite{id}+, como mostrado abaixo:
\begin{table}[!h]
  \centering
  \begin{tabular}{ll}
    \hline
    Comando & Resultado \\ \hline
    \lstinline+\textcite{Sauer:2004:Parcolumns}+ & \textcite{Sauer:2004:Parcolumns} \\
    \lstinline+\textcite{Neves:AprendendoLaTeX}+ & \textcite{Neves:AprendendoLaTeX} \\
    \lstinline+\textcite{Pakin:2009:Symbol}+ & \textcite{Pakin:2009:Symbol} \\
    \lstinline+\textcite{Moses:2007:Listings}+ & \textcite{Moses:2007:Listings} \\ \hline
  \end{tabular}
\end{table}

Para inserir apenas o nome dos autores utiliza-se o comando
\lstinline+\citeauthor{id}+, como mostrado abaixo:
\begin{table}[!h]
  \centering
  \begin{tabular}{ll}
    \hline
    Comando & Resultado \\ \hline
    \lstinline+\citeauthor{Sauer:2004:Parcolumns}+ & \citeauthor{Sauer:2004:Parcolumns} \\
    \lstinline+\citeauthor{Neves:AprendendoLaTeX}+ & \citeauthor{Neves:AprendendoLaTeX} \\
    \lstinline+\citeauthor{Pakin:2009:Symbol}+ & \citeauthor{Pakin:2009:Symbol} \\
    \lstinline+\citeauthor{Moses:2007:Listings}+ & \citeauthor{Moses:2007:Listings} \\ \hline
  \end{tabular}
\end{table}

Para inserir apenas o título da referência utiliza-se o comando
\lstinline+\citetitle{id}+, como mostrado abaixo:
\begin{table}[!h]
  \centering
  \begin{tabular}{ll}
    \hline
    Comando & Resultado \\ \hline
    \lstinline+\citetitle{Sauer:2004:Parcolumns}+ & \citetitle{Sauer:2004:Parcolumns} \\
    \lstinline+\citetitle{Neves:AprendendoLaTeX}+ & \citetitle{Neves:AprendendoLaTeX} \\
    \lstinline+\citetitle{Pakin:2009:Symbol}+ & \citetitle{Pakin:2009:Symbol} \\
    \lstinline+\citetitle{Moses:2007:Listings}+ & \citetitle{Moses:2007:Listings} \\ \hline
  \end{tabular}
\end{table}

Para inserir apenas o ano de publicação da referência utiliza-se o comando
\lstinline+\citeyear{id}+, como mostrado abaixo:
\begin{table}[!h]
  \centering
  \begin{tabular}{lc}
    \hline
    Comando & Resultado \\ \hline
    \lstinline+\citeyear{Sauer:2004:Parcolumns}+ & \citeyear{Sauer:2004:Parcolumns} \\
    \lstinline+\citeyear{Neves:AprendendoLaTeX}+ & \citeyear{Neves:AprendendoLaTeX} \\
    \lstinline+\citeyear{Pakin:2009:Symbol}+ & \citeyear{Pakin:2009:Symbol} \\
    \lstinline+\citeyear{Moses:2007:Listings}+ & \citeyear{Moses:2007:Listings} \\ \hline
  \end{tabular}
\end{table}

Para citações múltiplas, utiliza-se os comandos \lstinline+\cites{id1,id2,id3}+
ou \lstinline+\textcites{id1,id2,id3}+, como mostrado abaixo:
\begin{table}[!h]
  \centering
  \begin{tabular}{lc}
    \hline
    Comando & Resultado \\ \hline
    \lstinline+\cites{Neves:AprendendoLaTeX,Sauer:2004:Parcolumns}+ & \cites{Neves:AprendendoLaTeX,Sauer:2004:Parcolumns} \\
    \lstinline+\cites{Moses:2007:Listings,Pakin:2009:Symbol}+ & \cites{Moses:2007:Listings,Pakin:2009:Symbol} \\
    \lstinline+\textcites{Neves:AprendendoLaTeX,Sauer:2004:Parcolumns}+ & \textcites{Neves:AprendendoLaTeX,Sauer:2004:Parcolumns} \\
    \lstinline+\textcites{Moses:2007:Listings,Pakin:2009:Symbol}+ & \textcites{Moses:2007:Listings,Pakin:2009:Symbol} \\ \hline
  \end{tabular}
\end{table}

Por último, caso deseje incluir uma referência na referência bibliográfica mas
suprimi-la ao longo do texto você deve utilizar o comando
\lstinline+\nocite{id}+.
