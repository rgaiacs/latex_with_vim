% Filename: slide04.tex
% This code is part of 'Cursos CAMECC: Introducao ao LaTeX para o Curso 29 - Licenciatura em Matematica'
% 
% Description: This file correspond to the slides of using for cap04@camecc_lic.tex of the course.
% 
% Created: 07.06.12 11:30:10 AM
% Last Change: 07.06.12 11:30:30 AM
% 
% Authors:
% - Raniere Silva, r.gaia.cs@gmail.com
% 
% Organization: CAMECC - Centro Academico dos Estudantes do IMECC
% 
% Copyright (c) 2012, Raniere Silva. All rights reserved.
% 
% This work is licensed under the Creative Commons Attribution-ShareAlike 3.0 Unported License. To view a copy of this license, visit http://creativecommons.org/licenses/by-sa/3.0/ or send a letter to Creative Commons, 444 Castro Street, Suite 900, Mountain View, California, 94041, USA.
%
% This work is distributed in the hope that it will be useful, but WITHOUT ANY WARRANTY; without even the implied warranty of MERCHANTABILITY or FITNESS FOR A PARTICULAR PURPOSE.
%
\documentclass[12pt]{beamer}
\usetheme{CambridgeUS}
\let\Tiny=\tiny  % Redefine at least \Tiny for avoid warning
% Filename: use_pack.tex
% This code is part of 'Cursos CAMECC: Introducao ao LaTeX para o Curso 29 - Licenciatura em Matematica'
% 
% Description: This file is only a list of the packages used in the course.
% 
% Created: 07.06.12 11:30:10 AM
% Last Change: 07.06.12 11:30:30 AM
% 
% Authors:
% - Raniere Silva, r.gaia.cs@gmail.com
% 
% Organization: CAMECC - Centro Academico dos Estudantes do IMECC
% 
% Copyright (c) 2012, Raniere Silva. All rights reserved.
% 
% This work is licensed under the Creative Commons Attribution-ShareAlike 3.0 Unported License. To view a copy of this license, visit http://creativecommons.org/licenses/by-sa/3.0/ or send a letter to Creative Commons, 444 Castro Street, Suite 900, Mountain View, California, 94041, USA.
%
% This work is distributed in the hope that it will be useful, but WITHOUT ANY WARRANTY; without even the implied warranty of MERCHANTABILITY or FITNESS FOR A PARTICULAR PURPOSE.
%
\usepackage[brazil]{babel}
\usepackage[utf8]{inputenc}
\usepackage[T1]{fontenc} 
\usepackage{pb-diagram}
\usepackage{graphicx, color}
% \usepackage[top=3cm,left=2cm,right=2cm,bottom=3cm]{geometry} % Must be repeat in the file.
\usepackage{subfig}
\usepackage{enumerate}
\usepackage{algorithmic}
\usepackage{algorithm}
\usepackage{listings}
\usepackage{listingsutf8}
\usepackage{indentfirst}
\usepackage{multirow}
\usepackage{amsthm}
\usepackage{amsmath}
\usepackage{amsfonts}
\usepackage{amssymb}
\usepackage{makeidx}  % For index.
\makeindex  % For index.
\usepackage{tikz}
\usetikzlibrary{patterns}
\usepackage{epsfig}
\usepackage{latexsym}
\usepackage{makeidx}
\usepackage{url}
\usepackage{hyperref}
\usepackage{breakurl}
\usepackage{multicol}
\usepackage{parcolumns}
\usepackage[official]{eurosym}
\lstset{
breaklines=true,
literate={é}{{\'e}}1 {á}{{\'a}}1 {ã}{{\~a}}1,
}

\renewcommand{\lstlistingname}{Código}
\lstset{
language=TeX,                   % the language of the code
basicstyle=\ttfamily\footnotesize,     % the size of the fonts that are used for the code
%numbers=left,                   % where to put the line-numbers
%numberstyle=\footnotesize,      % the size of the fonts that are used for the line-numbers
%stepnumber=5,                   % the step between two line-numbers. If it's 1, each line 
%numbersep=5pt,                  % how far the line-numbers are from the code
showspaces=false,               % show spaces adding particular underscores
showstringspaces=false,         % underline spaces within strings
showtabs=false,                 % show tabs within strings adding particular underscores
tabsize=2,                      % sets default tabsize to 2 spaces
%captionpos=t,                   % sets the caption-position to bottom
breaklines=true,                % sets automatic line breaking
breakatwhitespace=false,        % sets if automatic breaks should only happen at whitespace
%caption={\texttt{\lstname}},    % show the filename of files included with \lstinputlisting;
}

% New commands and enviromments
\newcommand{\TikZ}{Ti\emph{k}Z }
\newcommand{\PGF}{\textsc{PGF} }
\newcommand{\lcode}[1]{\texttt{#1}}  % Code in the same line.
\lstnewenvironment{code}{}{}  % Code in new line.
\newcommand{\fcode}[1]{\lstinputlisting[firstline=18]{#1}}  % Code in a file.
\newcommand{\example}[1]{  % For examples.
\begin{minipage}[htb]{0.55\textwidth}
    \fcode{#1}
\end{minipage} \hfill \vrule \hfill
\begin{minipage}[htb]{0.35\textwidth}
    \include{#1}
\end{minipage}
}


% Para os slides com codigos do Beamer.
\newenvironment{slide}[1]
{\begin{frame}[fragile,environment=slide]
    #1}
{\end{frame}}

\begin{document}
\title[Introdução ao LaTeX - 04/04]{Cursos CAMECC \\ Introdução ao LaTeX para o Curso 29 \\ Licenciatura em Matemática \\ Aula 04}
\author{Raniere Silva}
\institute[CAMECC]{Centro Acadêmico dos Estudantes do IMECC}
\begin{frame}
    \titlepage
\end{frame}

\begin{frame}
    \frametitle{Licença}
    Este trabalho foi licenciado com a Licença Creative Commons Atribuição - CompartilhaIgual 3.0 Não Adaptada. Para ver uma cópia desta licença, visite \url{http://creativecommons.org/licenses/by-sa/3.0/} ou envie um pedido por carta para Creative Commons, 444 Castro Street, Suite 900, Mountain View, California, 94041, USA.
    %
    %Com base na obra disponível em \url{https://github.com/r-gaia-cs/latex_with_vim/}. 
    %
    %Podem estar disponíveis permissões adicionais ao âmbito desta licença em \url{https://github.com/r-gaia-cs/latex_with_vim/}.
    %
    \begin{center}
        \includegraphics[keepaspectratio=true]{../../figures/cc-by-sa.png}
    \end{center}
\end{frame}

\begin{frame}
    \frametitle{Sumário}
    \tableofcontents
\end{frame}

\section{Obtendo Ajuda}
\begin{frame}[fragile]
    \frametitle{RFM!!!}
    No terminal, utilize
    \begin{code}
texdoc tikz
    \end{code}
    para o manual do TikZ e
    \begin{code}
texdoc beamer
    \end{code}
    para o manual do Beamer.
\end{frame}

\section{TikZ}
\begin{frame}[fragile]
    \frametitle{Algumas formas}
    \begin{code}
\documentclass{article}
\usepackage{tikz}
\begin{document}
\begin{tikzpicture}
    \draw (0,0) -- (1,0);
    \draw (0,-1) -- (1,0) -- (2,-1);
    \draw (-2,-2) rectangle (0,-3);
    \fill (2,-2) circle (1);
\end{tikzpicture}
\end{document}
    \end{code}
\end{frame}

\begin{frame}[fragile]
    \frametitle{Sorvete} 
    \begin{code}
\documentclass{article}
\usepackage{tikz}
\begin{document}
\begin{tikzpicture}
    \draw[fill=orange, rotate=45] (0,1) rectangle (4,1.5);
    \draw[fill=red!70] (0,1) circle (2);
    \draw[fill=brown] (-2,0) -- (0,-4) -- (2,0);
\end{tikzpicture}
\end{document}
    \end{code}
\end{frame}

\begin{frame}[fragile]
    \frametitle{\^{A}ngulos} 
    \begin{code}
\documentclass{article}
\usepackage{tikz}
\begin{document}
\begin{tikzpicture}
    \draw[->] (0,0) -- (1,0) node[below right]{A} -- (2.5,0);
    \draw[->] (0,0) -- (0,2.5);
    \draw[dotted] (-2,-2) grid (2,2);
    \draw[blue] (0,0) circle (1);
    % (60:1) significa 60 graus e raio 1.
    \draw (0,0) node[below left]{O} -- (60:1) node[above right]{B};
    \draw[red] (.2,0) arc (0:60:0.2);
\end{tikzpicture}
\end{document}
    \end{code}
\end{frame}

\begin{frame}[fragile]
    \frametitle{Gráficos de Funções} 
    \begin{code}
\begin{tikzpicture}[domain=0:4]
    \draw[very thin,color=gray] (-0.1,-1.1) grid (3.9,3.9);
    \draw[->] (-0.2,0) -- (4.2,0) node[right] {$x$};
    \draw[->] (0,-1.2) -- (0,4.2) node[above] {$f(x)$};
    \draw[color=red] plot (\x,\x) node[right] {$f(x) =x$};
    \draw[color=blue] plot (\x,{sin(\x r)}) node[right] {$f(x) = \sin x$};
    \draw[color=orange] plot (\x,{0.05*exp(\x)}) node[right] {$f(x) = \frac{1}{20} e^x$};
\end{tikzpicture}
    \end{code}
    O exemplo acima foi retirado de ``The TikZ and PGF Packages - Manual for version 2.10'' de Till Tantau.
\end{frame}

\section{Beamer}
\begin{slide}
    \frametitle{Hello} 
    \begin{code}
\documentclass{beamer}
\begin{document}
\begin{frame}
    Hello.
\end{frame}
\end{document}
    \end{code}
\end{slide}

\begin{slide}
    \frametitle{Título} 
    \begin{code}
\documentclass{beamer}
\begin{document}
\begin{frame}
    \frametitle{Hello}
    Me chamo \ldots
\end{frame}
\end{document}
    \end{code}
\end{slide}

\begin{slide}
    \frametitle{Tema} 
    \begin{code}
\documentclass{beamer}
% Mais temas em http://www.hartwork.org/beamer-theme-matrix/
\usetheme{CambridgeUS}
\begin{document}
\begin{frame}
    \frametitle{Hello}
    Me chamo \ldots
\end{frame}
\end{document}
    \end{code}
\end{slide}

\begin{slide}
    \frametitle{``Animações''} 
    \begin{code}
\documentclass{beamer}
\usepackage{tikz}
\begin{document}
\begin{frame}
\begin{tikzpicture}
    \draw[fill=orange, rotate=45] (0,1) rectangle (4,1.5); \pause
    \draw[fill=red!70] (0,1) circle (2); \pause
    \draw[fill=brown] (-2,0) -- (0,-4) -- (2,0);
\end{tikzpicture}
\end{frame}
\end{document}
    \end{code}
\end{slide}

\begin{frame}
    \frametitle{Fim}
    \begin{center}
        Perguntas.
    \end{center}
\end{frame}
\end{document}
