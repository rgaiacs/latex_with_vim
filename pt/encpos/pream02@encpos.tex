\chapter{Alguns pacotes úteis}
No capítulo anterior foi apresentado três pacotes (\pkgname{inputenc},
\pkgname{babel} e \pkgname{geometry}) que costuma estar presentes em
todo documente LaTeX. Nesse capítulo vamos apresentar alguns outros pacotes mais
alguns pacotes.

\section{Cor}
Para alterar a cor\index{fonte!cor} do texto é necessário os pacotes
\pkgname{graphicx}\index{pacote!graphicx@\pkgname{graphicx}} e
\pkgname{color}\index{pacote!color@\pkgname{color}} e pode-se utilizar um dos
comandos: \lstinline!\textcolor!\index{comando!textcolor@\lstinline+\textcolor+}
ou \lstinline!\color!\index{comando!color@\lstinline+\color+}.

A seguir apresentamos um exemplo. \\
\example{codes/hello_color@latex_with_vim.tex}

\section{Endereços da internet}
Nos endereços da internet é muito comum a presença de caracteres especiais para
o LaTeX. Para inserir um endereço da internet facilmente pode-se utilizar o
comando \lstinline!\verb!\index{comando!verb@\lstinline+\verb+} que foi
apresentado anteriormente ou utilizar o comando
\lstinline!\url!\index{comando!url@\lstinline+\url+} disponível no pacote
\pkgname{url}\index{pacote!url@\pkgname{url}}.

\section{Hiperligação e metadados}
Uma das capacidades do \lcode{pdf} é possuir metadados (informações para serem lidas por
máquinas) e hiperligação internos e/ou externos (marcações ao longo do texto que
possibilita ao usuário uma leitura não linear do documento).

Hiperligações são muito úteis ao leitor para que esse localize facilmente o
texto a que uma referência cruzada refere-se. A criação das hiperligações é
feita ao incluir o pacote \pkgname{hyperref}.

A inclusão de alguns metadados também é feita pelo pacote \pkgname{hyperref}.
Para a inclusão nos metadados do \lcode{pdf} do título e autor da obra pode-se utilizar
\begin{code}
\hypersetup{
  pdfinfo={
    Title={Titulo da obra},
    Author={Nome do autor},
  }
}
\end{code}

\section{Figuras}
No LaTeX é possível inserir figuras\index{figura} contidas em um arquivo de
imagem ou desenhar uma\footnote{Ver a Seção~\ref{sse:tikz}}. Também podemos
adicionar uma legenda para a figura.

\subsection{Arquivos de imagem}
Para inserir arquivos de imagem é necessário o pacote
\pkgname{graphicx}\index{pacote!graphicx@\pkgname{graphicx}}. A imagem a ser
inserida pode encontrar-se em um dos seguintes formatos: \lcode{jpg},
\lcode{png}, \lcode{pdf} ou \lcode{eps}\footnote{Este formato requer instalada o
TeX Live 2011 ou superior pois a partir dessa versão o pacote para conversão do
arquivo \lcode{eps} para um formato suportado é nativa.}.

O comando
\lstinline!\includegraphics!\index{comando!includegraphics@\lstinline+\includegraphics+}
é o responsável por indicar a figura que será inserida, sendo a figura inserida
ao longo do texto. A síntaxe deste comando é
\begin{code}
  \includegraphics[parametro=comprimento]{arquivo}
\end{code}
em que \lcode{parametro} é um comando disponíveis (algumas opções disponíveis
são apresentadas na Tabela \ref{tab:figure_size}), \lcode{comprimento} é uma medida
para \lcode{parametro} e \lcode{arquivo} é o nome do arquivo que contem a imagem.
\begin{table}[!htb]
  \centering
  \caption{Opções disponíveis para \envname{parametro}.}
  \label{tab:figure_size}
  % Filename: figure_size@latex_with_vim.tex
% This code is part of LaTeX with Vim.
% 
% Description: LaTeX with Vim is free book about Vim, LaTeX and Git.
% 
% Created: 30.03.12 12:13:41 AM
% Last Change: 30.03.12 12:13:48 AM
% 
% Author: Raniere Gaia Costa da Silva, r.gaia.cs@gmail.com
% Organization:  
% 
% Copyright (c) 2010, 2011, 2012, Raniere Gaia Costa da Silva. All rights 
% reserved.
% 
% This file is license under the terms of a Creative Commons Attribution 
% 3.0 Unported License, or (at your option) any later version. More details
% at <http://creativecommons.org/licenses/by/3.0/>.
\begin{tabular}{lp{0.8\textwidth}}
    \hline
    Código & Descrição \\ \hline
    \textsf{width} & Corresponde a largura da figura. \\
    \textsf{height} & Corresponde a altura da figura. \\
    \textsf{scale} & Corresponde a escala da figura. \\
    \textsf{angle} & Corresponde a uma rotação no sentido horário. \\
    \textsf{page} & Apenas para \textsf{PDF}'s, indica a página a ser utilizada. \\ \hline
\end{tabular}

\end{table}

Uma dica é que para \lcode{comprimento} podemos utilizar medidas correspondente a
folha escolhida como por exemplo \lstinline!\textwidth! ou
\lstinline!\textheight!.\\
\example{codes/includegraphics@latex_with_vim.tex}

Maiores informações podem ser encontradas em
\url{http://en.wikibooks.org/wiki/LaTeX/Importing_Graphics}.

\subsection{\envname{figure}}
O ambiente \envname{figure}\index{ambiente!figure@\envname{figure}} possibilita
a inclusão de uma legenda para a figura e trabalha a mesma como um objeto
flutuante. A síntaxe deste ambiente é
\begin{code}
  \begin{figure}[place]
    imagem
    \caption{legenda}
    \label{P:imagem}
  \end{figure}
\end{code}
onde \lcode{place} é o parâmetro que indica onde a figura deve ser
preferencialmente inserida (as opções disponíveis são apresentadas na Tabela
\ref{tab:figure_place} e a opção padrão é \lcode{tbp}), \lcode{imagem}
corresponde ao código da figura a ser inserida,
\lstinline!\caption!\index{comando!caption@\lstinline+\caption+} é o comando
correspondente a legenda e \lcode{legenda} é o texto a ser apresentado como
legenda, \lstinline!\label! é o comando para referência cruzada como já
apresentado. \\
\example{codes/figure_centering@latex_with_vim.tex}
\begin{table}[!htb]
  \centering
  \caption{Opções disponíveis para \envname{place}.}
  \label{tab:figure_place}
  % Filename: figure_place@latex_with_vim.tex
% This code is part of LaTeX with Vim.
% 
% Description: LaTeX with Vim is free book about Vim, LaTeX and Git.
% 
% Created: 30.03.12 12:13:22 AM
% Last Change: 30.03.12 12:13:28 AM
% 
% Author: Raniere Gaia Costa da Silva, r.gaia.cs@gmail.com
% Organization:  
% 
% Copyright (c) 2010, 2011, 2012, Raniere Gaia Costa da Silva. All rights 
% reserved.
% 
% This file is license under the terms of a Creative Commons Attribution 
% 3.0 Unported License, or (at your option) any later version. More details
% at <http://creativecommons.org/licenses/by/3.0/>.
\begin{tabular}{lp{0.8\textwidth}}
    \hline
    Código & Descrição \\ \hline
    \textsf{h} & Na posição onde o código se encontra. \\
    \textsf{t} & No topo de uma página. \\
    \textsf{b} & No fim de uma página. \\
    \textsf{p} & Em uma página separada. \\
    \textsf{!} & Modifica algumas configurações a respeito de boa posição para objeto flutuante. \\ \hline
\end{tabular}

\end{table}

Uma dica útil é que o comando
\lstinline!\clearpage!\index{comando!clearpage@\lstinline+\clearpage+} que força
as figuras pendentes a serem inseridas.

Outras informações podem ser encontradas em
\url{http://en.wikibooks.org/wiki/LaTeX/Floats,_Figures_and_Captions}.
