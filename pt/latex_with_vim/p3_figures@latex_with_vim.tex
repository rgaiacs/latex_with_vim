% Filename: p3_figures@latex_with_vim.tex
% This code is part of LaTeX with Vim.
% 
% Description: LaTeX with Vim is free book about Vim, LaTeX and Git.
% 
% Created: 29.03.12 11:36:31 PM
% Last Change: 30.03.12 12:08:16 AM
% 
% Author: Raniere Gaia Costa da Silva, r.gaia.cs@gmail.com
% Organization:  
% 
% Copyright (c) 2010, 2011, 2012, Raniere Gaia Costa da Silva. All rights 
% reserved.
% 
% This file is license under the terms of a Creative Commons Attribution 
% 3.0 Unported License, or (at your option) any later version. More details
% at <http://creativecommons.org/licenses/by/3.0/>.
\chapter{Figuras}
No LaTeX é possível inserir figuras contidas em um arquivo de imagem ou desenhar uma\footnote{Os leitores interessados em desenhar uma figura utilizado diretamente no LaTeX devem procurar informações sobre o ambiente \textsf{picture}.}. Também podemos adicionar uma legenda para a figura.

\section{Arquivos de imagem}

Para inserir arquivos de imagem é necessário o pacote \textsf{graphicx}. Quando o arquivo de saída for um \textsf{PDF} a imagem a ser inserida pode encontrar-se em um dos seguintes formatos: \textsf{jpg}, \textsf{png} ou \textsf{pdf}.

O comando \textbackslash\textsf{includegraphics} é o responsável por indicar a figura que será inserida, sendo a figura inserida ao longo do texto.. A síntaxe deste comando é
\begin{latexcode}
    \includegraphics[parameter=length]{file}
\end{latexcode}
em que \textsf{parameter} é um comando disponíveis (algumas opções disponíveis são apresentadas na Tabela \ref{tab:figure_size}), \textsf{length} é uma medida para \textsf{parameter} e \textsf{file} é o nome do arquivo que contem a imagem.
\begin{table}[!htb]
    \centering
    \caption{Opções disponíveis para \textsf{parameter}.}
    \label{tab:figure_size}
    % Filename: figure_size@latex_with_vim.tex
% This code is part of LaTeX with Vim.
% 
% Description: LaTeX with Vim is free book about Vim, LaTeX and Git.
% 
% Created: 30.03.12 12:13:41 AM
% Last Change: 30.03.12 12:13:48 AM
% 
% Author: Raniere Gaia Costa da Silva, r.gaia.cs@gmail.com
% Organization:  
% 
% Copyright (c) 2010, 2011, 2012, Raniere Gaia Costa da Silva. All rights 
% reserved.
% 
% This file is license under the terms of a Creative Commons Attribution 
% 3.0 Unported License, or (at your option) any later version. More details
% at <http://creativecommons.org/licenses/by/3.0/>.
\begin{tabular}{lp{0.8\textwidth}}
    \hline
    Código & Descrição \\ \hline
    \textsf{width} & Corresponde a largura da figura. \\
    \textsf{height} & Corresponde a altura da figura. \\
    \textsf{scale} & Corresponde a escala da figura. \\
    \textsf{angle} & Corresponde a uma rotação no sentido horário. \\
    \textsf{page} & Apenas para \textsf{PDF}'s, indica a página a ser utilizada. \\ \hline
\end{tabular}

\end{table}

Uma dica é que para \textsf{lenfth} podemos utilizar medidas correspondente a folha escolhida como por exemplo \textbackslash\textsf{textwidth} ou \textbackslash\textsf{textheight}.\\
\begin{minipage}[t]{0.47\linewidth}
    \vspace{-8pt}
    \begin{latexcode}
        \vspace{0pt}
        \includegraphics[height=2cm]{fractal} \\
        Fonte: \url{http://egpjoinville.files.wordpress.com/2009/10/fractal-2.jpg}
    \end{latexcode}
\end{minipage} \hfill
\begin{minipage}[t]{0.47\linewidth}
    \vspace{0pt}
    \includegraphics[height=2cm]{figures/fractal} \\
    Fonte: \url{http://egpjoinville.files.wordpress.com/2009/10/fractal-2.jpg}
\end{minipage}

Maiores informações podem ser encontradas em  \url{http://en.wikibooks.org/wiki/LaTeX/Importing_Graphics}

\section{\textsf{figure}}

O ambiente \textsf{figure} possibilita a inclusão de uma legenda para a figura e trabalha a mesma como um objeto flutuante. A síntaxe deste ambiente é
\begin{latexcode}
    \begin{figure}[place]
        imagem
        \caption{legend}
        \label{P:imagem}
    \end{figure}
\end{latexcode}
onde \textsf{place} é o parâmetro que indica onde a figura deve ser preferencialmente inserida (as opções disponíveis são apresentadas na Tabela \ref{tab:figure_place} e a opção padrão é \textsf{tbp}), \textsf{imagem} corresponde ao código da figura a ser inserida, \textbackslash\textsf{caption} é o comando correspondente a legenda e \textsf{legend} é o texto a ser apresentado como legenda, \textbackslash\textsf{label} é o comando para referência cruzada como já apresentado.
\begin{table}[!htb]
    \centering
    \caption{Opções disponíveis para \textsf{place}.}
    \label{tab:figure_place}
    % Filename: figure_place@latex_with_vim.tex
% This code is part of LaTeX with Vim.
% 
% Description: LaTeX with Vim is free book about Vim, LaTeX and Git.
% 
% Created: 30.03.12 12:13:22 AM
% Last Change: 30.03.12 12:13:28 AM
% 
% Author: Raniere Gaia Costa da Silva, r.gaia.cs@gmail.com
% Organization:  
% 
% Copyright (c) 2010, 2011, 2012, Raniere Gaia Costa da Silva. All rights 
% reserved.
% 
% This file is license under the terms of a Creative Commons Attribution 
% 3.0 Unported License, or (at your option) any later version. More details
% at <http://creativecommons.org/licenses/by/3.0/>.
\begin{tabular}{lp{0.8\textwidth}}
    \hline
    Código & Descrição \\ \hline
    \textsf{h} & Na posição onde o código se encontra. \\
    \textsf{t} & No topo de uma página. \\
    \textsf{b} & No fim de uma página. \\
    \textsf{p} & Em uma página separada. \\
    \textsf{!} & Modifica algumas configurações a respeito de boa posição para objeto flutuante. \\ \hline
\end{tabular}

\end{table}

O exemplo a seguir foi gerado pelo código
\begin{latexcode}
    \begin{figure}[h!]
        \includegraphics[height=2cm]{fractal}
        \caption{Um exemplo de fractal.}
        \label{F:fractal}
    \end{figure}
\end{latexcode}
\begin{figure}[h!]
    \includegraphics[height=2cm]{figures/fractal}
    \caption{Um exemplo de fractal.}
    \label{F:fractal}
\end{figure}

Como podemos observar pelo exemplo anterior, a figura é alinhada, por padrão, com a margem esquerda. Quando deseja-se centralizá-la pode-se utilizar o comando \textbackslash\textsf{centering}, como exemplificado abaixo.
\begin{latexcode}
    \begin{figure}[h!]
        \centering
        \includegraphics[height=2cm]{fractal}
        \caption{Um exemplo de fractal.}
        \label{F:fractal}
    \end{figure}
\end{latexcode}
\begin{figure}[h!]
    \centering
    \includegraphics[height=2cm]{figures/fractal}
    \caption{Um exemplo de fractal.}
    \label{F:fractal}
\end{figure}

Uma dica útil é que o comando \textbackslash\textsf{clearpage} força as figuras pendentes a serem inseridas.

Outras informações podem ser encontradas em \url{http://en.wikibooks.org/wiki/LaTeX/Floats,_Figures_and_Captions}.

\section{Figuras lado a lado}

Algumas é preciso apresentar duas ou mais figuras lado a lado e para isso podemos utilizar o ambiente \textsf{minipage} ou o pacote \textsf{subfig}.

\subsection{\textsf{minipage}}

Como já vimos, o ambiente \textsf{minipage} permite construir duas ou mais colunas. Para inserir duas figuras lado a lado podemos utilizar o seguinte esquema:
\begin{latexcode}
    \begin{figure}[place]
        \begin{minipage}[pos]{width}
            imagem1
            \caption{legend1}
            \label{P:imagem1}
        \end{minipage}\hfill
        \begin{minipage}[pos]{width}
            imagem2
            \caption{legend2}
            \label{P:imagem2}
        \end{minipage}
    \end{figure}
\end{latexcode}

Ao utilizar o esquema anterior, cada figura recebe sua própria legenda, como pode ser observado no exemplo a seguir, de modo que para que cada figura receba uma sublegenda deve-se utilizar o pacote \textsf{subfig}.
\begin{latexcode}
    \begin{figure}[h!]
        \begin{minipage}[t]{0.47\linewidth}
            \centering
            \includegraphics[height=2cm]{fractal}
            \caption{Um exemplo de fractal.}
        \end{minipage} \hfill
        \begin{minipage}[t]{0.47\linewidth}
            \centering
            \includegraphics[height=2cm]{fractal2}
            \caption{Outro exemplo de fractal.}
        \end{minipage}
    \end{figure}
\end{latexcode}
\begin{figure}[h!]
    \begin{minipage}[t]{0.47\linewidth}
        \centering
        \includegraphics[height=2cm]{figures/fractal}
        \caption{Um exemplo de fractal.}
    \end{minipage} \hfill
    \begin{minipage}[t]{0.47\linewidth}
        \centering
        \includegraphics[height=2cm]{figures/fractal2}
        \caption{Outro exemplo de fractal.}
    \end{minipage}
\end{figure}

\subsection{\textsf{subfig}}

Para inserir duas figuras lado a lado podemos utilizar o seguinte esquema:
\begin{latexcode}
    \begin{figure}[place]
        \centering
        \subfloat[legend1]{\label{P:imagem1}image1}
        \subfloat[legend2]{\label{P:imagem2}imagem2}
        \caption{legend}
        \label{P:imagem}
    \end{figure}
\end{latexcode}

A seguir apresentamos um exemplo.
\begin{latexcode}
    \begin{figure}[h!]
        \centering
        \subfloat[Fractal 1.]{\label{F:fractal1}\includegraphics[height=2cm]{fractal}} \hspace{0.5cm}
        \subfloat[Fractal 2.]{\label{F:fractal2}\includegraphics[height=2cm]{fractal2}}
        \caption{Exemplos de fractais.}
        \label{F:fractais}
    \end{figure}
\end{latexcode}
\begin{figure}[h!]
    \centering
    \subfloat[Fractal 1.]{\label{F:fractal1}\includegraphics[height=2cm]{figures/fractal}} \hspace{0.5cm}
    \subfloat[Fractal 2.]{\label{F:fractal2}\includegraphics[height=2cm]{figures/fractal2}}
    \caption{Exemplos de fractais.}
    \label{F:fractais}
\end{figure}

Ao invés do pacote \textsf{subfig} pode-se utilizar o pacote \textsf{subfigure} que é muito semelhante ao primeiro.
