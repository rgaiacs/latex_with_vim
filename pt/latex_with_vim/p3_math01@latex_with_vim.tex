% Filename: p3_math01@latex_with_vim.tex
% This code is part of LaTeX with Vim.
% 
% Description: LaTeX with Vim is free book about Vim, LaTeX and Git.
% 
% Created: 29.03.12 11:40:38 PM
% Last Change: 29.03.12 11:41:14 PM
% 
% Author: Raniere Gaia Costa da Silva, r.gaia.cs@gmail.com
% Organization:  
% 
% Copyright (c) 2010, 2011, 2012, Raniere Gaia Costa da Silva. All rights 
% reserved.
% 
% This file is license under the terms of a Creative Commons Attribution 
% 3.0 Unported License, or (at your option) any later version. More details
% at <http://creativecommons.org/licenses/by/3.0/>.
\chapter{Expressões matemáticas: Parte 1} \label{sch:latex:math1}
Neste e nos próximos dois capítulos, abordaremos o modo matemático do LaTeX, tendo estes capítulos sido baseados em \cite{Graetzer:2007:MoreMath} e é necessário o uso dos pacotes \textsf{amsmath}, \textsf{amsfonts}, \textsf{amssymb} e \textsf{amsthm}.

Neste capítulo iremos apresentar o modo matemático e os comandos básicos para escrita de fórmulas e expressões matemáticas. No capítulo \ref{sch:latex:math2} apresentamos algumas tabelas e algumas dicas enquanto no capítulo \ref{sch:latex:math3} abordaremos alguns tópicos mais avançados do modo matemático.

\section{Modo matemático}
Para que expressões matemáticas seja processadas corretamentes, deve-se mudar do modo texto para o modo matemático, o que pode ser feito de várias maneiras.

A apresentação de expressões matemáticas pode ocorrer de duas maneiras: \textit{inline}, quando aparecem na mesma linha do texto, e \textit{displayed}, quando aparecem em uma linha própria e centralizada.

A seguir, informaremos como proceder para produzir expressões matemáticas \textit{inline} ou \textit{displayed}.

\subsection{\textit{Inline}}
Expressões matemáticas \textit{inline} devem ser iniciadas por \texttt{\$} e fechadas por \texttt{\$} ou iniciadas por \textbackslash\texttt{)} e fechadas por \textbackslash\texttt{)}. \\
\begin{minipage}[t]{0.47\linewidth} \vspace{-8pt}
    \begin{latexcode}
        $1 + 1 = 2$ \\
        \(1 + 1 = 2\)
    \end{latexcode}
\end{minipage} \hfill
\begin{minipage}[t]{0.47\linewidth} \vspace{0pt}
    $1 + 1 = 2$ \\
    \(1 + 1 = 2\)
\end{minipage}

\subsection{\textit{Displayed}}
Expressões matemáticas \textit{displayed} devem ser iniciadas por \texttt{\$\$} e fechadas por \texttt{\$\$} ou iniciadas por \textbackslash\texttt{[} e fechadas por \textbackslash\texttt{]}. \\
\begin{minipage}[t]{0.47\linewidth} \vspace{-8pt}
    \begin{latexcode}
        $$1 + 1 = 2$$
        \[1 + 1 = 2\]
    \end{latexcode}
\end{minipage} \hfill
\begin{minipage}[t]{0.47\linewidth} \vspace{0pt}
    $$1 + 1 = 2$$
    \[1 + 1 = 2\]
\end{minipage}

Alguns ambientes, como \textsf{equation}, \textsf{eqnarray} e \textsf{align}, também produzem expressões matemáticas \textit{displayed}.

\section{Primeiros comandos no modo matemático}

A seguir enunciaremos como proceder para produzir as primeiras equações, mas antes é importante saber que o modo matemático ignora os espaços, salvo quando este é indicado explicitamente, isto é, utiliza-se o comando \textbackslash.

\subsection{Operações aritméticas básicas}
As operações aritméticas básicas são escritas normalmente, exceto pela multiplicação que utiliza-se dos comandos \textbackslash\textsf{times} ou \textbackslash\textsf{cdot} e das frações representada pelo comando \textbackslash\textsf{frac}. \\
\begin{minipage}[t]{0.47\linewidth} \vspace{-8pt}
    \begin{latexcode}
        $a + b$ \\
        $a - b$ \\
        $a b$ \text{ ou } $a \times b$  \text{ ou } $a \cdot b$ \\
        $a / b$ \text{ ou } $\frac{a}{b}$
    \end{latexcode}
\end{minipage} \hfill
\begin{minipage}[t]{0.47\linewidth} \vspace{0pt}
    $a + b$ \\
    $a - b$ \\
    $a b$ \text{ ou } $a \times b$  \text{ ou } $a \cdot b$ \\
    $a / b$ \text{ ou } $\frac{a}{b}$
\end{minipage}

\subsection{Índices e expoentes}
Já indices e expoentes são indicados pelos respectivos comandos: underscore, \_, e caret,\textasciicircum. Por padrão apenas o primeiro símbolo depois do comando é alterado, quando for necessário mais de um símbolo deve-se utilizar chaves.

O símbolo prime, muito utilizado para derivadas, já vem posicionado corretamente. \\
\begin{minipage}[t]{0.47\linewidth} \vspace{-8pt}
    \begin{latexcode}
        $a a = a^2$ \\
        $a_1, a_2, \dots, a_11, a_{12}$ \\
        $f'(x)$
    \end{latexcode}
\end{minipage} \hfill
\begin{minipage}[t]{0.47\linewidth} \vspace{0pt}
    $a a = a^2$ \\
    $a_1, a_2, \dots, a_11, a_{12}$ \\
    $f'(x)$
\end{minipage}

\subsection{Limitadores}
Parênteses, colchetes e chaves são exemplos de limitadores. Pode-se utilizar parênteses e colchetes normalmente e para chaves os comandos \textbackslash\{ e \textbackslash\}. \\
\begin{minipage}[t]{0.47\linewidth} \vspace{-8pt}
    \begin{latexcode}
        $(a + b) + c = a + (b + c)$ \\
        $[a + b] + c = a + [b + c]$ \\
        $\{a + b\} + c = a + \{b + c\}$
    \end{latexcode}
\end{minipage} \hfill
\begin{minipage}[t]{0.47\linewidth} \vspace{0pt}
    $(a + b) + c = a + (b + c)$ \\
    $[a + b] + c = a + [b + c]$ \\
    $\{a + b\} + c = a + \{b + c\}$
\end{minipage}

Para expressões matemáticas longas é aconselável utilizar os comandos \textbackslash\texttt{left} e \textbackslash\texttt{right} anteriormente ao limitador para ajustá-lo verticalmente. \\
\begin{minipage}[t]{0.47\linewidth} \vspace{-8pt}
    \begin{latexcode}
        $\left( \frac{a}{b} \right) = a \left( \frac{1}{b} \right)$
    \end{latexcode}
\end{minipage} \hfill
\begin{minipage}[t]{0.47\linewidth} \vspace{0pt}
    $\left( \frac{a}{b} \right) = a \left( \frac{1}{b} \right)$
\end{minipage}

A Tabela \ref{tab:math_delimiter} apresenta outros limitadores disponíveis no LaTeX.
\begin{table}[h!tb]
    \centering
    \caption{Limitadores disponíveis no LaTeX.}
    \label{tab:math_delimiter}
\end{table}

\subsection{Matrizes}
Para a construção de matrizes utiliza-se o ambiente \textsf{matrix} onde as colunas são separadas por \& e as linhas por \textbackslash\textbackslash. \\
\begin{minipage}[t]{0.47\linewidth} \vspace{-8pt}
    \begin{latexcode}
        $\begin{matrix}
            2 & a+b \\
            \frac{a}{b} & a^2
        \end{matrix}$
    \end{latexcode}
\end{minipage} \hfill
\begin{minipage}[t]{0.47\linewidth} \vspace{0pt}
    $\begin{matrix}
        2 & a+b \\
        \frac{a}{b} & a^2
    \end{matrix}$
\end{minipage}

Destaca-se que o ambiente \textsf{matrix} só pode ser utilizado dentro do ambiente matemático e que na última linha não utiliza-se o comando \textbackslash\textbackslash, mudança de linha.

Pode-se utilizar limitadores envolvendo o ambiente \textsf{matrix} ou utilizar uma variante: \textsf{pmatrix} ou \textsf{vmatrix}. \\
\begin{minipage}[t]{0.47\linewidth} \vspace{-8pt}
    \begin{latexcode}
        $\begin{pmatrix}
            2 & a+b \\
            \frac{a}{b} & a^2
        \end{pmatrix} \\
        \begin{vmatrix}
            2 & a+b \\
            \frac{a}{b} & a^2
        \end{vmatrix}$
    \end{latexcode}
\end{minipage} \hfill
\begin{minipage}[t]{0.47\linewidth} \vspace{0pt}
    $\begin{pmatrix}
        2 & a+b \\
        \frac{a}{b} & a^2
    \end{pmatrix} \\
    \begin{vmatrix}
        2 & a+b \\
        \frac{a}{b} & a^2
    \end{vmatrix}$
\end{minipage}

\section{Acentos}
Os acentos disponíveis no modo matemático são apresentados na Tabela \ref{tab:math_accents}.
\begin{table}[h!tb]
    \centering
    \caption{Acentos disponíveis no modo matemático, utilizando como exemplo \textsf{a}.}
    \label{tab:math_accents}
\end{table}

\section{Textos}

Para a inclusão de textos dentro de expressões matemáticas utiliza-se o comando \textbackslash\textsf{text}. \\
\begin{minipage}[t]{0.47\linewidth} \vspace{-8pt}
    \begin{latexcode}
        $a = b,\text{por hipotese.} \\
        a = b,\text{ por hipotese.}$
    \end{latexcode}
\end{minipage} \hfill
\begin{minipage}[t]{0.47\linewidth} \vspace{0pt}
    $a = b,\text{por hipotese.} \\
    a = b,\text{ por hipotese.}$
\end{minipage}
Como podemos observar pelo exemplo, o comando \textbackslash\textsf{text} é sensivel a espaços.

\section{Equações e referenciação} \label{sse:latex:equation}
Para o uso de expressões matemáticas a serem referenciadas posteriormente, recomenda-se o ambiente \textsf{equation} em conjunto com o comando \textbackslash\textsf{label}. \\
\begin{minipage}[t]{0.47\linewidth} \vspace{-8pt}
    \begin{latexcode}
        \begin{equation}\label{E:TeoPit}
            a^2 = b^2 + c^2
        \end{equation}
    \end{latexcode}
\end{minipage} \hfill
\begin{minipage}[t]{0.47\linewidth} \vspace{0pt}
    \begin{equation}\label{E:TeoPit}
        a^2 = b^2 + c^2
    \end{equation}
\end{minipage}

Sendo que \textsf{E:TeoPit} correspondente ao parâmetro do comando \textbackslash\textsf{label}, como apresentado na seção \ref{sse:cross_reference}. A referência a equação ocorre pelo comando \textbackslash\textsf{ref} ou \textbackslash\textsf{eqref}. \\
\begin{minipage}[t]{0.47\linewidth} \vspace{-8pt}
    \begin{latexcode}
        Na equacao (\ref{E:TeoPit}) $a$ corresponde a hipotenusa de um triangulo e os catetos sao $b$ e $c$. \\
        A equacao \eqref{E:TeoPit} e conhecida como Teorema de Pitagoras.
    \end{latexcode}
\end{minipage} \hfill
\begin{minipage}[t]{0.47\linewidth} \vspace{0pt}
    Na equacao (\ref{E:TeoPit}) $a$ corresponde a hipotenusa de um triangulo e os catetos sao $b$ e $c$. \\
    A equacao \eqref{E:TeoPit} e conhecida como Teorema de Pitagoras.
\end{minipage}
Como notamos pelo exemplo, ao utilizar o comando \textbackslash\textsf{eqref} não precisamos envolvé-lo com parênteses.

\subsection{Tags}
O comando \textsf{\textbackslash tag} do LaTeX nomeia uma equação e a referência passa a ser feito por este. \\
\begin{minipage}[t]{0.47\linewidth} \vspace{-8pt}
    \begin{latexcode}
        Sem tag: \begin{equation}\label{E:TeoPit_st}
            a^2 + b^2 = c^2
        \end{equation} \\
        Com tag: \begin{equation}\label{E:TeoPit_ct} 
            \tag{Teorema de Pitagoras}
            a^2 + b^2 = c^2
        \end{equation} \\
        \eqref{E:TeoPit_st} e \eqref{E:TeoPit_ct} sao equivalentes.
    \end{latexcode}
\end{minipage} \hfill
\begin{minipage}[t]{0.47\linewidth} \vspace{0pt}
    Sem tag: \begin{equation}\label{E:TeoPit_st}
        a^2 + b^2 = c^2
    \end{equation} \\
    Com tag: \begin{equation}\label{E:TeoPit_ct} 
        \tag{Teorema de Pitagoras}
        a^2 + b^2 = c^2
    \end{equation} \\
    \eqref{E:TeoPit_st} e \eqref{E:TeoPit_ct} sao equivalentes.
\end{minipage}

No exemplo anterior nota-se que o nome atribuido pelo comando \textsf{\textbackslash tag} encontra-se entre parênteses, para excluir os parênteses pode-se utilizar o comando \textbackslash\textsf{tag}\textasteriskcentered . \\
\begin{minipage}[t]{0.47\linewidth} \vspace{-8pt}
    \begin{latexcode}
        Sem parenteses: \begin{equation}\label{E:TeoPit_ctp} 
            \tag*{Teorema de Pitagoras}
            a^2 + b^2 = c^2
        \end{equation}
    \end{latexcode}
\end{minipage}\hfill
\begin{minipage}[t]{0.47\linewidth} \vspace{0pt}
    Sem parenteses: \begin{equation}\label{E:TeoPit_ctp} 
        \tag*{Teorema de Pitagoras}
        a^2 + b^2 = c^2
    \end{equation}
\end{minipage}

Vale destacar que podemos utilizar o comando \textbackslash\textsf{label} como parâmetro do comando \textbackslash\textsf{tag}.

\section{\textbackslash\textsf{newtheorem}}
O comando \textbackslash\textsf{newtheorem} deve ser inserido no \textit{preâmbulo} e é responsável por criar um ambiente numerado para informações. Sua síntaxe é
\begin{latexcode}
    \newtheorem{nome}{texto}
\end{latexcode}
onde \textsf{nome} é o nome do ambiente a ser criado e \textsf{texto} é a sequência de caracteres que precede a numeração.

É possível que o ambiente a ser criado utilize também uma outra numeração disponível. No seguinte exemplo
\begin{latexcode}
    \newtheorem{nome}{texto}[section]
\end{latexcode}
utiliza-se a numeração da seção.

Quando desejar-se não numerar deve-se utilizar a síntaxe
\begin{latexcode}
    \newtheorem*{nome}{texto}
\end{latexcode}

Para fazer uso do novo ambiente deve-se utilizar a síntaxe padrão para um ambiente
\begin{latexcode}
    \begin{nome}
        ...
    \end{nome}
\end{latexcode}
ou ainda
\begin{latexcode}
    \begin{nome}[XXX]
        ...
    \end{nome}
\end{latexcode}
onde \textsf{XXX} é uma sequencia de caracteres que aparece entre parênteses logo após a numeração.

\section{\textsf{proof}}

O ambiente \textsf{proof} é destinada a demonstrações e caracterizado por terminar com o comando \textbackslash\textsf{qed}. \\
\begin{minipage}[t]{0.47\linewidth} \vspace{-8pt}
    \begin{latexcode}
        \begin{proof}
            $a^2 + b^2 = c^2$
        \end{proof}
    \end{latexcode}
\end{minipage} \hfill
\begin{minipage}[t]{0.47\linewidth} \vspace{0pt}
    \begin{proof}
        $a^2 + b^2 = c^2$
    \end{proof}
\end{minipage}

Para trocar o nome \textsf{Demonstração} que precede o ambiente \textsf{proof} pode-se utilizar a sintaxe
\begin{latexcode}
    \begin{proof}[XXX]
        ...
    \end{proof}
\end{latexcode}
onde \textsf{XXX} é a sequencia de caracteres que vai aparecer no lugar de \textsf{Demonstração}.

O ambiente \textsf{proof}, como podemos observar no exemplo abaixo, não trabalha adequadamente quando é finalizado com uma expressão matemática \textit{displayed}. \\
\begin{minipage}[t]{0.47\linewidth} \vspace{-8pt}
    \begin{latexcode}
        \begin{proof}
            $$a^2 + b^2 = c^2$$
        \end{proof}
    \end{latexcode}
\end{minipage} \hfill
\begin{minipage}[t]{0.47\linewidth} \vspace{0pt}
    \begin{proof}
        $$a^2 + b^2 = c^2$$
    \end{proof}
\end{minipage}

Para corrigir essa falha deve-se proceder como a seguir. \\
\begin{minipage}[t]{0.47\linewidth} \vspace{-8pt}
    \begin{latexcode}
        \begin{proof}
            $$a^2 + b^2 = c^2 \qedhere$$
        \end{proof}
    \end{latexcode}
\end{minipage} \hfill
\begin{minipage}[t]{0.47\linewidth} \vspace{0pt}
    \begin{proof}
        $$a^2 + b^2 = c^2 \qedhere$$
    \end{proof}
\end{minipage}
