% Filename: p3_bibtex@latex_with_vim.tex
% This code is part of LaTeX with Vim.
% 
% Description: LaTeX with Vim is free book about Vim, LaTeX and Git.
% 
% Created: 29.03.12 11:35:07 PM
% Last Change: 30.03.12 12:09:28 AM
% 
% Author: Raniere Gaia Costa da Silva, r.gaia.cs@gmail.com
% Organization:  
% 
% Copyright (c) 2010, 2011, 2012, Raniere Gaia Costa da Silva. All rights 
% reserved.
% 
% This file is license under the terms of a Creative Commons Attribution 
% 3.0 Unported License, or (at your option) any later version. More details
% at <http://creativecommons.org/licenses/by/3.0/>.
\chapter{Bibliografias - BibTeX} \label{sch:latex:bibtex}
Neste capítulo abordaremos como construir bibliográficas utilizando-se do BibTex.

\section{Introdução}

Para produzir a bibliografia deve-se adicionar as sequintes linhas de comando
\begin{latexcode}
    \bibliographystyle{estilo}
    \bibliography{banco}
\end{latexcode}
onde \textsf{estilo} é o nome do estilo que será utilizado para criar a bibliografia e \textsf{banco} é o nome dos arquivos \textsf{.bib} com o banco de dados bibliográfico. Também é possível utilizar o pacote \textsf{biblatex}, mas este funciona um pouco diferente.

Nas próximas seções trataremos de como incluir um item na bibliografia e posteriormente do arquivo \textsf{.bib}, dos \textsf{estilo}'s disponíveis para a bibliografia e de como compilar corretamente.

\section{\textbackslash\textsf{cite}}

A referência a uma obra bibliográfica ocorre pelo comando \textbackslash\textsf{cite} que funciona de forma muito semelhante ao comando \textbackslash\textsf{ref}, diferem apenas que o comando \textbackslash\textsf{ref} utiliza-se do comando \textbackslash\textsf{label} enquanto que o comando \textbackslash\textsf{cite} de uma referência existente em um arquivo \textsf{.bib}.

Uma das características do comando \textbackslash\textsf{cite} é ele inserir a indicação da bibliografia. Quando desejar-se inserir um item na bibliografia sem fazer mensão deste ao longo do texto deve-se utilizar o comando \textbackslash\textsf{nocite}.

Para inserir todos os items do arquivo \textsf{.bib} na bibliografia pode-se utilizar o comando \textbackslash\textsf{nocite\{\textasteriskcentered\}}.

\section{\textsf{.bib}}

A extensão \textsf{.bib} corresponde ao arquivo com o banco de dados a ser utilizado para a bibliografia.

Cada item do banco de dados apresenta a seguinte sintaxe:
\begin{latexcode}
    @tipo{nome,
    campo = "informacao",
    }
\end{latexcode}
onde \textsf{tipo} indica o tipo do material da referência, \textsf{nome} é a sequencia de caracteres que o identificará ao longo do arquivo \textsf{.tex}, \textsf{campo} é alguma informação a respeito da referência e \textsf{informacao} é o texto a ser impresso para o referido \textsf{campo}. Para referenciar um item utiliza-se do comando
\begin{latexcode}
    \cite{nome}
\end{latexcode}
como foi apresentado na seção anterior.

Para cada \textsf{tipo} disponível no BibTeX existem \textsf{campo}'s que são obrigatórios e outros opcionais. A impressão ou não de um campo também depende do \textsf{estilo} utilizado.

\subsection{\textsf{tipo}}
Uma lista com alguns dos \textsf{tipo}'s permitido pelo BibTeX é apresentada na Tabela \ref{tab:bibtex_entry_type}.
\begin{table}[h!tb]
    \centering
    \caption{\textsf{tipo}'s disponíveis no BibTeX padrão.}
    \label{tab:bibtex_entry_type}
    % Filename: bibtex_entry_type@latex_with_vim.tex
% This code is part of LaTeX with Vim.
% 
% Description: LaTeX with Vim is free book about Vim, LaTeX and Git.
% 
% Created: 30.03.12 12:12:14 AM
% Last Change: 30.03.12 12:12:19 AM
% 
% Author: Raniere Gaia Costa da Silva, r.gaia.cs@gmail.com
% Organization:  
% 
% Copyright (c) 2010, 2011, 2012, Raniere Gaia Costa da Silva. All rights 
% reserved.
% 
% This file is license under the terms of a Creative Commons Attribution 
% 3.0 Unported License, or (at your option) any later version. More details
% at <http://creativecommons.org/licenses/by/3.0/>.
\begin{tabular}{lp{0.8\textwidth}}
    \hline
    Código & Descrição \\ \hline
    \textsf{article} & Um artigo presente em algum periódico, revista, jornal que forme uma unidade própria e possua título. \\
    \textsf{book} & Um livro com um ou mais autores que levam crédito pela obra. \\
    \textsf{inbook} & Uma parte de um livro que forme uma unidade própria e possua título. \\
    %\textsf{suppbook} & Um material suplementar de um livro. \\
    \textsf{booklet} & Material com as características de um livro, mas que não foi formalmente publicado. \\
    %\textsf{collection} & Um livro composto dos trabalhos de vários autores, normalmente possui um editor. \\
    \textsf{incollection} & Uma parte de um livro composto dos trabalhos de vários autores, normalmente possui um editor. \\
    %\textsf{suppcollection} & Um suplemento de um livro composto dos trabalhos de vários autores, normalmente possui um editor. \\
    \textsf{proceedings} & Uma palestra de uma conferência. \\
    \textsf{inproceedings} & Um artigo apresentado em uma conferência. \\
    %\textsf{periodical} & Um periódico em sua totalidade. \\
    %\textsf{suppperiodical} & Um suplemento de um periódico. \\
    \textsf{manual} & Um documento técnico, pode não estar disponível em versão impressa. \\
    \textsf{techreport} & Um documento técnico produzido por uma instituição de ensino, comércio \dots \\
    \textsf{mastersthesis} & Uma tese de mestrado escrita para uma instituição de ensino. \\
    \textsf{phdthesis} & Uma tese de doutorado escrita para uma instituição de ensino. \\
    %\textsf{thesis} & Uma tese escrita para uma instituição de ensino como requisito para uma formação. \\
    %\textsf{report} &  \\
    %\textsf{patent} & Uma patente ou requerimento de patente. \\
    %\textsf{reference} & Uma enciclopédia ou dicionário. \\
    \textsf{unpublished} & Um trabalho que não foi formalmente publicado, como um manuscrito. \\
    %\textsf{online} & Um documento disponível apenas on-line, como por exemplo um site. \\
    \textsf{misc} & Utilizado quando a obra não se encaixa nos \textsf{tipo}'s anteriores.
\end{tabular}

\end{table}

\subsection{\textsf{campo}}

Uma lista com alguns dos \textsf{campo}'s permitido pelo BibTeX é apresentada na Tabela \ref{tab:bibtex_entry_field}.
\begin{table}[h!tb]
    \centering
    \caption{\textsf{campo}'s disponíveis no BibTeX padrão.}
    \label{tab:bibtex_entry_field}
    % Filename: bibtex_entry_field@latex_with_vim.tex
% This code is part of LaTeX with Vim.
% 
% Description: LaTeX with Vim is free book about Vim, LaTeX and Git.
% 
% Created: 30.03.12 12:11:50 AM
% Last Change: 30.03.12 12:11:56 AM
% 
% Author: Raniere Gaia Costa da Silva, r.gaia.cs@gmail.com
% Organization:  
% 
% Copyright (c) 2010, 2011, 2012, Raniere Gaia Costa da Silva. All rights 
% reserved.
% 
% This file is license under the terms of a Creative Commons Attribution 
% 3.0 Unported License, or (at your option) any later version. More details
% at <http://creativecommons.org/licenses/by/3.0/>.
\begin{tabular}{lp{0.8\textwidth}}
    \hline
    Código & Descrição \\ \hline
    \textsf{author} & Autor(es) da obra. \\
    \textsf{editor} & Editor da obra, caso exista. \\
    %\textsf{translator} & Tradutor(es) da obra. \\
    \textsf{publisher} & Editora da obra. \\
    \textsf{title} & Título da obra. \\
    %\textsf{subtitle} & Subtítulo da obra, caso exista. \\
    \textsf{booktitle} & Quando a obra encontra-se como parte de um livro utiliza-se este campo para o título do livro. \\
    %\textsf{booksubtitle} & Quando a obra encontra-se como parte de um livro utiliza-se este campo para o subtítulo do livro, caso exista. \\
    \textsf{journal} & Título do jornal ou periódico que contem a obra. \\
    %\textsf{journalsubtitle} & Subtítulo do jornal ou periódico, caso exista, que contem a obra. \\
    %\textsf{date} & Data de publicação da obra. \\
    \textsf{month} & Mês da publicação da obra. \\
    \textsf{year} & Ano da publicação da obra, deve ser um inteiro. \\
    \textsf{edition} & Edição da obra. Deve ser um número inteiro. \\
    %\textsf{eventdate} & Data de uma conferência. \\
    %\textsf{eventtitle} & Título de uma conferência. \\
    %\textsf{chapter} & Qualquer unidade da obra, como um capítulo ou seção. \\
    %\textsf{file} & Link para o \textsf{PDF} ou outra versão da obra. \\
    %\textsf{holder} & Detentor de uma patente. \\
    \textsf{howpublished} & Tipo de publicação não usual. \\
    \textsf{school} & Instituição detentora da obra. \\
    %\textsf{number} & Número de um jornal, revista ou livro, para o caso de uma série. \\
    \textsf{pages} & Uma página ou mais de um trabalho. \\
    %\textsf{url} & Endereço eletrônico de uma publicação, utilizado para o \textsf{tipo online}.  \\
    %\textsf{urldate} & Data de acesso do endereço eletrônico. \\
    %\textsf{doi} & Digital Object Identifier da obra. \\
    %\textsf{eid} & Eletronic Identifier para um artigo. \\
    %\textsf{isbn} & International Standard Book Number de um livro. \\
    %\textsf{issn} & International Standard Serial Number de um periódico.
    \textsf{note} & Alguma informação que não adequa-se aos \textsf{camp}'s anteriores.
\end{tabular}

\end{table}

Para o \textsf{campo} \textsf{author} algumas regras deve ser seguidas:
\begin{enumerate}
    \item É permitido escrever o nome de duas maneiras:
        \begin{enumerate}
            \item normal;
            \item começando pelo nome da família seguido de vírgula.
        \end{enumerate}
    \item Para dois ou mais autores deve-se separa cada um pelo comando \textsf{and}.
\end{enumerate}

Para os \textsf{campo}'s \textsf{title} e \textsf{subtitle} algumas regras deve ser seguidas:
\begin{enumerate}
    \item Não utilizar ponto final ao final do título.
    \item Palavras importantes devem ter a primeira letra maiúscula.
\end{enumerate}

\subsection{Fontes da internet e URLs}

Como pode ser observado na Tabela \ref{tab:bibtex_entry_field} não existe um \textsf{campo} adequado para incluir endereços da internet. Uma opção é preencher o \textsf{campo} \textsf{note} com o comando \textbackslash\textsf{url}.

\subsection{Gerenciadores}

Existem alguns softwares disponíveis que facilitam a criação do arquivo \textsf{.bib}, como por exemplo:
\begin{enumerate}
    \item KBibTeX: 
    \item JabRef: 
    \item Pybliographic: 
\end{enumerate}


\section{\textsf{estilo}}

Alguns dos \textsf{estilo}'s mais utilizados são: \textsf{plain}, \textsf{alpha}, \textsf{amsplain}, \textsf{amsalpha}, \textsf{siam} e \textsf{ieeetr}. Exemplos destes e outros \textsf{estilo}'s pode ser visualizados em \url{http://www.cs.stir.ac.uk/~kjt/software/latex/showbst.html}.

Para as normas da ABNT é possível utilizar o \textsf{estilo} encontrado em \url{http://www.if.ufrgs.br/hadrons/abnt/abnt.html}.

\section{Compilando}

Para que o arquivo de saída seja produzido corretamente é importante seguir os seguintes passos.
\begin{enumerate}
    \item Os arquivos \textsf{.tex} e \textsf{.bib} encontrarem-se no mesmo diretório.
    \item Compilar o arquivo \textsf{.tex}.
    \item Rodar o BibTeX.
    \item Compilar duas vezes arquivo \textsf{.tex}.
\end{enumerate}
