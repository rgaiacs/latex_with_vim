% Filename: p3_hello_world@latex_with_vim.tex
% This code is part of LaTeX with Vim.
% 
% Description: LaTeX with Vim is free book about Vim, LaTeX and Git.
% 
% Created: 29.03.12 11:37:53 PM
% Last Change: 29.03.12 11:39:37 PM
% 
% Author: Raniere Gaia Costa da Silva, r.gaia.cs@gmail.com
% Organization:  
% 
% Copyright (c) 2010, 2011, 2012, Raniere Gaia Costa da Silva. All rights 
% reserved.
% 
% This file is license under the terms of a Creative Commons Attribution 
% 3.0 Unported License, or (at your option) any later version. More details
% at <http://creativecommons.org/licenses/by/3.0/>.
\chapter{Hello world} \label{sch:latex:hello_world}
Nos capítulos anteriores foi apresentado quais os aplicativos necessários para trabalhar com LaTeX e como instalá-los, as duas partes principais do arquivo \lcode{.tex} e descrevemos o \textit{preâmbulo}. Apartir deste capítulo abordaremos a segunda parte do arquivo \lcode{.tex} denominada de \textit{informação} que corresponde ao texto a ser gerado e neste primeiro momento apresentaremos os mais básicos conhecimentos necessários.

\section{O teclado}

Em todos os teclados encontramos as 52 letras (26 letras minúsculas + 26 letras maiúsculas) do alfabeto americano, os dez dígitos indo-arábicos, seis sinais de pontuação (\verb+ , ; . ? ! : +) e quatro parenteses (\verb+ ( ) [ ] +). Todos estas teclas são interpretadas como elas mesmas pelo LaTeX.

Na seção \ref{sss:latex:babel} abordaremos como o LaTeX interpreta o espaço e enter (mudança de linha).

As teclas correspondentes a \verb+ ` +, acento grave, \verb+ ' +, apóstrofe, e \verb+ - +, hífen, são interpretadas pelo LaTeX de acordo com os caracteres adjacentes.

Os seis símbolos matemáticos (\verb: * + = < > / :) são interpretados de maneira diferentes quando no modo texto e no modo matemático\footnote{O modo matemático é apresentado nos capítulos \ref{sch:latex:math1}, \ref{sch:latex:math2}, \ref{sch:latex:math3} e \ref{sch:latex:math4}}.

Existem, também, 13 símbolos especiais (\verb+ # $ % & ~ _ ^ \ { } @ " |+) que são interpretados pelo LaTeX de acordo com os caractéres adjacentes.

Os demais caracteres disponíveis no teclado, quando utilizados, costumam produzir erro.

\section{Idioma} \label{sss:latex:babel}
O LaTeX foi desenvolvido inicialmente para o idioma inglês e, por isso, sua utilização em outros idiomas, pode ser um pouco complicada. Para facilitar o uso do LaTeX em outros idiomas pode-se utilizar o pacote \lcode{babel} de Johannes L. Braams.

Para que o pacote \lcode{babel} funcione adequadamente, deve-se configurar o LaTeX de uma forma especial de acordo com codificação utilizada pelo arquivo \lcode{.tex}. As codificações mais comuns são Latin1 e UFT-8\footnote{O pacote \lcode{listings} não funciona adequadamente com esta codificação e por isso é fortemente aconcelhável utilizar a codificação Latin1.} sendo que para arquivos codificados com Latin1 deve-se adicionar a seguinte linha no preâmbulo
\begin{latexcode}
    \usepackage[latin1]{inputenc}
\end{latexcode}
enquanto que para arquivos codificados com UFT-8
\begin{latexcode}
    \usepackage[utf8]{inputenc}
\end{latexcode}

É importante que o editor que esteja sendo usado também esteja configurado para trabalhar com a codificação especificada. Quando uma codificação errada estiver sendo usada, o editor pode trocar ou omitir alguns caracteres.

Está disponível no pacote \lcode{babel} as seguintes opções para o idioma português: \lcode{portuges}, \lcode{portuguese}, \lcode{brazil}, \lcode{brazilian}. As opções definem alguns ajustes referentes a traduções de alguns termos e uso de caixa alta e maiores detalhes podem ser encontrados em \cite{Braams:2008:Babel}.

Em resumo, para os usuários que trabalham no Windows as seguintes linhas de código devem ser adicionas ao preâmbulo
\begin{latexcode}
    \usepackage[latin1]{inputenc}
    \usepackage[brazil]{babel}
\end{latexcode}
enquanto que os usuários do Linux
\begin{latexcode}
    \usepackage[utf8]{inputenc}
    \usepackage[brazil]{babel}
\end{latexcode}

\section{Primeiro documento}

Com o que foi apresentado agora é possível gerar a saída para um arquivo \lcode{.tex} bem simples. \\ 
\begin{minipage}[t]{0.47\linewidth}
    \vspace{-8pt}
    \begin{latexcode}
        \documentclass[10pt,a4paper]{article}
        \begin{document}
        Hello world.
        \end{document}
    \end{latexcode}
\end{minipage} \hfill
\begin{minipage}[t]{0.47\linewidth} \vspace{0pt}
    Hello world.
\end{minipage}

Os exemplos que serão apresentados aparecerão seguindo o modelo acima, isto é, em duas colunas sendo a coluna da esquerda contendo o código LaTeX e a coluna da direita contendo a saída obtida. Por simplicidade, nos demais exemplos iremos apresentar apenas a \textit{informação}.

\section{Espaços, linhas, parágrafos e páginas} \label{sss:lates:space}

No LaTeX o espaço entre palavras apresenta uma particularidade, ao compilar o \lcode{MAIN.tex}, ele ignora dois ou mais espaços seguidos, como podemos observar a seguir.. \\
\begin{minipage}[t]{0.47\linewidth} \vspace{-8pt}
    \begin{latexcode}
        Hello  world.(2 spaces)
        Hello   world.(3 spaces)
    \end{latexcode}
\end{minipage} \hfill
\begin{minipage}[t]{0.47\linewidth} \vspace{0pt}
    Hello  world.(2 spaces)
    Hello   world.(3 spaces)
\end{minipage}

Quando for necessário gerar dois ou mais espaços seguidos deve-se utilizar a barra invertida entre os espaços como ilustrado a seguir. \\
\begin{minipage}[t]{0.47\linewidth} \vspace{-8pt}
    \begin{latexcode}
        Hello \ world.(2 spaces)
        Hello \ \ world.(3 spaces)
    \end{latexcode}
\end{minipage} \hfill
\begin{minipage}[t]{0.47\linewidth} \vspace{0pt}
    Hello \ world.(2 spaces)
    Hello \ \ world.(3 spaces)
\end{minipage}

Nos dois exemplos anteriores é possível verificar que a mudança de linha no código não produz uma nova linha no documento gerado. A mudança de linha no LaTeX é representada pelos comandos \textbackslash\textbackslash \ ou \textbackslash\lcode{newline}, como ilustrada a seguir. \\
\begin{minipage}[t]{0.47\linewidth} \vspace{-8pt}
    \begin{latexcode}
        Hello world.[1] \\
        Hello world.[2] \newline
        Hello world.[3]
    \end{latexcode}
\end{minipage} \hfill
\begin{minipage}[t]{0.47\linewidth} \vspace{0pt}
    Hello world.[1] \\
    Hello world.[2] \newline
    Hello world.[3]
\end{minipage}

Já a mudança de parágrafo é indicada por uma linha em branco. 

Quando for necessário forçar uma mudança de página utiliza-se o comando \textbackslash\lcode{newpage}. Assim como o LaTeX ignora dois ou mais espaços seguidos a mudança de linha e de página também é ignorada.

Por último é importante avisar que, uma regra tipográfica padrão, o primeiro parágrafo de  capítulo, seções, \dots, não é identado. Quando desejar-se identar o primeiro parágrago uma solução é utilizar o pacote \lcode{indentfirst}.

\section{Hifenização}

O LaTeX tenta balancear o tamanho das linhas a serem geradas e para isso utiliza-se de um banco de dados para hifenizar, quando necessário, alguma palavra.

Algumas vezes a hifenização ocorre de maneira inadequada e para corrigir devemos utilizar o comando \textbackslash\lcode{hyphenation} cujo parâmetro é uma lista de palavras, separadas por espaço, onde o comando - é utilizado para indicar onde a palavra pode ser separada.

\section{Acentos}
Para a acentuação deve-se proceder como na Tabela \ref{tab:diacritic}. Quando utilizado o pacote \lcode{babel}, apresentado na seção \ref{sss:latex:babel}, pode-se também proceder normalmente.
\begin{table}[!htb]
    \centering
    \caption{Acentuação (utilizando a vogal ``o'' para exemplo).} \label{tab:diacritic}
    % Filename: diacrict@latex_with_vim.tex
% This code is part of LaTeX with Vim.
% 
% Description: LaTeX with Vim is free book about Vim, LaTeX and Git.
% 
% Created: 30.03.12 12:12:36 AM
% Last Change: 30.03.12 12:12:40 AM
% 
% Author: Raniere Gaia Costa da Silva, r.gaia.cs@gmail.com
% Organization:  
% 
% Copyright (c) 2010, 2011, 2012, Raniere Gaia Costa da Silva. All rights 
% reserved.
% 
% This file is license under the terms of a Creative Commons Attribution 
% 3.0 Unported License, or (at your option) any later version. More details
% at <http://creativecommons.org/licenses/by/3.0/>.
\begin{tabular}{cc|cc|cc|cc}
    \hline
    Com. & Res. & Com. & Res. & Com. & Res. & Com. & Res. \\ \hline
    \lstinline!\'{o}! & \'{o} & \lstinline!\={o}! & \={o} & \lstinline!\u{o}! & \u{o} & \lstinline!\.{o}! & \.{o} \\
    \lstinline!\v{o}! & \v{o} & \lstinline!\r{o}! & \r{o} & \lstinline!\c{c}! & \c{c} & \lstinline!\t{oo}! & \t{oo} \\
    \lstinline!\^{o}! & \^{o} & \lstinline!\~{o}! & \~{o} & \lstinline!\"{o}! & \"{o} & \lstinline!\d{o}! & \d{o} \\
    \lstinline!\H{o}! & \H{o} & \lstinline!\b{o}! & \b{o} & \lstinline!\`{o}! & \`{o} & \lstinline!\i! & \i \\ \hline
\end{tabular}

\end{table}

\section{Caracteres especiais}
No LaTeX alguns caracteres apresentam forma própria de representação. A seguir enunciaremos alguns.

\subsection{Aspas}
Para as aspas não deve-se usar o caracter de aspas. Para abrir as aspas deve-se utilizar o acento simples e para fechar a aspa simples. \\
\begin{minipage}[t]{0.47\linewidth} \vspace{-8pt}
    \begin{latexcode}
        `Hello world.' (aspas simples) \\
        ``Hello world.'' (aspas dupla) \\
        "Hello world." (errado)
    \end{latexcode}
\end{minipage} \hfill
\begin{minipage}[t]{0.47\linewidth} \vspace{0pt}
    `Hello world.' (aspas simples) \\
    ``Hello world.'' (aspas dupla) \\
    "Hello world." (errado)
\end{minipage}

\subsection{Traço}
LaTeX admite três tipos de traço. \\
\begin{minipage}[t]{0.47\linewidth} \vspace{-8pt}
    \begin{latexcode}
        sem-terra \\
        08--10 hours \\
        Campinas --- SP
    \end{latexcode}
\end{minipage} \hfill
\begin{minipage}[t]{0.47\linewidth} \vspace{0pt}
    sem-terra \\
    08--10 hours \\
    Campinas --- SP
\end{minipage}

\subsection{Pontos sucessivos}
Utiliza-se o comando \textbackslash\lcode{dots} ou \textbackslash\lcode{ldots} para pontos sucessivos. \\
\begin{minipage}[t]{0.47\linewidth} \vspace{-8pt}
    \begin{latexcode}
        patatoes, carrots \ldots (correta) \\
        patatoes, carrots \dots (correta) \\
        patatoes, carrots ... (errada) \\
    \end{latexcode}
\end{minipage} \hfill
\begin{minipage}[t]{0.47\linewidth} \vspace{0pt}
    patatoes, carrots \ldots (correta) \\
    patatoes, carrots \dots (correta) \\
    patatoes, carrots ... (errada) \\
\end{minipage}

\subsection{Pontuação e demais símbolos}
Para pontuação e demais símbolos especias deve-se proceder como na Tabela \ref{tab:symbols}.
\begin{table}[h!tb]
    \centering
    \caption{Para pontuação e símbolos especias.}
    \label{tab:symbols}
    % File: symbols@latex-with-vim.tex
% This code is part of LaTeX with Vim.
% 
% Description: LaTeX with Vim is free book about Vim, LaTeX and Git.
% 
% Created: 30.03.12 12:19:38 AM
% Last Change: 30.03.12 12:19:44 AM
% 
% Author: Raniere Gaia Costa da Silva, r.gaia.cs@gmail.com
% Organization:  
% 
% Copyright (c) 2010, 2011, 2012, Raniere Gaia Costa da Silva. All rights 
% reserved.
% 
% This file is license under the terms of a Creative Commons Attribution 
% 3.0 Unported License, or (at your option) any later version. More details
% at <http://creativecommons.org/licenses/by/3.0/>.

\begin{tabular}{cc|cc|cc}
    \hline
    Comando & Resultado & Comando & Resultado & Comando & Resultado \\ \hline
    \textbackslash \& & \& & \textbackslash textasteriskcentered & \textasteriskcentered & \textbackslash textbackslash & \textbackslash \\
    \textbackslash textbar & \textbar & \textbackslash \{ & \{ & \textbackslash \} & \} \\
    \textbackslash texbullet & \textbullet & \textbackslash textasciitilde & \textasciitilde & \textbackslash textasciicircum & \textasciicircum \\
    \textbackslash copyright & \copyright & \textbackslash textregistered & \textregistered & \textbackslash texttrademark & \texttrademark \\
    \textbackslash textperiodcentered & \textperiodcentered & \textbackslash textexclamdown & \textexclamdown & \textbackslash textquestiondown & \textquestiondown \\
    \textbackslash \% & \% & \textbackslash textgreater & \textgreater & \textbackslash textless & \textless  \\
    \textbackslash \# & \# & \textbackslash S & \S & \textbackslash P & \P \\
    \textbackslash \_ & \_ & \textbackslash dag & \dag & \textbackslash ddag & \ddag \\
    \textbackslash pounds & \pounds & \textbackslash textsuperscript\{a\} & \textsuperscript{a} & \textbackslash textcircled\{a\} & \textcircled{a} \\
    \textbackslash textvisiblespace & \textvisiblespace & \textbackslash \$ & \$ & \textbackslash euro & \euro \\ \hline
\end{tabular}

\end{table}

Destaca-se que para que o símbolo \euro \ seja impresso é necessário que o \textit{preâmbulo} contenha a seguinte linha de código
\begin{latexcode}
    \usepackage[official]{eurosym}
\end{latexcode}

\section{Comentários}
Também é possível inserir comentários no arquivo \lcode{.tex}, utilizando-se para isso do caractere \% de forma que todo o texto posterior ao mesmo e na mesma linha é considerado comentário e não é processado. \\
\begin{minipage}[t]{0.47\linewidth} \vspace{-8pt}
    \begin{latexcode}
        Hello world.%comentario
    \end{latexcode}
\end{minipage} \hfill
\begin{minipage}[t]{0.47\linewidth} \vspace{0pt}
    Hello world.%comentario
\end{minipage}
