% Filename: p3_math02@latex_with_vim.tex
% This code is part of LaTeX with Vim.
% 
% Description: LaTeX with Vim is free book about Vim, LaTeX and Git.
% 
% Created: 29.03.12 11:41:27 PM
% Last Change: 29.03.12 11:42:24 PM
% 
% Author: Raniere Gaia Costa da Silva, r.gaia.cs@gmail.com
% Organization:  
% 
% Copyright (c) 2010, 2011, 2012, Raniere Gaia Costa da Silva. All rights 
% reserved.
% 
% This file is license under the terms of a Creative Commons Attribution 
% 3.0 Unported License, or (at your option) any later version. More details
% at <http://creativecommons.org/licenses/by/3.0/>.
\chapter{Expressões matemáticas: Parte 2} \label{sch:latex:math2}
Neste capítulo continuaremos trabalhando com fórmulas matemáticas e nosso objetivo é apresentar alguns comandos úteis.

\section{Relações binárias}
\begin{table}[h!tbp]
    \caption{Relações binárias} \centering
    \label{tab:math_binary_relations}
    % Filename: math_binary_relations@latex_with_vim.tex
% This code is part of LaTeX with Vim.
% 
% Description: LaTeX with Vim is free book about Vim, LaTeX and Git.
% 
% Created: 30.03.12 12:15:58 AM
% Last Change: 30.03.12 12:16:09 AM
% 
% Author: Raniere Gaia Costa da Silva, r.gaia.cs@gmail.com
% Organization:  
% 
% Copyright (c) 2010, 2011, 2012, Raniere Gaia Costa da Silva. All rights 
% reserved.
% 
% This file is license under the terms of a Creative Commons Attribution 
% 3.0 Unported License, or (at your option) any later version. More details
% at <http://creativecommons.org/licenses/by/3.0/>.
\begin{tabular}{cc|cc|cc|cc}
    \hline
    Comando & Res. & Comando & Res. & Comando & Res. & Comando & Res. \\ \hline
    \textless & $<$ & \textbackslash\textsf{nless} & $\nless$ & \textgreater & $>$ & \textbackslash\textsf{ngtr} & $\ngtr$ \\ 
    \textbackslash\textsf{ll} & $\ll$ & \textbackslash\textsf{lll} & $\lll$ & \textbackslash\textsf{gg} & $\gg$ & \textbackslash\textsf{ggg} & $\ggg$ \\ 
    = & $=$ & \textbackslash\textsf{neq} & $\neq$ & : & $:$ &  \textbackslash\textsf{doteq} & $\doteq$  \\ 
    \textbackslash\textsf{sim} & $\sim$ & \textbackslash\textsf{nsim} & $\nsim$ & \textbackslash\textsf{cong} & $\cong$ &  \textbackslash\textsf{ncong} & $\ncong$  \\ 
    \textbackslash\textsf{simeq} & $\simeq$ & \textbackslash\textsf{approx} & $\approx$ &  \textbackslash\textsf{equiv} & $\equiv$ & & \\
    \textbackslash\textsf{leq} ou \textbackslash\textsf{le} & $\leq$ & \textbackslash\textsf{nleq} & $\nleq$ & \textbackslash\textsf{geq} ou \textbackslash\textsf{ge} & $\geq$ & \textbackslash\textsf{ngeq} & $\ngeq$ \\ 
    \textbackslash\textsf{leqslant} & $\leqslant$ &  \textbackslash\textsf{nleqslant} & $\nleqslant$  & \textbackslash\textsf{geqslant} & $\geqslant$ & \textbackslash\textsf{ngeqslant} & $\ngeqslant$ \\ 
    \textbackslash\textsf{eqslantless}  & $\eqslantless$  &  &  & \textbackslash\textsf{eqslantgtr} & $\eqslantgtr$ &  &  \\ 
    \textbackslash\textsf{leqq} & $\leqq$ & \textbackslash\textsf{nleqq} & $\nleqq$ & \textbackslash\textsf{geqq} & $\geqq$ & \textbackslash\textsf{ngeqq} & $\ngeqq$ \\ 
    \textbackslash\textsf{lesssim}  & $\lesssim$  & \textbackslash\textsf{lessapprox}  & $\lessapprox$ & \textbackslash\textsf{gtrsim} & $\gtrsim$ & \textbackslash\textsf{gtrapprox} & $\gtrapprox$ \\ 
    \textbackslash\textsf{prec} & $\prec$ & \textbackslash\textsf{nprec} & $\nprec$ & \textbackslash\textsf{succ} & $\succ$ & \textbackslash\textsf{nsucc} & $\nsucc$ \\ 
    \textbackslash\textsf{preceq} & $\preceq$ & \textbackslash\textsf{npreceq} & $\npreceq$  & \textbackslash\textsf{succeq} & $\succeq$ & \textbackslash\textsf{nsucceq} & $\nsucceq$ \\
    \textbackslash\textsf{in} & $\in$ & \textbackslash\textsf{notin} & $\notin$ & \textbackslash\textsf{owns} & $\owns$ &  &  \\ 
    \textbackslash\textsf{subset} & $\subset$ &  &  & \textbackslash\textsf{supset} & $\supset$ &  &  \\ 
    \textbackslash\textsf{subseteq} & $\subseteq$ & \textbackslash\textsf{nsubseteq}  & $\nsubseteq$ & \textbackslash\textsf{supseteq} & $\supseteq$ & \textbackslash\textsf{nsupseteq} & $\nsupseteq$ \\ 
    \textbackslash\textsf{subseteqq}  & $\subseteqq$ & \textbackslash\textsf{nsubseteqq}  & $\nsubseteqq$ & \textbackslash\textsf{supseteqq}  & $\supseteqq$ & \textbackslash\textsf{nsupseteqq} & $\nsupseteqq$ \\ 
    \textbackslash\textsf{sqsubset} & $\sqsubset$ & \textbackslash\textsf{sqsubseteq} & $\sqsubseteq$ & \textbackslash\textsf{sqsupset} & $\sqsupset$ & \textbackslash\textsf{sqsupseteq} & $\sqsupseteq$  \\ 
    \textbackslash\textsf{smile} & $\smile$ & \textbackslash\textsf{smallsmile} & $\smallsmile$ & \textbackslash\textsf{frown} & $\frown$ & \textbackslash\textsf{smallfrown} & $\smallfrown$ \\ 
    \textbackslash\textsf{perp} & $\perp$ &  &  & \textbackslash\textsf{models} & $\models$ &  &  \\ 
    \textbackslash\textsf{mid} & $\mid$ & \textbackslash\textsf{nmid} & $\nmid$ & \textbackslash\textsf{parallel} & $\parallel$ & \textbackslash\textsf{nparallel} & $\nparallel$ \\ 
    \textbackslash\textsf{shortmid} & $\shortmid$ & \textbackslash\textsf{nshortmid} & $\nshortmid$ & \textbackslash\textsf{shortparallel} & $\shortparallel$ & \textbackslash\textsf{nshortparallel} & $\nshortparallel$ \\ 
    \textbackslash\textsf{vdash} & $\vdash$ & \textbackslash\textsf{nvdash} & $\nvdash$ & \textbackslash\textsf{dashv} & $\dashv$ &  &  \\ 
    \textbackslash\textsf{vDash}  & $\vDash$ & \textbackslash\textsf{nvDash}  & $\nvDash$ & \textbackslash\textsf{Vdash}  & $\Vdash $ & \textbackslash\textsf{nVdash} & $\nVdash$ \\ 
    \textbackslash\textsf{propto} & $\propto$ & \textbackslash\textsf{asymp} & $\asymp$ & \textbackslash\textsf{bowtie} & $\bowtie$ & \textbackslash\textsf{Join} & $\Join$ \\ 
    \textbackslash\textsf{vartriangleleft}  & $\vartriangleleft$ & \textbackslash\textsf{ntriangleleft}  & $\ntriangleleft$ & \textbackslash\textsf{vartriangleright}  & $\vartriangleright$ & \textbackslash\textsf{ntriangleright} & $\ntriangleright$ \\ 
    \textbackslash\textsf{trianglelefteq}  & $\trianglelefteq$ & \textbackslash\textsf{ntrianglelefteq}  & $\ntrianglelefteq$ & \textbackslash\textsf{trianglerighteq}  & $\trianglerighteq$ & \textbackslash\textsf{ntrianglerighteq} & $\ntrianglerighteq$ \\ 
    \textbackslash\textsf{blacktriangleleft} & $\blacktriangleleft$ &  &  & \textbackslash\textsf{blacktriangleright} & $\blacktriangleright$ &  &  \\ 
    \textbackslash\textsf{between}  & $\between$ & \textbackslash\textsf{pitchfork} & $\pitchfork$ & \textbackslash\textsf{therefore} & $\therefore$ & \textbackslash\textsf{because} & $\because$ \\ \hline
\end{tabular}

\end{table}

\section{Operadores binários}
\begin{table}[h!tb]
    \centering
    \caption{Operadores binários}
    \label{tab:math_binary_operations}
    % Filename: math_binary_operations@latex_with_vim.tex
% This code is part of LaTeX with Vim.
% 
% Description: LaTeX with Vim is free book about Vim, LaTeX and Git.
% 
% Created: 30.03.12 12:15:37 AM
% Last Change: 30.03.12 12:15:42 AM
% 
% Author: Raniere Gaia Costa da Silva, r.gaia.cs@gmail.com
% Organization:  
% 
% Copyright (c) 2010, 2011, 2012, Raniere Gaia Costa da Silva. All rights 
% reserved.
% 
% This file is license under the terms of a Creative Commons Attribution 
% 3.0 Unported License, or (at your option) any later version. More details
% at <http://creativecommons.org/licenses/by/3.0/>.
\begin{tabular}{cc|cc|cc|cc}
    \hline
    Comando & Res. & Comando & Res. & Comando & Res. & Comando & Res. \\ \hline
    + & $+$ & - & $-$ & \textbackslash\textsf{pm} & $\pm$ & \textbackslash\textsf{mp} & $\mp$ \\
    \textbackslash\textsf{times} & $\times$ & \textbackslash\textsf{cdot} & $\cdot$ & \textbackslash\textsf{div} & $\div$ & \textbackslash\textsf{And} & $\And$ \\
    \textbackslash\textsf{setminus} & $\setminus$ & \textbackslash\textsf{smallsetminus} & $\smallsetminus$ & \textbackslash\textsf{dagger} & $\dagger$ & \textbackslash\textsf{ddagger} & $\ddagger$ \\
    \textbackslash\textsf{ast} & $\ast$ & \textbackslash\textsf{star} & $\star$ & \textbackslash\textsf{wedge} &    $\wedge$ &  \textbackslash\textsf{vee} & $\vee$ \\
    \textbackslash\textsf{cap} & $\cap$  & \textbackslash\textsf{cup} & $\cup$  & \textbackslash\textsf{sqcap} & $\sqcap$  & \textbackslash\textsf{sqcup} & $\sqcup$ \\
    \textbackslash\textsf{oplus} & $\oplus$ & \textbackslash\textsf{ominus} & $\ominus$ & \textbackslash\textsf{otimes} & $\otimes$ & \textbackslash\textsf{oslash} & $\oslash$ \\
    \textbackslash\textsf{odot} & $\odot$  & \textbackslash\textsf{bigcirc} & $\bigcirc$  & \textbackslash\textsf{circ} & $\circ$ & \textbackslash\textsf{bullet} & $\bullet$ \\
    \textbackslash\textsf{bigtriangleup} & $\bigtriangleup$ & \textbackslash\textsf{bigtriangledown} & $\bigtriangledown$  & \textbackslash\textsf{triangleleft} & $\triangleleft$ & \textbackslash\textsf{triangleright} & $\triangleright$  \\
    \textbackslash\textsf{diamond} & $\diamond$ & \textbackslash\textsf{wr} & $\wr$ & \textbackslash\textsf{amalg} & $\amalg$ \\ \hline
\end{tabular}

\end{table}

\section{Outros operadores}

No LaTeX encontramos três categorias de operadores: \textit{puros}, não precisam de parâmetro, \textit{com intervalos}, são definidos apenas em um intervalo e para indicá-lo utilizas-se underscore e caret, e \textit{limites}. Nas Tabelas \ref{tab:math_functions1}, \ref{tab:math_functions2} e \ref{tab:math_functions3} apresentamos os operadores disponíveis no LaTeX.

\begin{table}[h!tb]
    \centering
    \caption{Operadores \textit{puros}.}
    \label{tab:math_functions1}
    % Filename: math_functions1@latex_with_vim.tex
% This code is part of LaTeX with Vim.
% 
% Description: LaTeX with Vim is free book about Vim, LaTeX and Git.
% 
% Created: 30.03.12 12:16:46 AM
% Last Change: 30.03.12 12:16:51 AM
% 
% Author: Raniere Gaia Costa da Silva, r.gaia.cs@gmail.com
% Organization:  
% 
% Copyright (c) 2010, 2011, 2012, Raniere Gaia Costa da Silva. All rights 
% reserved.
% 
% This file is license under the terms of a Creative Commons Attribution 
% 3.0 Unported License, or (at your option) any later version. More details
% at <http://creativecommons.org/licenses/by/3.0/>.
\begin{tabular}{cc|cc|cc}
    \hline
    Com. & Res. & Com. & Res. & Com. & Res. \\ \hline
    \lstinline!\log! & $\log$ & \lstinline!\ln! & $\ln$ & \lstinline!\exp! & $\exp$ \\
    \lstinline!\arccos! & $\arccos$ & \lstinline!\arcsin! & $\arcsin$ & \lstinline!\arctan! & $\arctan$ \\
    \lstinline!\cos! & $\cos$ & \lstinline!\sin! & $\sin$ & \lstinline!\tan! & $\tan$ \\
    \lstinline!\csc! & $\csc$ & \lstinline!\sec! & $\sec$ & \lstinline!\cot! & $\cot$ \\
    \lstinline!\cosh! & $\cosh$ & \lstinline!\sinh! & $\sinh$ & \lstinline!\tanh! & $\tanh$ \\
    \lstinline!\lg! & $\lg$ & \lstinline!\arg! & $\arg$ & \lstinline!\hom! & $\hom$ \\
    \lstinline!\dim! & $\dim$ & \lstinline!\ker! & $\ker$ & \lstinline!\det! & $\det$ \\
    \lstinline!\gcd! & $\gcd$ & & & & \\ \hline
\end{tabular}

\end{table}

\begin{table}[h!tb]
    \centering
    \caption{Operadores \textit{com intervalos}.}
    \label{tab:math_functions2}
    % Filename: math_functions2@latex_with_vim.tex
% This code is part of LaTeX with Vim.
% 
% Description: LaTeX with Vim is free book about Vim, LaTeX and Git.
% 
% Created: 30.03.12 12:17:02 AM
% Last Change: 30.03.12 12:17:06 AM
% 
% Author: Raniere Gaia Costa da Silva, r.gaia.cs@gmail.com
% Organization:  
% 
% Copyright (c) 2010, 2011, 2012, Raniere Gaia Costa da Silva. All rights 
% reserved.
% 
% This file is license under the terms of a Creative Commons Attribution 
% 3.0 Unported License, or (at your option) any later version. More details
% at <http://creativecommons.org/licenses/by/3.0/>.
\begin{tabular}{cc|cc|cc|cc}
    \hline
    Comando & Resultado & Comando & Resultado & Comando & Resultado & Comando & Resultado \\ \hline
    \textbackslash\textsf{int} & $\int$ & \textbackslash\textsf{iint} & $\iint$ & \textbackslash\textsf{iiint} & $\iiint$ & \textbackslash\textsf{iiiint} & $\iiiint$ \\
    \textbackslash\textsf{idotsint} & $\idotsint$ & \textbackslash\textsf{oint} & $\oint$ & \textbackslash\textsf{prod} & $\prod$ & \textbackslash\textsf{coprod} & $\coprod$ \\ 
    \textbackslash\textsf{bigcap} & $\bigcap$ & \textbackslash\textsf{bigcup} & $\bigcup$ & \textbackslash\textsf{bigwedge} & $\bigwedge$ & \textbackslash\textsf{bigvee} & $\bigvee$ \\ 
    \textbackslash\textsf{bigsqcup} & $\bigsqcup$ & \textbackslash\textsf{biguplus} & $\biguplus$ & \textbackslash\textsf{bigotimes} & $\bigotimes$ & \textbackslash\textsf{bigoplus} & $\bigoplus$ \\ 
    \textbackslash\textsf{bigodot} & $\bigodot$ & \textbackslash\textsf{sum} & $\sum$ & & & & \\ \hline
\end{tabular}

\end{table}

\begin{table}[!htb]
    \centering
    \caption{\textit{Limites}.}
    \label{tab:math_functions3}
    % Filename: math_functions3@latex_with_vim.tex
% This code is part of LaTeX with Vim.
% 
% Description: LaTeX with Vim is free book about Vim, LaTeX and Git.
% 
% Created: 30.03.12 12:17:16 AM
% Last Change: 30.03.12 12:17:21 AM
% 
% Author: Raniere Gaia Costa da Silva, r.gaia.cs@gmail.com
% Organization:  
% 
% Copyright (c) 2010, 2011, 2012, Raniere Gaia Costa da Silva. All rights 
% reserved.
% 
% This file is license under the terms of a Creative Commons Attribution 
% 3.0 Unported License, or (at your option) any later version. More details
% at <http://creativecommons.org/licenses/by/3.0/>.
\begin{tabular}{cc|cc|cc}
    \hline
    Com. & Res. & Com. & Res. & Com. & Res.  \\ \hline
    \lstinline!\lim! & $\lim$ & \lstinline!\inf! & $\inf$ & \lstinline!\sup! & $\sup$ \\
     \lstinline!\max! & $\max$ & \lstinline!\injlim! & $\injlim$ & \lstinline!\liminf! & $\liminf$ \\
     \lstinline!\limsup! & $limsup$ & \lstinline!\min! & $\min$ & \lstinline!\varinjlim! & $\varinjlim$ \\
     \lstinline!\varliminf! & $\varliminf$ & \lstinline!\varlimsup! & $\varlimsup$ & \lstinline!\Pr! & $\Pr$ \\
    \lstinline!\projlim! & $\projlim$ & \lstinline!\varprojlim! & $\varprojlim$ & \\ \hline
\end{tabular}

\end{table}

Além destes operadores ainda temos o comando \textbackslash\textsf{sqrt} para raízes.

A seguir alguns exemplos. \\
\begin{minipage}[t]{0.47\linewidth} \vspace{-8pt}
    \begin{latexcode}
        $\sin \frac{\pi}{2} = 1$ \\
        $\int_0^\pi \sin x \, dx = 2$ \\
        $\lim_{x \to 0} x = 0$ \\
        $\sqrt{4} = 2$ \\
        $\sqrt[3]{8} = 2$
    \end{latexcode}
\end{minipage} \hfill
\begin{minipage}[t]{0.47\linewidth} \vspace{0pt}
    $\sin \frac{\pi}{2} = 1$ \\
    $\int_0^\pi \sin x \, dx = 2$ \\
    $\lim_{x \to 0} x = 0$ \\
    $\sqrt{4} = 2$ \\
    $\sqrt[3]{8} = 2$
\end{minipage}

O LaTeX também permite definirmos novos operadores \textit{puros} pelo comando
\begin{latexcode}
    \DeclareMathOperator{comando}{resultado}
\end{latexcode}
onde \textsf{comando} corresponde ao comando a ser utilizado para o novo operador e \textsf{resultado} é o texto apresentado como resultado. Este comando é muito utilizado para a tradução de alguns operadores, como por exemplo o seno.

\subsection{Binomial}
Utiliza-se o comando \textbackslash\textsf{binom} para os binomios. \\
\begin{minipage}[t]{0.47\linewidth} \vspace{-8pt}
    \begin{latexcode}
        $\binom{a}{b} + \binom{a+1}{b} = \binom{a+1}{b+1}$
    \end{latexcode}
\end{minipage} \hfill
\begin{minipage}[t]{0.47\linewidth} \vspace{0pt}
    $\binom{a}{b} + \binom{a+1}{b} = \binom{a+1}{b+1}$
\end{minipage}

\subsection{Congruências}
A forma mais comum para congruências corresponde ao uso dos comandos \textbackslash\textsf{equiv} e \textbackslash\textsf{pmod}. \\
\begin{minipage}[t]{0.47\linewidth} \vspace{-8pt}
    \begin{latexcode}
        $a \equiv b \pmod{v}$
    \end{latexcode}
\end{minipage} \hfill
\begin{minipage}[t]{0.47\linewidth} \vspace{0pt}
    $a \equiv b \pmod{v}$
\end{minipage}

\section{Setas}
\begin{table}[h!tb]
    \centering
    \caption{Setas}
    \label{tab:math_arrows}
    % Filename: math_arrows@latex_with_vim.tex
% This code is part of LaTeX with Vim.
% 
% Description: LaTeX with Vim is free book about Vim, LaTeX and Git.
% 
% Created: 30.03.12 12:15:12 AM
% Last Change: 30.03.12 12:15:22 AM
% 
% Author: Raniere Gaia Costa da Silva, r.gaia.cs@gmail.com
% Organization:  
% 
% Copyright (c) 2010, 2011, 2012, Raniere Gaia Costa da Silva. All rights 
% reserved.
% 
% This file is license under the terms of a Creative Commons Attribution 
% 3.0 Unported License, or (at your option) any later version. More details
% at <http://creativecommons.org/licenses/by/3.0/>.
\begin{tabular}{cc|cc|cc|cc}
    \hline
    Comando & Res. & Comando & Res. & Comando & Res. & Comando & Res. \\ \hline
    \textbackslash\textsf{leftarrow} & $\leftarrow$ & \textbackslash\textsf{rightarrow} & $\rightarrow$ & \textbackslash\textsf{longleftarrow} & $\longleftarrow$ & \textbackslash\textsf{longrightarrow} & $\longrightarrow$ \\
    \textbackslash\textsf{Leftarrow} & $\Leftarrow$ & \textbackslash\textsf{Rightarrow} & $\Rightarrow$ & \textbackslash\textsf{Longleftarrow} & $\Longleftarrow$ & \textbackslash\textsf{Longrightarrow} & $\Longrightarrow$ \\
    \textbackslash\textsf{nleftarrow} & $\nleftarrow$ & \textbackslash\textsf{nrightarrow} & $\nrightarrow$ & \textbackslash\textsf{nLeftarrow} & $\nLeftarrow$ & \textbackslash\textsf{nRightarrow} & $\nRightarrow$ \\
    \textbackslash\textsf{leftrightarrow} & $\leftrightarrow$ & \textbackslash\textsf{longleftrightarrow} & $\longleftrightarrow$ & \textbackslash\textsf{Leftrightarrow} & $\Leftrightarrow$ & \textbackslash\textsf{Longleftrightarrow} & $\Longleftrightarrow$ \\
    \textbackslash\textsf{nleftrightarrow} & $\nleftrightarrow$ & \textbackslash\textsf{nLeftrightarrow} & $\nLeftrightarrow$ & \textbackslash\textsf{dashleftarrow} & $\dashleftarrow$ & \textbackslash\textsf{dashrightarrow} & $\dashrightarrow$ \\
    \textbackslash\textsf{leftrightharpoons} & $\leftrightharpoons$ & \textbackslash\textsf{rightleftharpoons} & $\rightleftharpoons$ & \textbackslash\textsf{leftrightarrows} & $\leftrightarrows$ & \textbackslash\textsf{rightleftarrows} & $\rightleftarrows$ \\
    \textbackslash\textsf{mapsto} & $\mapsto$ & \textbackslash\textsf{longmapsto} & $\longmapsto$ & \textbackslash\textsf{iff} & $\iff$ & &  \\
    \textbackslash\textsf{uparrow} & $\uparrow$ & \textbackslash\textsf{downarrow} & $\downarrow$ & \textbackslash\textsf{Uparrow} & $\Uparrow$ & \textbackslash\textsf{Downarrow} & $\Downarrow$ \\
    \textbackslash\textsf{updownarrow} & $\updownarrow$ & \textbackslash\textsf{Updownarrow} & $\Updownarrow$ & \textbackslash\textsf{Lsh} & $\Lsh$ & \textbackslash\textsf{Rsh} & $\Rsh$ \\
    \textbackslash\textsf{curvearrowleft} & $\curvearrowleft$ & \textbackslash\textsf{curvearrowright} & $\curvearrowright$ & \textbackslash\textsf{circlearrowleft} & $\circlearrowleft$ & \textbackslash\textsf{circlearrowright} & $\circlearrowright$ \\ \hline
\end{tabular}

\end{table}

\section{Outros símbolos}
\begin{table}[!htb]
    \centering
    \caption{Outros símbolos matemáticos}
    \label{tab:math_others}
    % Filename: math_others@latex_with_vim.tex
% This code is part of LaTeX with Vim.
% 
% Description: LaTeX with Vim is free book about Vim, LaTeX and Git.
% 
% Created: 30.03.12 12:18:26 AM
% Last Change: 30.03.12 12:18:30 AM
% 
% Author: Raniere Gaia Costa da Silva, r.gaia.cs@gmail.com
% Organization:  
% 
% Copyright (c) 2010, 2011, 2012, Raniere Gaia Costa da Silva. All rights 
% reserved.
% 
% This file is license under the terms of a Creative Commons Attribution 
% 3.0 Unported License, or (at your option) any later version. More details
% at <http://creativecommons.org/licenses/by/3.0/>.
\begin{tabular}{cc|cc|cc|cc}
    \hline
    Comando & Res. & Comando & Res. & Comando & Res. & Comando & Res. \\ \hline
    \textbackslash\textsf{Re} & $\Re$ & \textbackslash\textsf{Im} & $\Im$ & \textbackslash\textsf{nabla} & $\nabla$ & \textbackslash\textsf{partial} & $\partial$  \\
    \textbackslash\textsf{infty} & $\infty$ & \textbackslash\textsf{emptyset} & $\emptyset$ & \textbackslash\textsf{varnothing} & $\varnothing$ \\
    \textbackslash\textsf{forall} & $\forall$ & \textbackslash\textsf{exists} & $\exists$ & \textbackslash\textsf{nexists} & $\nexists$ \\
    \textbackslash\textsf{angle} & $\angle$ & \textbackslash\textsf{measuredangle} & $\measuredangle$ & \textbackslash\textsf{sphericalangle} & $\sphericalangle$ \\
    \textbackslash\textsf{top} & $\top$ & \textbackslash\textsf{bot} & $\bot$ & \textbackslash\textsf{diagup} & $\diagup$ & \textbackslash\textsf{diagdown} & $\diagdown$ \\
    \textbackslash\textsf{triangle} & $\triangle$ & \textbackslash\textsf{triangledown} & $\triangledown$ & \textbackslash\textsf{blacktriangle} & $\blacktriangle$ & \textbackslash\textsf{blacktriangledown} & $\blacktriangledown$ \\
    \textbackslash\textsf{Diamond} & $\Diamond$ & \textbackslash\textsf{lozenge} & $\lozenge$ & \textbackslash\textsf{blacklozenge} & $\blacklozenge$ & \textbackslash\textsf{bigstar} & $\bigstar$ \\
    \textbackslash\textsf{Box} & $\Box$ & \textbackslash\textsf{square} & $\square$ & \textbackslash\textsf{blacksquare} & $\blacksquare$ \\
    \textbackslash\textsf{clubsuit} & $\clubsuit$ & \textbackslash\textsf{diamondsuit} & $\diamondsuit$ & \textbackslash\textsf{heartsuit} & $\heartsuit$ & \textbackslash\textsf{spadesuit} & $\spadesuit$ \\ \hline
\end{tabular}

\end{table}

\begin{table}[h!tb]
    \centering
    \caption{Alfabeto Hebreu}
    \label{tab:math_hebrew}
    % Filename: math_hebrew@latex_with_vim.tex
% This code is part of LaTeX with Vim.
% 
% Description: LaTeX with Vim is free book about Vim, LaTeX and Git.
% 
% Created: 30.03.12 12:18:04 AM
% Last Change: 30.03.12 12:18:16 AM
% 
% Author: Raniere Gaia Costa da Silva, r.gaia.cs@gmail.com
% Organization:  
% 
% Copyright (c) 2010, 2011, 2012, Raniere Gaia Costa da Silva. All rights 
% reserved.
% 
% This file is license under the terms of a Creative Commons Attribution 
% 3.0 Unported License, or (at your option) any later version. More details
% at <http://creativecommons.org/licenses/by/3.0/>.
\begin{tabular}{cc|cc|cc|cc}
    \hline
    Comando & Resultado & Comando & Resultado & Comando & Resultado & Comando & Resultado \\ \hline
    \textbackslash\textsf{aleph} & $\aleph$ & \textbackslash\textsf{beth} & $\beth$ & \textbackslash\textsf{daleth} & $\daleth$ & \textbackslash\textsf{gimel} & $\gimel$ \\ \hline
\end{tabular}

\end{table}

\begin{table}[h!tb]
    \centering
    \caption{Alfabeto Grego, letras minúsculas}
    \label{tab:math_greek}
    % Filename: math_greek@latex_with_vim.tex
% This code is part of LaTeX with Vim.
% 
% Description: LaTeX with Vim is free book about Vim, LaTeX and Git.
% 
% Created: 30.03.12 12:17:35 AM
% Last Change: 30.03.12 12:17:39 AM
% 
% Author: Raniere Gaia Costa da Silva, r.gaia.cs@gmail.com
% Organization:  
% 
% Copyright (c) 2010, 2011, 2012, Raniere Gaia Costa da Silva. All rights 
% reserved.
% 
% This file is license under the terms of a Creative Commons Attribution 
% 3.0 Unported License, or (at your option) any later version. More details
% at <http://creativecommons.org/licenses/by/3.0/>.
\begin{tabular}{cc|cc|cc|cc}
    \hline
    Com. & Res. & Com. & Res. & Com. & Res. & Com. & Res. \\ \hline
    \lstinline!\alpha! & $\alpha$ & \lstinline!\beta! & $\beta$ & \lstinline!\gamma! & $\gamma$ & \lstinline!\delta! & $\delta$ \\
    \lstinline!\epsilon! & $\epsilon$ & \lstinline!\zeta! & $\zeta$ & \lstinline!\eta! & $\eta$ & \lstinline!\theta! & $\theta$ \\
    \lstinline!\iota! & $\iota$ & \lstinline!\kappa! & $\kappa$ & \lstinline!\lambda! & $\lambda$ & \lstinline!\mu! & $\mu$ \\
    \lstinline!\nu! & $\nu$ & \lstinline!\xi! & $\xi$ & \lstinline!\pi! & $\pi$ & \lstinline!\rho! & $\rho$ \\
    \lstinline!\sigma! & $\sigma$ & \lstinline!\tau! & $\tau$ & \lstinline!\upsilon! & $\upsilon$ & \lstinline!\phi! & $\phi$ \\
    \lstinline!\chi! & $\chi$ & \lstinline!\psi! & $\psi$ & \lstinline!\omega! & $\omega$ & \lstinline!\digamma! & $\digamma$ \\
    \lstinline!\varepsilon! & $\varepsilon$ & \lstinline!\vartheta! & $\vartheta$ & \lstinline!\varkappa! & $\varkappa$ & \lstinline!\varpi! & $\varpi$ \\
    \lstinline!\varrho! & $\varrho$ & \lstinline!\varsigma! & $\varsigma$ & \lstinline!\varphi! & $\varphi$ & & \\ \hline
\end{tabular}

\end{table}

\begin{table}[h!tb]
    \centering
    \caption{Alfabeto Grego, letras maiúsculo}
    \label{tab:math_greek_capital}
    % Filename: math_greek_capital@latex_with_vim.tex
% This code is part of LaTeX with Vim.
% 
% Description: LaTeX with Vim is free book about Vim, LaTeX and Git.
% 
% Created: 30.03.12 12:17:50 AM
% Last Change: 30.03.12 12:17:56 AM
% 
% Author: Raniere Gaia Costa da Silva, r.gaia.cs@gmail.com
% Organization:  
% 
% Copyright (c) 2010, 2011, 2012, Raniere Gaia Costa da Silva. All rights 
% reserved.
% 
% This file is license under the terms of a Creative Commons Attribution 
% 3.0 Unported License, or (at your option) any later version. More details
% at <http://creativecommons.org/licenses/by/3.0/>.
\begin{tabular}{cc|cc|cc|cc}
    \hline
    Comando & Resultado & Comando & Resultado & Comando & Resultado & Comando & Resultado \\ \hline
    \textbackslash\textsf{Gamma} & $\Gamma$ & \textbackslash\textsf{Delta} & $\Delta$ & \textbackslash\textsf{Theta} & $\Theta$ & \textbackslash\textsf{Lambda} & $\Lambda$ \\
    \textbackslash\textsf{Xi} & $\Xi$ & \textbackslash\textsf{Pi} & $\Pi$ & \textbackslash\textsf{Sigma} & $\Sigma$ & \textbackslash\textsf{Upsilon} & $\Upsilon$ \\
    \textbackslash\textsf{Phi} & $\Phi$ & \textbackslash\textsf{Psi} & $\Psi$ & \textbackslash\textsf{Omega} & $\Omega$ \\
    \textbackslash\textsf{varGamma} & $\varGamma$ & \textbackslash\textsf{varDelta} & $\varDelta$ & \textbackslash\textsf{varTheta} & $\varTheta$ & \textbackslash\textsf{varLambda} & $\varLambda$ \\
    \textbackslash\textsf{varXi} & $\varXi$ & \textbackslash\textsf{varPi} & $\varPi$ & \textbackslash\textsf{varSigma} & $\varSigma$ & \textbackslash\textsf{varUpsilon} & $\varUpsilon$ \\
    \textbackslash\textsf{varPhi} & $\varPhi$ & \textbackslash\textsf{varPsi} & $\varPsi$ &
    \textbackslash\textsf{varOmega} & $\varOmega$ & & \\ \hline
\end{tabular}

\end{table}
