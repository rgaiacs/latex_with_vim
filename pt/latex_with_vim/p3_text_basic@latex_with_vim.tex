% Filename: p3_text_basic@latex_with_vim.tex
% This code is part of LaTeX with Vim.
% 
% Description: LaTeX with Vim is free book about Vim, LaTeX and Git.
% 
% Created: 29.03.12 11:47:22 PM
% Last Change: 30.03.12 12:07:34 AM
% 
% Author: Raniere Gaia Costa da Silva, r.gaia.cs@gmail.com
% Organization:  
% 
% Copyright (c) 2010, 2011, 2012, Raniere Gaia Costa da Silva. All rights 
% reserved.
% 
% This file is license under the terms of a Creative Commons Attribution 
% 3.0 Unported License, or (at your option) any later version. More details
% at <http://creativecommons.org/licenses/by/3.0/>.
\chapter{Trabalhando um texto: Parte 1} \label{css:latex:text-basic}
Neste capítulo apresentaremos as ferramentas do LaTeX disponíveis para organizar um texto. Começaremos pela modificação da fonte utilizada e como fazer citações.

Ao fim do capítulo abordaremos a escrita de versos.

\section{Fonte}
No LaTeX estão disponíveis algumas fontes opcionais. Comandos da forma \textbackslash\lcode{textXX} são responsáveis por alterar a fonte sendo que \lcode{XX} corresponde ao código da fonte a serem utilizados. A Tabela \ref{tab:text} apresenta alguns das opções disponíveis.
\begin{table}[!htb]
    \centering
    \caption{Opções disponíveis para \lcode{XX} da fonte.} \label{tab:text}
    \input{tables/text@latex_with_vim.tex}
\end{table}

A seguir é ilustrado as opções apresentadas na Tabela \ref{tab:text}. \\
\begin{minipage}[t]{0.47\linewidth} \vspace{-8pt}
    \begin{latexcode}
        Italico: \textit{novo texto}. \\
        Negrito: \textbf{novo texto}. \\
        Romano: \textrm{novo texto}. \\
        Sans serif: \textsf{novo texto}. \\
        Maquina de escrever: \texttt{novo texto}. \\
        Caixa alta: \textsc{novo texto}.
    \end{latexcode}
\end{minipage} \hfill
\begin{minipage}[t]{0.47\linewidth} \vspace{0pt}
    Italico: \textit{novo texto}. \\
    Negrito: \textbf{novo texto}. \\
    Romano: \textrm{novo texto}. \\
    Sans serif: \textsf{novo texto}. \\
    Maquina de escrever: \texttt{novo texto}. \\
    Caixa alta: \textsc{novo texto}.
\end{minipage}

\subsection{Edição direta}
Algumas vezes deseja-se inserir um texto que não deve ser interpretado. Isso é possível pelo ambiente \lcode{verbatim}, coloca o texto em uma nova linha, e pelo comando \textbackslash\lcode{verb}, coloca o texto no mesmo parágrafo.

Tanto o ambiente \lcode{verbatim} como o comando \textbackslash\lcode{verb} apresentam uma fonte própria. \\
\begin{minipage}[t]{0.47\linewidth} \vspace{-8pt}
    \begin{latexcode}
        \textsc{texto interpretado.} \\
        \verb+Texto nao interpretado.+
    \end{latexcode}
\end{minipage} \hfill
\begin{minipage}[t]{0.47\linewidth} \vspace{0pt}
    \textsc{texto interpretado.} \\
    \verb+Texto nao interpretado.+
\end{minipage}

Vale destacar que o comando \textbackslash\lcode{verb} é ``flexível'' quando ao delimitador, os caracteres \verb:+: e \verb+:+ normalmente exercem satisfatoriamente esta função.

\section{Tamanho}
Uma das maneiras de mudar o tamanho da fonte em uma parte do texto é utilizando um dos ambiente  ou comando de tamanho (a Tabela \ref{tab:op_tamanho_fonte} apresenta algumas opções disponíveis).
\begin{table}[h!tb]
    \centering
    \caption{Opções disponíveis para o tamanho da fonte, em ordem crescente.}
    \label{tab:op_tamanho_fonte}
    \begin{tabular}{lp{0.7\textwidth}}
        \hline
        Código & Descrição \\ \hline
        \lcode{tiny} & O menor tamanho possível. \\
        \lcode{SMALL} ou \lcode{scriptsize} &  \\
        \lcode{Small} ou \lcode{footnotesize} & Tamanho utilizado em notas de rodapé. \\
        \lcode{small} &  \\
        \lcode{normalsize} & Tamanho padrão. \\
        \lcode{large} & \\
        \lcode{Large} & \\
        \lcode{LARGE} & \\
        \lcode{huge} & \\
        \lcode{Huge} & O maior tamanho disponível. \\ \hline
    \end{tabular}
\end{table}

Destaca-se que os tamanhos são baseados no tamanho padrão. A seguir um exemplo. \\
\begin{minipage}[t]{0.47\linewidth} \vspace{-8pt}
    \begin{latexcode}
        {\tiny muito pequeno} \\
        {\small pequeno} \\
        fonte padrao \\
        {\Large grande} \\
        {\Huge enorme}
    \end{latexcode}
\end{minipage} \hfill
\begin{minipage}[t]{0.47\linewidth} \vspace{0pt}
    {\tiny muito pequeno} \\
    {\small pequeno} \\
    fonte padrao \\
    {\Large grande} \\
    {\Huge enorme}
\end{minipage}

\section{Cor}
Para alterar a cor do texto é necessário os pacotes \lcode{graphicx} e \lcode{color} e pode-se utilizar um dos comandos: \textbackslash\lcode{textcolor} ou \textbackslash\lcode{color}.

A seguir apresentamos um exemplo. \\
\begin{minipage}[t]{0.47\linewidth} \vspace{-8pt}
    \begin{latexcode}
        \textcolor{blue}{azul} \\
        {\color{blue}azul}
    \end{latexcode}
\end{minipage} \hfill
\begin{minipage}[t]{0.47\linewidth} \vspace{0pt}
    \textcolor{blue}{azul} \\
    {\color{blue}azul}
\end{minipage}

\section{Espaçamento}
Nesta seção abordaremos como inserir espaços ao longo do texto no LaTeX, mas antes é importante destacar que podemos suprimir espaços ao utilizar medidas negativas.

\subsection{Espaçamento horizontal}
Para produzir um espaço horizontal utiliza-se o comando \textbackslash\lcode{hspace} que tem como parâmetro o tamanho do espaço a ser inserido. Se o comando ocorrer entre duas linhas ou no início de uma linha o LaTeX não produz o espaço e para este caso devemos utilizar  \textbackslash\lcode{hspace}\textasteriskcentered.

Para modificar a identação característica de um novo parágrafo deve-se utilizar o comando
\begin{latexcode}
    \setlength{\parident}{tam}
\end{latexcode} 
onde \lcode{tam} é o novo tamanho para a identação dos parágrafos. No caso de desejar-se suprimir a identação deve-se utilizar o comando \textbackslash\lcode{noindent}.

O comando \textbackslash\lcode{hfill} cria um espaço suficiente para dividir o texto de modo que o que estiver antes do comando é alinhado a esquerda e o que estiver depois é alinhado a direita. É permitido utilizar o comando mais de uma vez em uma linha. O comando é ignorado quando ocorrer entre duas linhas ou no início de uma linha, neste caso devemos utilizar  \textbackslash\lcode{hfill}\textasteriskcentered.

\subsection{Linha horizontal}
Os comandos \textbackslash\lcode{dotfill} e \textbackslash\lcode{hrulefill} funcionam de maneira semelhante ao comando \textbackslash\lcode{hfill}, mas ao invés de inserir um espaço em branco é introduzido, respectivamente uma linha pontilhada e uma linha contínua.

\subsection{Espaçamento vertical}
No capítulo anterior informamos como mudar de linha, nesta seção vamos trabalhar com o espaço entre as linhas.

O comando \textbackslash\lcode{baselineskip[tam]} estabelece o tamanho do espaçamento entre linhas para o texto posterior ao comando. Para modificar o tamanho entre duas linhas específicas pode-se utilizar o comando \textbackslash\textbackslash\lcode{[tam]} inicia uma nova linha de maneira que \lcode{tam} é o espaçamento entre as linhas.

Para aumentar o espaço entre parágrafos pode-se utilizar os comandos \textbackslash\lcode{smallskip}, \textbackslash\lcode{medskip} ou \textbackslash\lcode{bigskip}, sendo que o tamanho do espaço está relacionado com o tamanho da fonte padrão do documento.

Os comandos \textbackslash\lcode{vspace} e \textbackslash\lcode{vfill} funcionam, respectivamente, de modo muito semelhante aos comandos \textbackslash\lcode{hspace} e \textbackslash\lcode{hfill} só que na vertical.

\section{Alinhamento}
Por padrão, o alinhamento ocorre com a margem esquerda e para alterá-lo pode-se utilizar um dos seguintes ambientes: \lcode{center} (para texto centralizado), \lcode{flushleft} (alinhamento a esquerda) e \lcode{flushright} (alinhamento a direita). \\
\begin{minipage}[t]{0.47\linewidth} \vspace{-8pt}
    \begin{latexcode}
        \begin{flushleft}esquerda
        \end{flushleft} \\
        \begin{center}centralizado
        \end{center} \\
        \begin{flushright}direita
        \end{flushright}
    \end{latexcode}
\end{minipage} \hfill
\begin{minipage}[t]{0.47\linewidth} \vspace{0pt}
    \begin{flushleft}esquerda
    \end{flushleft}
    \begin{center}centralizado
    \end{center}
    \begin{flushright}direita
    \end{flushright}
\end{minipage}

Também é permitido utilizar os comandos: \textbackslash\lcode{centering} (para texto centralizado), \textbackslash\lcode{raggedleft} (alinhamento a esquerda) e \textbackslash\lcode{raggedright} (alinhamento a direita).

\section{Texto em colunas}
Como vimos no Capítulo \ref{sch:latex:preamble}, podemos gerar um documento com duas colunas utilizando a opção \lcode{twocolumn} no comando \textbackslash\lcode{documentoclass}. Nesta seção abordaremos como introduzir duas ou mais colunas em um partes de um documento.

Antes de começar é importante informar que os ambientes a seguir informados para trabalhar com colunas não são $100$\% compatíveis com os comandos e ambientes do LaTeX e apenas o método de tentativa e erro para saber o que funciona.

\subsection{\textbackslash\lcode{twocolumn}}
O comando \textbackslash\lcode{twocolumn[nome]} termina a página atual e inicia uma nova página com duas colunas por página. O argumento opcional \lcode{nome} é escrito no início da página em uma coluna com a largura da página. O comando \textbackslash\lcode{onecolumn} termina o modo de duas colunas.

\subsection{\lcode{multicols}} \nocite{Mitttelbach:2009:Multicolumn}
O ambiente \lcode{multicols}, definido no pacote \lcode{multicols}, apresenta a seguinte sintaxe:
\begin{latexcode}
    \begin{multicols}{number}
        texto
    \end{multicols}
\end{latexcode}
onde \lcode{number} é o número de colunas a serem criadas.

\subsection{\lcode{parcolumns}} \nocite{Sauer:2004:Parcolumns}
O ambiente \lcode{parcolumns}, definido no pacote \lcode{parcolumns}, apresenta sintaxe muito semelhante ao ambiente \lcode{multicols}:
\begin{latexcode}
    \begin{parcolumns}[options]{number}
        texto
    \end{parcolumns}
\end{latexcode}
onde \lcode{number} é o número de colunas a serem criadas e \lcode{options} algumas configurações opcionais (para maiores informações procurar \cite{Sauer:2004:Parcolumns}).

\subsection{\lcode{parallel}}
Ao trabalhar, principalmente, com documentos bilingues é interessante utilizar cada coluna para um idioma e deixar parágrafos correspondentes alinhados. Para isso pode-se utilizar o ambiente \lcode{parallel}, definido no pacote \lcode{parallel}, que apresenta a seguinte sintaxe:
\begin{latexcode}
    \begin{Parallel}{col1}{col2}
        \ParallelLText{...}
        \ParallelRText{...}
        \ParallelPar
    \end{Parallel}
\end{latexcode}
onde \lcode{col1} e \lcode{col2} é a largura das colunas.

O comando \textbackslash\lcode{ParallelLText} tem como parâmetro o texto da coluna da esquerda enquanto que o comando \textbackslash\lcode{ParallelRText} o texto da coluna da direita e o comando \textbackslash\lcode{ParallelPar} indica o alinhamento dos parágrafos.

\subsection{\lcode{minipage}}
O ambiente \lcode{minipage} é responsável por criar pequenas páginas, caixas, ao longo do texto. A síntaxe do ambiente é
\begin{latexcode}
    \begin{minipage}[pos]{largura}
    \end{minipage}
\end{latexcode}
onde \lcode{pos} é a posição do texto que está na minipágina e \lcode{largura} é a largura da minipágina.

O ambiente \lcode{minipage} não apresenta borda, para incluí-la pode-se utilizar o ambiente \lcode{boxedminipage} definido no pacote \lcode{boxedminipage}.

Para simular colunas utilizando-se do ambiente \lcode{minipage} precisa-se apenas utilizar duas ou mais vezes o ambiente em sequência como indicadono código abaixo
\begin{latexcode}
    \begin{minipage}[pos1]{largura1}
        ...
    \end{minipage} espaco
    \begin{minipage}[pos1]{largura1}
        ...
    \end{minipage}
\end{latexcode}
onde \lcode{espaco} é um comando para distanciar uma coluna da outra.

Ao tentar simular colunas com o ambiente \lcode{minipage} pode ocorrer que o texto de uma coluna inicie apenas ao fim do texto da outra coluna. Para resolver este problema deve-se adicionar o comando \textbackslash\lcode{vspace\{0pt\}} no início de cada \lcode{minipage}.

O ambiente \lcode{minipage}, ao contrário dos apresentados anteriormente, é tratado como um objeto flutuante de modo que seu conteudo não pode ser dividido entre páginas requerendo um pouco mais de atenção.
