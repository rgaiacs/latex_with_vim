% Filename: listings@latex_with_vim.tex
% This code is part of LaTeX with Vim.
% 
% Description: LaTeX with Vim is free book about Vim, LaTeX and Git.
% 
% Created: 30.03.12 12:14:04 AM
% Last Change: 30.03.12 12:14:32 AM
% 
% Author: Raniere Gaia Costa da Silva, r.gaia.cs@gmail.com
% Organization:  
% 
% Copyright (c) 2010, 2011, 2012, Raniere Gaia Costa da Silva. All rights 
% reserved.
% 
% This file is license under the terms of a Creative Commons Attribution 
% 3.0 Unported License, or (at your option) any later version. More details
% at <http://creativecommons.org/licenses/by/3.0/>.
\begin{tabular}{llp{9cm}}
    \hline
    Código & Opções & Descrição \\ \hline
    \textsf{language} & \textit{String} & Linguagem utilizada, em \cite{Moses07} encontra-se uma lista de todas as linguagens suportadas. \\ \hline
    \textsf{firstline} & Inteiro & Linha inicial a ser processada. \\ \hline
    \textsf{lastline} & Inteiro & Linha final a ser processada. \\ \hline
    \textsf{basicstyle}& Opções de fonte. & Fonte básica. \\ \hline
    \textsf{keywordstyle} & Opções de fonte. &  Fonte para palavras chaves. \\ \hline
    \textsf{commentstyle} & Opções de fonte. & Fonte para comentários. \\ \hline
    \textsf{stringstyle} & Opções de fonte. & Fonte para \textit{strings}. \\ \hline
    \multirow{2}{*}{\textsf{showstringspaces}} & \textsf{true} & Indica espaço em branco em \textit{strings}. \\
    & \textsf{false} & Não indica espaço em branco em \textit{strings}. \\ \hline
    \multirow{2}{*}{\textsf{showspaces}} & \textsf{true} & Indica espaço em branco. \\
    & \textsf{false} & Não indica espaço em branco. \\ \hline
    \multirow{2}{*}{\textsf{showtabs}} & \textsf{true} & Indica tabulação. \\
    & \textsf{false} & Não indica tabulação. \\ \hline
    \textsf{tab} & Símbolo & Símbolo a ser utilizado para indicar tabulação. \\ \hline
    \multirow{2}{*}{\textsf{numbers}} & \textsf{left} & Numeração de linhas a esquerda. \\
    & \textsf{right} & Numeração de linhas a esquerda. \\ \hline
    \textsf{numberstyle} & Opções de fonte. & Fonte para a numeração da linha. \\ \hline
    \textsf{stepnumber} & Inteiro & Número de linhas entre uma numeração e outra. \\ \hline
    \textsf{numbersep} & Tamanho de espaço & Espaço entre a numeração e a linha. \\ \hline
    \textsf{firstnumber} & Inteiro & Numeração da primeira linha. \\ \hline
    \textsf{aboveskip} & Tamanho de espaço. & Espaço vertical antes do código. \\ \hline
    \textsf{belowskip} & Tamanho de espaço. & Espaço vertical depois do código. \\ \hline
    \textsf{frame} & Buscar em \cite{Moses07} & Borda do código. \\ \hline
    \textsf{frameround} & Buscar em \cite{Moses07} & Aredonda os cantos da borda do código. \\ \hline
    \textsf{backgroundcolor} & Cor & Cor de fundo do código. \\ \hline
    \textsf{caption} &  \textit{String} & Nome do código, precedido do nome do ambiente e numeração. \\ \hline
    \textsf{title} & \textit{String} & Nome do código, não é precedido do nome do ambiente e numeração. \\ \hline
    \textsf{label} & \textit{String} & \textit{String} para referência cruzada. \\ \hline
\end{tabular}
