% Filename: package.tex
% This code is part of 'LaTeX with Vim'.
% 
% Description: This file correspond to the packages to be used.
% 
% Created: 07.06.12 11:30:10 AM
% Last Change: 26.06.12 11:07:21 PM
% 
% Authors:
% - Raniere Silva, r.gaia.cs@gmail.com
% 
% Copyright (c) 2012, Raniere Silva. All rights reserved.
% 
% This work is licensed under the Creative Commons Attribution-ShareAlike 3.0 Unported License. To view a copy of this license, visit http://creativecommons.org/licenses/by-sa/3.0/ or send a letter to Creative Commons, 444 Castro Street, Suite 900, Mountain View, California, 94041, USA.
%
% This work is distributed in the hope that it will be useful, but WITHOUT ANY WARRANTY; without even the implied warranty of MERCHANTABILITY or FITNESS FOR A PARTICULAR PURPOSE.
%
\usepackage[utf8]{inputenc}
\usepackage[T1]{fontenc} 
% \usepackage[top=3cm,left=2cm,right=2cm,bottom=3cm]{geometry}  % Set in the file.
% \usepackage[brazil]{babel}  % Set in the file.
% \usepackage{indentfirst}  % Set in the file.

% Text
\usepackage{enumerate}
\usepackage{latexsym}
\usepackage{parcolumns}
\usepackage[hyphens]{url}
\usepackage{hyperref}
\usepackage{breakurl}
\usepackage[official]{eurosym}

% Tables
\usepackage{multicol}
\usepackage{multirow}
\usepackage{array}

% Math
\usepackage{amsthm}
\usepackage{amsmath}
\usepackage{amsfonts}
\usepackage{amssymb}

% Index
\usepackage{makeidx}
\makeindex

% Figures
\usepackage{pb-diagram}
\usepackage{graphicx, color}
\usepackage{subfig}
\usepackage{tikz}
\usetikzlibrary{patterns}
\usepackage{epsfig}

% Algorithm
\usepackage{algorithmic}
\usepackage{algorithm}
\usepackage{listings}
\usepackage{listingsutf8}

% New commands and enviromments
\newcommand{\TikZ}{Ti\emph{k}Z }
\newcommand{\PGF}{\textsc{PGF} }
\newcommand{\flang}[1]{\textit{#1}}

\lstset{
language=TeX,
basicstyle=\ttfamily,
showspaces=false,
showstringspaces=false,
showtabs=false,
tabsize=2,
breaklines=true,
breakatwhitespace=false,
}
\newcommand{\lcode}[1]{\lstinline!#1!}  % Code in the same line. Deprecated: limitations to handle with backslash and curly brackets. 
\newcommand{\envname}[1]{\lstinline!#1!}  % Name of environments.
\newcommand{\pkgname}[1]{\lstinline!#1!}  % Name of environments.
% Name of commands use \lstinline!\command_name!.
\lstdefinestyle{example}{
language=TeX,
basicstyle=\ttfamily\footnotesize,
showspaces=false,               % show spaces adding particular underscores
showstringspaces=false,         % underline spaces within strings
showtabs=false,                 % show tabs within strings adding particular underscores
tabsize=2,                      % sets default tabsize to 2 spaces
breaklines=true,                % sets automatic line breaking
breakatwhitespace=false,        % sets if automatic breaks should only happen at whitespace
escapeinside={\%}{\^^M},
aboveskip=-12pt,
}
\renewcommand{\lstlistingname}{C\'{o}digo}
\lstnewenvironment{code}[1][style=example,]{}{}  % Code in new line.
\newcommand{\fcode}[1]{\lstinputlisting[style=example,]{#1}}  % Code in a file.
\newcommand{\example}[1]{  % For examples.
\begin{minipage}[c]{0.5\textwidth}
    \fcode{#1}
\end{minipage} \quad \vrule \quad
\begin{minipage}[c]{0.35\textwidth}
    \input{#1}
\end{minipage}
}
\newcommand{\examplebeamer}[1]{  % For examples.
\begin{minipage}[c]{0.5\textwidth}
    \fcode{#1}
\end{minipage} \quad \vrule \quad
\begin{minipage}[c]{0.35\textwidth}
    \fbox{\includegraphics[width=\textwidth]{#1}}
\end{minipage}
}
